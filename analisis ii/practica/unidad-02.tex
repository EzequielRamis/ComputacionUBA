\documentclass[../practica.root.tex]{subfiles}

\begin{document}
\section{Unidad 2}
\begin{enumerate}
    \item Graficar las siguientes curvas en $\R^2$ y, si es posible, encontrar $y = f(x)$
          \begin{enumerate}
              \item $x = 3 - 4t$, $y = 2 - 3t$

                    Encontrar $f(x)$
                    \[
                        \begin{cases}
                            x = 3 - 4t \\
                            y = 2 - 3t
                        \end{cases}
                        \begin{cases}
                            \frac{-x + 3}{4} = t \\
                            y = 2 - 3t
                        \end{cases}
                    \] \[
                        y = 2 - 3\left(\frac{-x +3}{4}\right) = 2 + \frac{3}{4}x - \frac{3}{4} = \boxed{f(x) = \frac{3}{4}x + \frac{5}{4}}
                    \]
                    \begin{tikzpicture}
                        \begin{axis}[anchor=origin,grid=both,xmin=-3,xmax=8,ymin=-1,ymax=5,axis lines=middle]
                            % Use gnuplot
                            % > set parametric
                            % > set trange [-2:2]
                            % > set table "ej-2.1.b.table"
                            % > fx(t) = 3 - 4t
                            % > fy(t) = 2 - 3t
                            % > plot fx(t),fy(t)
                            \addplot plot[mark=,smooth] file {practica/plots/ej-2.1.a.table};
                        \end{axis}
                    \end{tikzpicture}

              \item $x = 1 - t^2$, $y = t - 2$, $-2 \leq t \leq 2$

                    No existe $y = f(x)$ para esta curva, sin embargo existe $x = f(y)$
                    \begin{proof}
                        \[
                            \begin{cases}
                                x = 1 - t^2 \\
                                y = t - 2
                            \end{cases}
                            \begin{cases}
                                \sqrt{-x + 1} = \sqrt{t^2} = |t| \\
                                y = t - 2
                            \end{cases}
                        \]
                        \[
                            \begin{cases}
                                x = 1 - t^2 \\
                                y = t - 2
                            \end{cases}
                            \begin{cases}
                                x = 1 - t^2 \\
                                y + 2 = t
                            \end{cases}
                        \] \[
                            x = 1 - (y + 2)^2 = -(y^2 + 2y + 2y + 2^2) + 1 = -y^2 - 4y -3 = f(y) = -y^2 - 4y -3
                        \]
                    \end{proof}
                    \begin{tikzpicture}
                        \begin{axis}[anchor=origin,grid=both,xmax=1.5,axis lines=middle]
                            % Use gnuplot
                            % > set parametric
                            % > set trange [-2:2]
                            % > set table "ej-2.1.b.table"
                            % > fx(t) = 1 - t*t
                            % > fy(t) = t - 2
                            % > plot fx(t),fy(t)
                            \addplot plot[mark=,smooth] file {practica/plots/ej-2.1.b.table};
                        \end{axis}
                    \end{tikzpicture}

              \item $x = t^2 + t$, $y = t^2 - t$, $-2 \leq t \leq 2$

                    No existe $f(x)$ para esta curva ya que hay varias $y$ para un solo valor de $x$, ej: $(0,0)$ y $(0,2)$

                    \begin{tikzpicture}
                        \begin{axis}[anchor=origin,grid=both,xmin=-1,xmax=6,ymin=-1,ymax=5,axis lines=middle]
                            % Use gnuplot
                            % > set parametric
                            % > set trange [-2:2]
                            % > set table "ej-2.1.c.table"
                            % > fx(t) = t*t + t
                            % > fy(t) = t*t - t
                            % > plot fx(t),fy(t)
                            \addplot plot[mark=,smooth] file {practica/plots/ej-2.1.c.table};
                        \end{axis}
                    \end{tikzpicture}

              \item $x = t^2$, $y = t^3 - 4t$, $-3 \leq t \leq 3$

                    No existe $f(x)$ para esta curva ya que hay varias $y$ para un solo valor de $x$, ej: $(1,3)$ y $(1,-3)$

                    \begin{tikzpicture}
                        \begin{axis}[anchor=origin,grid=both,xmin=-1,xmax=9,ymin=-15,ymax=15,axis lines=middle]
                            % Use gnuplot
                            % > set parametric
                            % > set trange [-3:3]
                            % > set table "ej-2.1.d.table"
                            % > fx(t) = t*t
                            % > fy(t) = t*t*t - 4*t
                            % > plot fx(t),fy(t)
                            \addplot plot[mark=,smooth] file {practica/plots/ej-2.1.d.table};
                        \end{axis}
                    \end{tikzpicture}
          \end{enumerate}
    \item En cada casa, describir de forma paramétrica la circunferencia de radio $r$ y centro $p$
    \item \begin{enumerate}
              \item $r = 2$, $p = (0, 0)$: $x^2 + y^2 = 2^2$
              \item $r = 1$, $p = (1, 3)$: $(x - 1)^2 + (x - 3)^2 = 1$
              \item $r = 3$, $p = (0, 2)$: $x^2 + (y - 2)^2 = 3^2$
          \end{enumerate}
    \item \dots 4
\end{enumerate}
\subsection{Resueltos en clase}
\begin{enumerate}
    \item[11.] $2 < r < 3, \frac{5}{3}\pi \leq \theta \leq \frac{7}{3}\pi$

          \begin{tikzpicture}
              \foreach \t in {0,60,120,180,240,300} {
                      \draw (\t:0) -- (\t:3.2);
                  }
              \draw (60:2) node[rotate=60,anchor=south] {$\frac{1}{3}\pi \equiv \frac{7}{3}\pi$};
              \draw (-60:1.5) node[rotate=-60,anchor=north] {$\frac{5}{3}\pi$};
              \draw[dashed] (-60:2) arc (-60:60:2);
              \draw[dashed] (-60:3) arc (-60:60:3);
              \path[pattern=north east lines] (-60:2) arc (-60:60:2) -- (60:3) arc (60:-60:3) -- cycle;
          \end{tikzpicture}

    \item[13.] Encontrar la distancia entre $P_1 = (2,\pi/3)$ y $P_2 = (4, 2\pi/3)$

          \begin{tikzpicture}
              \node (1) at (60:2) {$P_1$};
              \node (2) at (120:4) {$P_2$};
              \draw (-3.2,0) -- (3.2,0);
              \draw (0,-3.2) -- (0,3.2);
              \begin{scope}[->]
                  \draw (0,0) -- (1) node[pos=0.5,anchor=west] {$a$};
                  \draw (0,0) -- (2) node[pos=0.5,anchor=east] {$b$};
                  \draw[|-|,thick,red] (1) -- (2) node[pos=0.5,anchor=south]{$c$};
              \end{scope}
              \draw (60:0.5) arc (60:120:0.5) node[pos=0.5,anchor=south] {$\theta$};
          \end{tikzpicture}
          \[ \theta = \left\|\frac{1}{3}\pi  - \frac{2}{3}\pi\right\| = \pi/3 \]
          \[ \text{Teorema del coseno: } c^2 = a^2 + b^2 - 2 \cdot a \cdot b \cdot \cos\theta \]
          \begin{align*}
              c^2 & = 2^2 + 4^2 - 2 \cdot 2 \cdot 4 \cos \pi/3 \\
                  & = 4 + 16 - 8                               \\
                  & = 12 \\
              c   & = \sqrt{12}
          \end{align*}
          \[ \boxed{c = 2\sqrt{3}} \]

    \item[15.] $r^2 = 5$
          \[ r^2 = \boxed{x^2 + y^2 = 5} \]

    \item[17.] $r = 2\cos\theta$
          \[ \cos\theta = \frac{x}{r} \]
          \[ r^2 = 2x \]
          \[ r^2 = x^2 + y^2 \]
          \[ x^2 - 2x + y^2 = 0 \]
          \[ x^2 - 2x + 1 + y^2 = 1 \]
          \[ \boxed{(x-1)^2 + y^2 = 1} \]

    \item[18.] $\theta = \frac{\pi}{3}$
          \[ \tan\theta = \frac{y}{x} = \sqrt{3} \]
          \[ \boxed{y = \sqrt{3}x} \]

\end{enumerate}
\end{document}