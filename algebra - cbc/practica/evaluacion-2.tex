\documentclass[../practica.root.tex]{subfiles}

\begin{document}

\section{Segunda Evaluación}
\begin{enumerate}
    \item Sea $B = \{v_1,v_2,v_3\}$ una base de $R^3$ y sea $f : R^3 \to R^3$ la transformación lineal tal que $f(v_1) = 3v_1 + kv_3$, $f(v_2) = v_1 + 2v_2$, $f(v_3) = v_1 + v_2 - 5v_3$. Encontrar el valor de $k$ tal que 4 sea autovalor de $f$
          \[
              f(B) = \begin{pmatrix}
                  3 & 0 & k  \\
                  1 & 2 & 0  \\
                  1 & 1 & -5
              \end{pmatrix}
          \] \[
              P(\lambda) = -\lambda^3 + k\lambda + 19\lambda - k - 30
          \] \[
              -(4)^3 + k(4) + 19(4) - k - 30 = 0
          \] \[
              -64 + 4k + 76 - k - 30 = -18 + 3k = 0 \implies \boxed{k = 6}
          \]

    \item Sean $S = \langle(1,-1,1,-1);(0,-1,1,2);(3,-4,4,-1)\rangle$ y $H = \{x \in R^4 / 3x_1 - x_2 - 3x_3 + x_4 = 0\}$. Una base de $R^4$ que contiene una base de $S$ y una base de $H$ es
%          \[
%              \begin{pmatrix}
%                  1 & -1 & 1 & -1 \\
%                  0 & -1 & 1 & 2  \\
%                  3 & -4 & 4 & -1
%              \end{pmatrix}
%              \begin{pmatrix}
%                  1 & -1 & 1 & -1 \\
%                  0 & -1 & 1 & 2  \\
%                  0 & -7 & 1 & 2
%              \end{pmatrix}
%          \] \[
%              3x_1 - x_2 - 3x_3 + x_4 = 0 \implies x_2 = 3x_1 - 3x_3 + x_4
%          \] \[
%              (x_1, x_2, x_3, x_4) = (x_1, 3x_1 - 3x_3 + x_4, x_3, x_4)
%          \] \[
%              = x_1(1,3,0,0) + x_3(0,-3,1,0) + x_4(0,1,0,1)
%          \] \[
%              \H = \langle(1,3,0,0),(0,-3,1,0),(0,1,0,1)\rangle
%          \]

    \item Sean $B = \{v_1, v_2, v_3\}$ y $B' = \{v_1 + v_2, v_1 - v_2 - v_3, v_1 + v_3\}$ bases de un espacio vectorial $V$ y sea $f:V \to V$ la transformación lineal tal que $M_{B'B}(f) = \begin{pmatrix}
              1 & 1 & 1 & \\ -1 & -1 & 2 \\ 2 & 2 & -1
          \end{pmatrix}$. Encontrar todos los $v \in V$ tales que $f(v) = f(-v_1 + 2v_2 + v_3)$
%          \[
%              M(f) = C_{BE} \cdot M_{B'B}(f) \cdot C_{EB'}
%          \]
%          Ya que $B$ está definido en los vectores $\{v_1, v_2 v_3\}$, y estos resultan ser la "minima expresion" con la que podemos representar los vectores del espacio vectorial $V$, podemos decir que $B = E$ y que $\{v_1, v_2 v_3\}$ son los vectores unitarios
%          \[ C_{BE} = C_{EE} = I \]
%          \[
%              C_{EB'} = (C_{B'E})^{-1} = \frac{1}{3}\begin{pmatrix}
%                  1 & 2  & -1 \\
%                  1 & -1 & -1 \\
%                  1 & -1 & 2  \\
%              \end{pmatrix}
%          \] \[
%              C_{B'E} = \begin{pmatrix}
%                  1 & 1  & 1 \\
%                  1 & -1 & 0 \\
%                  0 & -1 & 1 \\
%              \end{pmatrix}
%          \] \[
%              M(f) = I \cdot \begin{pmatrix}
%                  1 & 1 & 1 & \\ -1 & -1 & 2 \\ 2 & 2 & -1
%              \end{pmatrix} \cdot \frac{1}{3}\begin{pmatrix}
%                  1 & 2  & -1 \\
%                  1 & -1 & -1 \\
%                  1 & -1 & 2  \\
%              \end{pmatrix} = \begin{pmatrix}
%                  1 & 0  & 0  \\
%                  0 & -1 & 2  \\
%                  1 & 1  & -2 \\
%              \end{pmatrix}
%          \] \[
%              -v_1 + 2v_2 + v_3 = (-1, 2, 1)
%          \] \[
%              f(-v_1 + 2v_2 + v_3) = f((-1, 2, 1)) = \begin{pmatrix}
%                  1 & 0  & 0  \\
%                  0 & -1 & 2  \\
%                  1 & 1  & -2
%              \end{pmatrix}\cdot\begin{pmatrix}
%                  -1 \\ 2 \\ 1
%              \end{pmatrix} = (-1,0,-1)
%          \] \[
%              f(v) = (-1,0,-1)
%          \] \[
%              \begin{amatrix}{3}
%                  1 &  0 &  0 & -1 \\
%                  0 & -1 &  2 & 0 \\
%                  1 &  1 & -2 & -1
%              \end{amatrix}
%              =
%              \begin{amatrix}{3}
%                  1 & 0 & 0  & -1 \\
%                  0 & 1 & -2 & 0 \\
%                  0 & 0 & 0  & 0 \\
%              \end{amatrix}
%          \] \[
%              v_1 = -1, v_2 = 2v_3, 0 = 0
%          \] \[
%              (-1,2v_3,v_3) = (-1,0,0) + v_3(0,2,1)
%          \]
%          Mal?????


    \item Sea $f : R^4 \to R^4$ la transformación lineal $f(x) = (2x_1 + x_2, x_2 - x_3 + x_4, x_2 + x_3, 2x_1 + 3x_2 + x_4)$. Si $g : R^4 \to R^4$ es una transformación lineal tal que $\Nu(g) = \Img(f)$ y $\Img(g) = \Nu(f)$, encontrar los valores de $a$ y $b$ tales que $g(1, 2, 1, a) = 0$ y $(1, b, 2, 4) \in \Img(g)$.

    \item Sea $f : R^4 \to R^4$ la t. l. $f(x) = (2x_1+x_2, x_2-x_3+x_4, x_2+x_3, 2x_1+3x_2+x_4)$. Si $g : R^4 \to R^4$ es una t. l. tal que $\Nu(g) = \Img(f)$ y $\Img(g) = \Nu(f)$, los valores de $a$ y $b$ tales que $g(1,2,1,a) = 0$ y $(1,b,2,4) \in \Img(g)$ son\dots
          \[ \Nu f = \{ v \in \V / f(v) = \0 \} \]
%          \[
%              \begin{cases}
%                  2x_1+x_2 = 0      \\
%                  x_2-x_3+x_4 = 0   \\
%                  x_2+x_3 = 0       \\
%                  2x_1+3x_2+x_4 = 0 \\
%              \end{cases}
%              =
%              \begin{cases}
%                  -2x_1 = x_2  \\
%                  +x_4 = 2x_3  \\
%                  -2x_1 = -x_3 \\
%                  0 = 0        \\
%              \end{cases}
%          \] \[
%              (x_1, x_2, x_3, x_4) = (x_1, -2x_1, -2x_1, x_4) = x_1(1,-2,-2,0)+x_4(0,0,0,1)
%          \]

    \item Sea $B = \{(1,0,0), (-1,1,0), (-1,-1,1)\}$ una base de $\R^3$ y $\S = \{x \in \R^3 / x_1 - 2x_2 - x_3 = 0\}$. Si $v$ y $w \in \R^3$ satisfacen $(v + w)_B = (1,3,-1)$ y $(v - w)_B = (-2,3,5)$, calcular $\langle v, w\rangle \cap \S$

    \item Sea $f : \R^4 \to \R^4$, la t. l. definida por $f(x) = (x_1 - x_3, x_3 - x_4, x_1 + 2x_2 - x_4, x_1 - x_4)$ y sea $\H = \{x \in \R^4 / x_1 + x_3 - 2x_4 = 0\}$. Calcular $\Nu(f \circ f)\cap f^{-1}(\H)$

    \item Sea $B = \{(1,2,-2), (0,1,-1), (1,-3,2)\}$ base de $\R^3$. Si $v \in \R^3$ es tal que $(v)_B = (-1,2,-3)$, calcular $4v + (-2,1,1)_B$

    \item Sea $\S = \langle(3,4,a,9)\rangle$, $\T = \langle(3,1,0,0),(3,4,3,a^2)\rangle$ y $\H = \{x \in \R^4 / x_1 - 3x_2 + x_4 = 0\}$. Calcular el conjunto $a \in \R$ tal que $\S \oplus \T = \H$

    \item Sea $f : \R^3 \to \R^3$ la t. l. tal que $M_{BE}(f) = \begin{pmatrix}
              0 & 1 & 1 \\ -2 & a & 1 \\ 0 & 3 & -1
          \end{pmatrix}$ con $B = \{(0,1,0),(1,-1,0),(0,0,1)\}$. Calcular el conjunto $a \in \R$ tal que $f$ es diagonalizable
%          \[
%              M_{EE}(f) = C_{BE}\cdot M_{B'B}(f) \cdot C_{EB'}
%          \]
\end{enumerate}
\end{document}