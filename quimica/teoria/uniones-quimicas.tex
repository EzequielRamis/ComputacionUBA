\documentclass[../teoria.root.tex]{subfiles}

\begin{document}

\section{Uniones Químicas}
Las uniones químicas son fuerzas entre átomos o moléculas que los mantienen unidos entre sí, formando un compuesto químico.

\subsection{Polaridad}
La polaridad de una molécula se determina "dibujando" vectores que muestren los momentos dipolares entre las uniones atómicas, y luego sumándolos (Teniendo en cuenta la \hyperref[sec:trepev]{GM})

\subsection{Iónicas}
Una unión iónica es un tipo particularmente fuerte unión, y se da cuando la diferencia de electronegatividad es mayor o igual a 2. Debido a dicha fuerza, los compuestos iónicos van a tener puntos de fusión y ebullición bastantes altos.

\subsection{Covalentes}
Una unión covalente se da cuando la diferencia de electronegatividad entre dos átomos es menor a 2

\subsubsection{Simples y Múltiples}
% TODO: Hacer esto

\subsubsection{Dativa o Coordenada}
% TODO: Averiguar que es esto (?)

\end{document}
