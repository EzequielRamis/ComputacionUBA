\documentclass[../teoria.root.tex]{subfiles}

\begin{document}
\section{Uniones moleculares}
\subsection{Fuerzas intermoleculares}
Existen tres tipos de fuerzas intermoleculares, cada uno más intenso que el anterior, de forma creciente:
\begin{itemize}
	\item Fuerza de London
	\item Fuerza dipolo-dipolo
	\item Puente de hidrógeno
\end{itemize}
\subsubsection{Fuerza de London}
Características:
\begin{itemize}
	\item Presente en prácticamente cualquier molécula
	\item La menos significativa de las tres
	\item Aumenta su intensidad a mayor masa molar, debido a que las moléculas de mayor masa molar tienen más electrones
\end{itemize}
\subsubsection{Fuerza dipolo-dipolo}
Características:
\begin{itemize}
	\item Presente en moléculas polares
	\item Aumenta su intensidad a mayor momento dipolar
\end{itemize}
\subsubsection{Puente de hidrógeno}
Características:
\begin{itemize}
	\item Es un tipo de fuerza dipolo-dipolo particular
	\item Presente entre una molécula, con hidrógeno y un átomo electronegativo (N, O y F), y el par electrónico sin compartir de un átomo electronegativo de otra molécula
	\item La más significativa de las tres
\end{itemize}
\end{document}
