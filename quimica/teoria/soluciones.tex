\documentclass[../Teoría.root.tex]{subfiles}
 
\begin{document}

\section{Soluciones}
Una solución (Abreviado "sc") es un compuesto químico conformado por un soluto ("st") disuelto en un solvente ("sv"). Este tipo de reacción se representa con la siguiente ecuación:

\begin{equation}
    st + sv \rightarrow sc
\end{equation}

\subsection{Disolución y Polaridad}
Un soluto se disolverá mejor en un solvente de su misma polaridad. Por ejemplo: El \ch{NaCl}, un soluto muy polar, se disuelve particularmente bien en el \ch{H2O}, un solvente muy polar

\subsection{Concentración}
Existen las siguientes formas para representar las concentraciones:

\begin{equation} \label{eq:concentracion.m/m}
    \%m / m = \frac{\si{\g}\ st}{\si{\g}\ sc} \cdot 100
\end{equation}

\begin{equation} \label{eq:concentracion.v/v}
    \%v / v = \frac{\si{\mL}\ st}{\si{\mL}\ sc} \cdot 100
\end{equation}

\begin{equation} \label{eq:concentracion.m/v}
    \%m / v = \frac{\si{\g}\ st}{\si{\mL}\ sc} \cdot 100
\end{equation}

Molalidad:
\begin{equation} \label{eq:concentracion.molalidad}
    m = \frac{\si{\mol}\ st}{\si{\kg}\ sv} 
\end{equation}

Molaridad:
\begin{equation} \label{eq:concentracion.molaridad}
    M = \frac{\si{\mol}\ st}{\si{\L}\ sc} = \frac{\si{\mol}\ st}{\si{\dm\cubed}\ sc}
\end{equation}

Fracción molar soluto:
\begin{equation} \label{eq:concentracion.frac soluto}
    Xst = \frac{\si{\mol}\ st}{\si{\mol}\ totales}
\end{equation}

Fracción molar solvente:
\begin{equation} \label{eq:concentracion.frac solvente}
    Xsv = \frac{\si{\mol}\ sv}{\si{\mol}\ totales}
\end{equation}

Partes por millón:
\begin{equation} \label{eq:concentracion.ppm}
    ppm = \frac{\si{\mg}\ st}{\si{\L}\ sc} = \frac{\si{\mg}\ st}{\si{\kg}\ sc}
\end{equation}

\subsection{Ejemplos}
% TODO: Añadir ejemplos de soluciones y ecuaciones de concentraciones

\end{document}
