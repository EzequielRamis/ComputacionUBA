\documentclass[../Teoría.root.tex]{subfiles}
 
\begin{document}

\section{Palabras Clave}
\label{sec:palabras clave}

\begin{enumerate}
    \item Ion: Átomo o molecula con una carga diferente a 0
    \begin{enumerate}
        \item Anión: Ion de carga negativa
        \item Catión: Ion de carga positiva
        \item Valencia: Valor de la carga del ion.
    \end{enumerate}
    Ejemplos: 
    \begin{enumerate}
        \item El \ch{H+} es un \ii{Catión hidrogeno monovalente}
        \item El \ch{Cu^{2+}} es un \ii{Catión cobre divalente}
    \end{enumerate}

    \item Electronegatividad: Tendencia a atraer electrones
    
    \item Nº de oxidación: Carga electrica con la que actua un átomo
    
    \item Afijos para nombres:
    \begin{enumerate}
        \item -uro / hídrico
        \item Hipo- -oso/ito
        \item -oso/ito
        \item -ico/ato
        \item Per- ico/ato
    \end{enumerate}
    Los elementos con \num{4} Nº de oxidación usan los afijos \ii{b}-\ii{e}, mientras los que tienen \num{2} solo usan \ii{c} y \ii{d}. El \ii{a} es solo para cuando un elemento usa el Nº de oxidación \num{-1} % TODO: Mejorar explicación. Hay elementos con 3 Nº de ox.
    % TODO: Añadir ejemplos

    \item Numeral de Stock: Nº de oxidación del metal en una molecula ionica, expresado con numeros romanos en el nombre del compuesto.
    
    Ejemplos:
    \begin{enumerate}
        \item \ch{Cu^{2+}}: Ion cobre (II)
    \end{enumerate}

    \item Vector (momento) dipolar: Vector en dirección al elemento menos electronegativo
    
    \item Los cambios en el número de oxidación tienen dos nombres:
    \begin{enumerate}
        \item Oxidar: Aumentar el número de oxidación
        \item Reducir: Disminuir el número de oxidación
    \end{enumerate}

\end{enumerate}

\end{document}