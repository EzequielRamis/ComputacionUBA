\documentclass[../Práctica.root.tex]{subfiles}
 
% R = \frac{\SI{22,4}{\atm\dm\cubed}}{\SI{273,15}{\mole\kelvin}}

\begin{document}

\section{Unidad 6}
\subsection{Bloque 1}
\begin{enumerate}
    \item[1.] Un recipiente de tapa móvil contiene \SI{1,00}{\dm\cubed} de oxígeno gaseoso, a \SI{1520}{\torr} y a
          \SI{30,0}{\celsius}. Calculen la presión que ejercerá esa cantidad de oxígeno si el volumen se reduce
          hasta \SI{200}{\centi\m\cubed} y la temperatura a \SI{-20,0}{\celsius}.
          \begin{center}
              \[ PV = nRT \]
              \[ \frac{PV}{T} = nR \]
              \begin{tabular}{ l l l }
                  $ P_i = \SI{1520}{\torr} $ & $ V_i = \SI{1,00}{\dm\cubed} $  & $ T_i = \SI{30,0}{\celsius} $  \\
                  $ P_f = x $                & $ V_f = \SI{0,200}{\dm\cubed} $ & $ T_f = \SI{-20,0}{\celsius} $ \\
              \end{tabular}
              \[ \frac{P_i\cdot V_i}{T_i} = \frac{x\cdot V_f}{T_f} = nR \]
              \[
                  \frac{\SI{1520}{\torr}\cdot \SI{1,00}{\cancel\dm\cubed}}{\SI{30,0}{\cancel\celsius}}
                  = \frac{x\cdot \SI{0,200}{\cancel\dm\cubed}}{\SI{-20,0}{\cancel\celsius}}
              \]
              \[
                  \frac{\SI{1520}{\torr}}{\num{30,0}}
                  = x\cdot\num{-0,01}
              \]
              \[ \frac{\SI{1520}{\torr}}{\num{30,0}\cdot\num{-0,01}} = x \]
              \[ \boxed{x = \SI{5066,7}{\torr}} \]
          \end{center}

    \item[2.] Indiquen cuál es el volumen que ocuparán \num{2,40} moles de una sustancia en estado gaseoso a
          \SI{127}{\celsius} y a presión normal.
          \begin{center}
              \[ \SI{127}{\celsius} = \SI{400,15}{\kelvin} \]
              \[ PV = nRT \]
              \[
                  \SI{1}{\atm}\cdot x
                  = \SI{2,40}{\mole}
                  \cdot\frac{\SI{22,4}{\atm\dm\cubed}}{\SI{273,15}{\mole\kelvin}}
                  \cdot\SI{400,15}{\kelvin}
              \]
              \[
                  \cancel{\SI{1}{\atm}\cdot} x
                  = \SI{2,40}{\cancel\mole}
                  \cdot\frac{\SI{22,4}{\cancel\atm\dm\cubed}}{\SI{273,15}{\cancel\mole\cancel\kelvin}}
                  \cdot\SI{400,15}{\cancel\kelvin}
              \]
              \[ \boxed{x = \SI{78,8}{\dm\cubed}} \]
          \end{center}

    \item[4.] Calculen el volumen molar de un gas ideal, a \SI{27,0}{\celsius} y a una presión de \SI{2,00}{\atm}.
          \begin{center}
              \[ \SI{27,0}{\celsius} = \SI{260,15}{\kelvin} \]
              \[ PV = nRT \]
              \[
                  \SI{2,00}{\atm}\cdot x
                  = \SI{1}{\mole}
                  \cdot\frac{\SI{22,4}{\atm\dm\cubed}}{\SI{273,15}{\mole\kelvin}}
                  \cdot\SI{260,15}{\kelvin}
              \]
              \[
                  \SI{2,00}{\cancel\atm}\cdot x
                  = \cancel{\SI{1}{\mole}}
                  \cdot\frac{\SI{22,4}{\cancel\atm\dm\cubed}}{\SI{273,15}{\cancel\mole\cancel\kelvin}}
                  \cdot\SI{260,15}{\cancel\kelvin}
              \]
              \[ x = \frac{\SI{22,4}{\dm\cubed}\cdot\num{260,15}}{\num{273,15}\cdot\num{2}} \]
              \[ x = \frac{\SI{6723,4}{\dm\cubed}}{\num{546,30}} \]
              \[ \boxed{x = \SI{12,3}{\dm\cubed}} \]
          \end{center}

    \item[8.] Se dispone de un cilindro de \SI{100}{\liter} que contiene $N_2$ (g), a una presión de \SI{5,00}{\atm} y a una
          temperatura de \SI{22,0}{\celsius}. Calculen: \\
          $ \SI{100}{\liter} = \SI{100}{\dm\cubed}; \SI{22,0}{\celsius} = \SI{295,15}{\kelvin} $
          \begin{enumerate}
              \item la masa de nitrógeno en el recipiente;
                    \begin{center}
                        \[ PV = nRT \]
                        \[
                            \SI{5,00}{\atm}\cdot\SI{100}{\dm\cubed}
                            = n
                            \cdot \frac{\SI{22,4}{\atm\dm\cubed}}{\SI{273,15}{\mole\kelvin}}
                            \cdot \SI{295,15}{\kelvin}
                        \]
                        \[
                            \SI{5,00}{\cancel\atm}\cdot\SI{100}{\cancel\dm\cubed}
                            = n
                            \cdot \frac{\SI{22,4}{\cancel\atm\cancel\dm\cubed}}{\SI{273,15}{\mole\cancel\kelvin}}
                            \cdot \SI{295,15}{\cancel\kelvin}
                        \]
                        \[
                            \num{500}
                            = n
                            \cdot \SI{24,2}{\per\mole}
                        \]
                        \[ \num{500}/\SI{24,2}{\mole} = n \]
                        \[ n = \SI{20,5}{\mole} \]
                        \begin{tabular}{ c | c }
                            moles de $N_2$ & gramos \\
                            \hline
                            1              & 28     \\
                            \num{20,5}     & x
                        \end{tabular}
                        \[ \SI{28}{\g}\cdot\num{20,5} \]
                        \[ \boxed{\SI{574}{\g}} \]
                    \end{center}
              \item el volumen que ocuparía esa cantidad de gas si a temperatura constante, la presión se
                    reduce \num{5} veces.
                    \begin{center}
                        \[
                            \SI{1,00}{\atm}\cdot V
                            = \SI{20,5}{\mole}
                            \cdot \frac{\SI{22,4}{\atm\dm\cubed}}{\SI{273,15}{\mole\kelvin}}
                            \cdot \SI{295,15}{\kelvin}
                        \]
                        \[
                            \cancel{\SI{1,00}{\atm}}\cdot V
                            = \SI{20,5}{\cancel\mole}
                            \cdot \frac{\SI{22,4}{\cancel\atm\dm\cubed}}{\SI{273,15}{\cancel\mole\cancel\kelvin}}
                            \cdot \SI{295,15}{\cancel\kelvin}
                        \]
                        \[ \boxed{V = \SI{500}{\dm\cubed}} \]
                    \end{center}
          \end{enumerate}

    \item[20.] Un recipiente rígido de \SI{10,0}{\dm\cubed} contiene cierta masa de $CO_2$ (g) en CNPT. Se agrega $CO$ (g)
          hasta que la masa de la mezcla de gases es de \SI{60,0}{\gram}. Se produce una variación de la
          temperatura y un aumento en la presión de \SI{2,5}{\atm}. Indiquen:
          \begin{center}
              \[ \text{I: } \SI{10,0}{\dm\cubed}\cdot\SI{1}{\atm} = n_iR\cdot\SI{273,15}{\kelvin} \]
              \[ \text{F: } \SI{10,0}{\dm\cubed}\cdot\SI{3,5}{\atm} = \SI{60,0}{\gram}RT_f \]
              \[ \SI{10,0}{\dm\cubed} \cdot \SI{1}{\atm}
                  = n_i
                  \cdot \frac{\SI{22,4}{\atm\dm\cubed}}{\SI{273,15}{\mole\kelvin}}
                  \cdot \SI{273,15}{\kelvin}
              \]
              \[ \SI{10,0}{\cancel\dm\cubed} \cdot \cancel{\SI{1}{\atm}}
                  = n_i
                  \cdot \frac{\SI{22,4}{\cancel\atm\cancel\dm\cubed}}{\cancel{\SI{273,15}{\kelvin}}\si{\mole}}
                  \cdot \cancel{\SI{273,15}{\kelvin}}
              \]
              \[ \num{10} = n_i \cdot \SI{22,4}{\per\mole} \]
              \[ n_i = \SI{10/22,4}{\mole} \]
              \[ \text{Moles de $CO_2$: } n_i = \SI{0,45}{\mole} \]
              \begin{tabular}{ c | c }
                  moles      & Gramos $CO_2$ \\
                  \hline
                  1          & 44            \\
                  \num{0,45} & \num{19,8}
              \end{tabular}
              \[ \SI{60}{\g} - \SI{19,8}{\g} = \SI{40,2}{\g} \text{ de $CO$} \]
              \begin{tabular}{ c | c }
                  moles     & Gramos de $CO$ \\
                  \hline
                  1         & 28             \\
                  \num{1,4} & \num{40,2}
              \end{tabular}
          \end{center}
          \begin{enumerate}
              \item la temperatura final que alcanza el sistema;
                    \begin{center}
                        \[ \SI{10,0}{\dm\cubed}\cdot\SI{3,5}{\atm} = \SI{60,0}{\gram}RT_f \]
                        \[
                            \SI{10,0}{\dm\cubed}\cdot\SI{3,5}{\atm}
                            = (\num{1,4} + \num{0,45})\si{\mole}
                            \cdot\frac{\SI{22,4}{\atm\dm\cubed}}{\SI{273,15}{\mole\kelvin}}
                            \cdot T_f
                        \]
                        \[
                            \SI{10,0}{\cancel\dm\cubed}\cdot\SI{3,5}{\cancel\atm}
                            = \num{1,85}\si{\cancel\mole}
                            \cdot\frac{\SI{22,4}{\cancel\atm\cancel\dm\cubed}}{\SI{273,15}{\cancel\mole\kelvin}}
                            \cdot T_f
                        \]
                        \[
                            \SI{10,0}{\cancel\dm\cubed}\cdot\SI{3,5}{\cancel\atm}
                            = \num{1,85}\si{\cancel\mole}
                            \cdot\frac{\SI{22,4}{\cancel\atm\cancel\dm\cubed}}{\SI{273,15}{\cancel\mole\kelvin}}
                            \cdot T_f
                        \]
                        \[ \boxed{T_f = \SI{230}{\kelvin}} \]
                    \end{center}
              \item si la presión parcial del dióxido de carbono en la mezcla es mayor, igual o menor que
                    la del monóxido de carbono;
                    \begin{center}
                        \[ \SI{10,0}{\dm\cubed}\cdot P_{CO_2} = \SI{0,45}{\mole}\cdot R\cdot\SI{230}{\kelvin} \]
                        \[ P_{CO_2} = \SI{0,9}{\dm\cubed} \]
                        \[ P_{CO} = \SI{2,6}{\dm\cubed} \]
                    \end{center}
              \item el número de átomos de oxígeno que hay en la mezcla;
                    \begin{center}
                        \begin{tabular}{ l r }
                            $CO_2$ & \SI{0,45}{\mole} \\
                            $CO$   & \SI{1,4}{\mole}  \\
                            $O$    & \SI{2,3}{\mole}
                        \end{tabular}
                        \[ \SI{2,3}{\mole} = \num{2,3}\cdot\num{6,023E23} \]
                        \[ \boxed{\num{1,4E24}} \]
                    \end{center}
              \item si la temperatura final alcanzada aumenta, disminuye o no cambia, si en lugar de CO
                    se hubiera agregado $O_2$ (g) hasta tener la misma masa final de \SI{60,0}{\gram} y el mismo
                    aumento de presión. Justifiquen la respuesta.
                    \begin{center}
                        \begin{tabular}{ c | c }
                            moles     & Gramos de $O_2$ \\
                            \hline
                            1         & 32              \\
                            \num{1,3} & \num{40,2}
                        \end{tabular}
                        \[
                            \SI{10,0}{\dm\cubed}\cdot\SI{3,5}{\atm}
                            = (\num{1,3} + \num{0,45})\si{\mole}
                            \cdot\frac{\SI{22,4}{\atm\dm\cubed}}{\SI{273,15}{\mole\kelvin}}
                            \cdot T_f
                        \]
                        \[
                            \SI{10,0}{\cancel\dm\cubed}\cdot\SI{3,5}{\cancel\atm}
                            = \SI{1,75}{\cancel\mole}
                            \cdot\frac{\SI{22,4}{\cancel\atm\cancel\dm\cubed}}{\SI{273,15}{\cancel\mole\kelvin}}
                            \cdot T_f
                        \]
                        \[ \num{35} = \SI{1,75}\cdot\frac{\num{22,4}}{\SI{273,15}{\kelvin}}\cdot T_f \]
                        \[ \boxed{\SI{243}{\kelvin}} \]]
                    \end{center}
          \end{enumerate}
\end{enumerate}
\subsection{Bloque 2}
\begin{enumerate}
    \item[3.] Un recipiente de volumen variable contiene \SI{10,0}{\liter} de $C_4H_{10}$ gaseoso a \SI{600}{\torr} y a \SI{5,00}{\celsius}.
          Se calienta el sistema hasta que se verifica que la presión y el volumen se duplican.
          Calculen: \\
          $ \SI{5,00}{\celsius} = \SI{278,15}{\kelvin}; \SI{10,0}{\liter} = \SI{10,0}{\dm\cubed}; \SI{600}{\torr} = \SI{0,7895}{\atm} $
          \begin{enumerate}
              \item la temperatura final del sistema;
                    \begin{center}
                        \[ \frac{P_i\cdot V_i}{\SI{278,15}{\kelvin}}
                            = \frac{\num{2}\cdot P_i\cdot \num{2}\cdot V_i}{T_f}  \]
                        \[ \frac{\cancel{P_i\cdot V_i}}{\SI{278,15}{\kelvin}}
                            = \frac{\num{4}\cdot\cancel{P_i\cdot V_i}}{T_f}  \]
                        \[ T_f = \num{4}\cdot\SI{278,15}{\kelvin} \]
                        \[ \boxed{T_f = \SI{1112,6}{\kelvin}} \]
                    \end{center}
              \item la masa de $C_4H_{10}$ en el recipiente
                    \begin{center}
                        \[ PV = nRT \]
                        \[
                            \SI{0,7895}{\atm}\cdot\SI{10,0}{\dm\cubed}
                            = n
                            \cdot\frac{\SI{22,4}{\atm\dm\cubed}}{\SI{273,15}{\mole\kelvin}}
                            \cdot\SI{278,15}{\kelvin}
                        \]
                        \[
                            \SI{0,7895}{\cancel\atm}\cdot\SI{10,0}{\cancel\dm\cubed}
                            = n
                            \cdot\frac{\SI{22,4}{\cancel\atm\cancel\dm\cubed}}{\SI{273,15}{\mole\cancel\kelvin}}
                            \cdot\SI{278,15}{\cancel\kelvin}
                        \]
                        \[
                            \num{0,7895}\cdot\num{10,0}
                            = n
                            \cdot\frac{\num{22,4}}{\SI{273,15}{\mole}}
                            \cdot\num{278,15}
                        \]
                        \[
                            n =
                            \frac{
                                \num{0,7895}\cdot\num{10,0}\cdot\SI{273,15}{\mole}
                            }{
                                \num{22,4}\cdot\num{278,15}
                            }
                        \]
                        \[ n = \SI{0,346}{\mole} \]
                        \begin{tabular}{ c | c }
                            moles       & Gramos de $C_4H_{10}$ \\
                            \hline
                            1           & 58                    \\
                            \num{0,346} & \num{20,068}
                        \end{tabular}
                        \[ \boxed{ \text{Gramos de $C_4H_{10}$:} \num{20,068}} \]
                    \end{center}
          \end{enumerate}
    \item[11.] La densidad de una sustancia en estado gaseoso, a \SI{25,0}{\celsius} y a \SI{760}{\torr}, es de
          \SI{1,80}{\g\per\dm\cubed}. Calculen: \\
          $ \SI{25,0}{\celsius} = \SI{298,15}{\kelvin}; \SI{760}{\torr} = \SI{1}{\atm} $
          \begin{enumerate}
              \item la masa molar de la sustancia;
                    \begin{center}
                        \[ M = \frac{\rho RT}{P} \]
                        \[
                            M
                            = \frac{
                                \SI{1,80}{\g\per\dm\cubed}
                                \cdot \frac{\SI{22,4}{\atm\dm\cubed}}
                                {\SI{273,15}{\mole\kelvin}}
                                \cdot\SI{298,15}{\kelvin}
                            }
                            {
                                \SI{1}{\atm}
                            }
                        \]
                        \[
                            M
                            = \frac{
                                \SI{1,80}{\g\per\cancel\dm\cubed}
                                \cdot \frac{\SI{22,4}{\cancel\atm\cancel\dm\cubed}}
                                {\SI{273,15}{\mole\cancel\kelvin}}
                                \cdot\SI{298,15}{\cancel\kelvin}
                            }
                            {
                                \cancel{\SI{1}{\atm}}
                            }
                        \]
                        \[
                            M = \num{1,80}\cdot\num{22,4/273,15}\cdot\num{298,15}\si[]{\g\per\mole}
                        \]
                        \[ \boxed{M = \SI{44,0}{\g\per\mole}} \]
                    \end{center}
              \item la densidad del gas en CNPT.
                    \begin{center}
                        \[ \rho = \frac{PM}{RT} \]
                        \[
                            \rho = \frac{
                                \SI{1}{\atm}\cdot\SI{44,0}{\g\per\mole}
                            }{
                                \frac{\SI{22,4}{\atm\dm\cubed}}{\SI{273,15}{\mole\kelvin}} \cdot\SI{273,15}{\kelvin}
                            }
                        \]
                        \[
                            \rho = \frac{
                                \cancel{\SI{1}{\atm}}\cdot\SI{44,0}{\g\per\cancel\mole}
                            }{
                                \frac{\SI{22,4}{\cancel\atm\dm\cubed}}{\cancel{\SI{273,15}{\mole\kelvin}}} \cdot\cancel{\SI{273,15}{\kelvin}}
                            }
                        \]
                        \[ \rho = \SI{44,0/22,4}{\g\per\dm\cubed } \]
                        \[ \boxed{\rho = \SI{1,97}{\g\per\dm\cubed}} \]
                    \end{center}
          \end{enumerate}
    \item[12.] A determinada temperatura un recipiente rígido contiene \SI{3,00}{\mole} de un gas que ejercen
          una presión de \SI{1,80}{\atm}. A temperatura constante, se abre una llave hasta que la presión se
          iguala a la presión atmosférica (\SI{1,00}{\atm}). Calculen la cantidad de gas, expresada en moles,
          que se perdió
          \begin{center}
              \[ PV = nRT \]
              \[ \frac{P}{n} = \frac{RT}{V} \]
              \[ \frac{P_i}{n_i} = \frac{P_f}{n_f} \]
              \[ \frac{\SI{1,80}{\atm}}{\SI{3,00}{\mole}} = \frac{\SI{1,00}{\atm}}{n_f} \]
              \[ \frac{\SI{1,80}{\cancel\atm}}{\SI{3,00}{\mole}} = \frac{\cancel{\SI{1,00}{\atm}}}{n_f} \]
              \[ n_f = \frac{\SI{3,00}{\mole}}{\num{1,80}} \]
              \[ n_f = \SI{1,67}{\mole} \]
              \[ \SI{3,00}{\mole} - \SI{1,67}{\mole} \]
              \[ \boxed{\SI{1,33}{\mole}} \]
          \end{center}
    \item[15.] Una mezcla, formada por masas iguales de $CO_2$, $CO$ y $O_2$, se encuentra a \SI{25,0}{\celsius} en un
          recipiente rígido de \SI{1,37}{\dm\cubed}. Determinen: \\
          $ \SI{25,0}{\celsius} = \SI{298,15}{\kelvin} $
          \begin{enumerate}
              \item cuál de los gases ejerce mayor presión en la mezcla;
                    \begin{center}
                        (Masa de cada gas (\si{\g}): $m$) \\
                        \begin{tabular}{ l | c }
                            Molecula     & Moles (por \si{\g})               \\
                            \hline
                            $ n_{CO_2} $ & $ \SI{1/44}{\mole\per\g}\cdot m $ \\
                            $ n_{CO} $   & $ \SI{1/28}{\mole\per\g}\cdot m $ \\
                            $ n_{O_2} $  & $ \SI{1/32}{\mole\per\g}\cdot m $
                        \end{tabular}
                    \end{center}
                    $CO_2$:
                    \begin{center}
                        \[
                            P_{CO_2}\cdot\SI{1,37}{\dm\cubed}
                            = (\SI{1/44}{\mole\per\g}\cdot m)
                            \cdot\frac{\SI{22,4}{\atm\dm\cubed}}{\SI{273,15}{\mole\kelvin}}
                            \cdot\SI{298,15}{\kelvin}
                        \]
                        \[
                            P_{CO_2}\cdot\SI{1,37}{\cancel\dm\cubed}
                            = (\SI{1/44}{\cancel\mole\per\g}\cdot m)
                            \cdot\frac{\SI{22,4}{\atm\cancel\dm\cubed}}{\SI{273,15}{\cancel\mole\cancel\kelvin}}
                            \cdot\SI{298,15}{\cancel\kelvin}
                        \]
                        \[
                            P_{CO_2}\cdot\num{1,37}
                            = (\SI{1/44}{\g^{-1}}\cdot m)
                            \cdot\frac{\SI{22,4}{\atm}}{\num{273,15}}
                            \cdot\num{298,15}
                        \]
                        \[
                            P_{CO_2} = m
                            \cdot\frac{
                                \num{22,4}\cdot\num{298,15}
                            }{
                                \num{44}\cdot\num{273,15}\cdot\num{1,37}
                            }
                            \cdot\si{\atm\per\g}
                        \]
                        \[
                            P_{CO_2} = m\cdot\SI{0,40}{\atm\per\g}
                        \]
                    \end{center}
                    $CO$:
                    \begin{center}
                        \[
                            P_{CO} = m
                            \cdot\frac{
                                \num{22,4}\cdot\num{298,15}
                            }{
                                \num{28}\cdot\num{273,15}\cdot\num{1,37}
                            }
                            \cdot\si{\atm\per\g}
                        \]
                        \[
                            P_{CO} = m\cdot\SI{0,68}{\atm\per\g}
                        \]
                    \end{center}
                    $O_2$:
                    \begin{center}
                        \[
                            P_{O_2} = m
                            \cdot\frac{
                                \num{22,4}\cdot\num{298,15}
                            }{
                                \num{32}\cdot\num{273,15}\cdot\num{1,37}
                            }
                            \cdot\si{\atm\per\g}
                        \]
                        \[
                            P_{O_2} = m\cdot\SI{0,56}{\atm\per\g}
                        \]
                    \end{center}
                    \begin{center}
                        \begin{tabular}{ l | l }
                            Molecula & Presión (por \si{\g})                 \\
                            \hline
                            $CO_2$   & $m\cdot\SI{0,40}{\atm\per\g}$         \\
                            $CO$     & \boxed{$m\cdot\SI{0,68}{\atm\per\g}$} \\
                            $O_2$    & $m\cdot\SI{0,56}{\atm\per\g}$
                        \end{tabular}
                    \end{center}
              \item si la fracción molar del $O_2$ es mayor, menor o igual que la del $CO$;
                    \begin{center}
                        \[ n_T = m\cdot(\num{1/44}+\num{1/28}+\num{1/32})\si{\mole\per\g} =  \]
                        \[ X_{O_2} = \frac{n_{O_2}}{n_T} = \num{0,35} \]
                        \[ X_{CO} = \frac{n_{CO}}{n_T} = \num{0,40} \]
                        \[ \boxed{X_{O_2} < X_{CO}} \]
                    \end{center}
              \item si la presión de la mezcla aumenta, disminuye o no cambia, al elevar la temperatura
                    hasta \SI{400}{\kelvin}. Justifiquen la respuesta
                    \begin{center}
                        \[ \frac{P}{T} = \frac{nR}{V} \]
                        \[ \frac{P_i}{T_i} = \frac{P_f}{T_f} \]
                        \[ \frac{P_i}{\SI{298,15}{\kelvin}} = \frac{P_f}{\SI{400}{\kelvin}} \]
                        \[ P_i\cdot\frac{\SI{400}{\kelvin}}{\SI{298,15}{\kelvin}} = P_f \]
                        \[ P_f = \num{1,34}P_i \]
                        \[ \boxed{P_f > P_i} \]
                    \end{center}
          \end{enumerate}
    \item[17.] En un recipiente rígido de \SI{30,0}{\dm\cubed}, se colocan \SI{0,500}{\mole} de $CO_2$ (g) y \SI{35,0}{\g} de $Cl_2O$ a
          \SI{25}{\celsius}. A temperatura constante se agrega cierta masa de neón, lo que produce un aumento en
          la presión de \SI{0,415}{\atm}. Calculen: \\
          $ \SI{35,0}{\g} $ de $ Cl_2O = \SI{0,40}{\mole} $; $ \SI{25,0}{\celsius} = \SI{298,15}{\kelvin} $
          \begin{enumerate}
              \item la masa de neón agregada;
                    \begin{center}
                        \[ P_i = P_{CO_2} + P_{Cl_2O} \]
                        \[ P_f = P_{CO_2} + P_{Cl_2O} + P_{Ne} = P_i + P_{Ne}\]
                        \[ \Delta P = P_{Ne} = \SI{0,415}{\atm} \]
                        \[
                            \SI{0,415}{\atm}
                            =
                            \frac{n_{ne}
                                \cdot\frac{\SI{22,4}{\atm\dm\cubed}}{\SI{273,15}{\mole\kelvin}}
                                \cdot\SI{298,15}{\kelvin}}{\SI{30,0}{\dm\cubed}}
                        \]
                        \[
                            \SI{0,415}{\cancel\atm}
                            =
                            \frac{n_{ne}
                                \cdot\frac{\SI{22,4}{\cancel\atm\cancel\dm\cubed}}{\SI{273,15}{\mole\cancel\kelvin}}
                                \cdot\SI{298,15}{\cancel\kelvin}}{\SI{30,0}{\cancel\dm\cubed}}
                        \]
                        \[
                            \num{0,415}
                            =
                            \frac{
                                n_{ne}
                                \cdot\frac{\num{22,4}}{\SI{273,15}{\mole}}
                                \cdot\num{298,15}
                            }{\num{30,0}}
                        \]
                        \[
                            \num{0,415}\cdot\num{30,0}
                            =
                            n_{ne}
                            \cdot\frac{\num{22,4}}{\SI{273,15}{\mole}}
                            \cdot\num{298,15}
                        \]
                        \[
                            \SI{12,45}{\mole}
                            =
                            n_{ne}
                            \cdot\num{24,45}
                        \]
                        \[ n_{ne} = \SI{0,51}{\mole} \]
                        \begin{tabular}{ c | c }
                            Moles      & Gramos             \\
                            \hline
                            1          & 10                 \\
                            \num{0,51} & \boxed{\num{10,3}}
                        \end{tabular}
                    \end{center}
              \item la presión total luego del agregado del neón;
                    \begin{center}
                        \[
                            P\cdot\SI{30,0}{\dm\cubed}
                            = (\num{0,51}+\num{0,500}+\num{0,40})\si{\mole}
                            \cdot
                            \frac{
                                \SI{22,4}{\atm\dm\cubed}
                            }{
                                \SI{273,15}{\mole\kelvin}
                            }
                            \cdot\SI{298,15}{\kelvin}
                        \]
                        \[
                            P\cdot\SI{30,0}{\cancel\dm\cubed}
                            = (\num{0,51}+\num{0,500}+\num{0,40})\si{\cancel\mole}
                            \cdot
                            \frac{
                                \SI{22,4}{\atm\cancel\dm\cubed}
                            }{
                                \SI{273,15}{\cancel\mole\cancel\kelvin}
                            }
                            \cdot\SI{298,15}{\cancel\kelvin}
                        \]
                        \[
                            P\cdot\num{30,0} = \num{1,41}\cdot\SI{22,4/273,15}{\atm}\cdot\num{298,15}
                        \]
                        \[ \boxed{P = \SI{1,15}{\atm}} \]
                    \end{center}
              \item la fracción molar del $Cl_2O$.
                    \begin{center}
                        \[ X_{Cl_2O} = \frac{n_{Cl_2O}}{n_T} \]
                        \[ n_T = (\num{0,51}+\num{0,500}+\num{0,40})\si{\mole} = \SI{1,41}{\mole} \]
                        \[ X_{Cl_2O} = \frac{\SI{0,40}{\mole}}{\SI{1,41}{\mole}} \]
                        \[ \boxed{X_{Cl_2O} = \num{0,284}} \]
                    \end{center}
          \end{enumerate}
\end{enumerate}

\end{document}