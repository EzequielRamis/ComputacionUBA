\documentclass[../practica.root.tex]{subfiles}

\begin{document}
\section{Unidad 10}
\subsection{Bloque 1}%1 , 3, 5, 6, 8,9, 10 y 11
\begin{enumerate}
	\item[1.] Indiquen si las siguientes afirmaciones son correctas (C) o incorrectas (I). Justifiquen las
	      respuestas.
	      \begin{enumerate}
		      \item Un ácido fuerte es aquel que en solución acuosa se encuentra parcialmente ionizado.
		      \item En una solución básica, el pOH es menor que el pH.
		      \item Se denomina base a la especie que en solución acuosa capta un ion hidrógeno.
		      \item El pH de una solución ácida es siempre mayor que \num{7,00}.
		      \item En una solución neutra se cumple que: $pH = pOH = 7$, a cualquier temperatura.
		      \item El pOH es una medida de la concentración de iones hidróxido.
		      \item Una base fuerte es aquella que en solución acuosa se encuentra totalmente disociada.
		      \item La ionización de un ácido débil es un proceso reversible.
		      \item Volúmenes diferentes de la misma solución presentan distintos valores de pH.
	      \end{enumerate}

	\item[3.] Se dispone de cinco tubos de ensayo que contienen agua destilada a los que se les agrega,
	      respectivamente, las siguientes sustancias:
	      \begin{enumerate}
		      \item tubo 1: $NaOH$
		      \item tubo 2: $H_2SO_4$
		      \item tubo 3: $CH_3COOH$
		      \item tubo 4: $NH_3$
		      \item tubo 5: $NaCl$
	      \end{enumerate}
	      Predigan si el valor del pH en cada tubo de ensayo, aumenta, disminuye o no cambia, luego
	      del agregado de cada sustancia

	\item[5.] En dos matraces aforados se preparan dos soluciones:

	      Matraz l: \SI{100}{\mL} de $HNO_3$ \SI{0,200}{\MR}
	      Matraz II: \SI{200}{\mL} de $HI$ 0,100M
	      \begin{enumerate}
		      \item Escriban la ecuación de ionización de cada ácido.
		      \item Indiquen cuáles de las siguientes afirmaciones son correctas (C):
		            \begin{enumerate}
			            \item $[H_3O^+]_I = [H_3O^+]_{II}$
			            \item $pH_I < pH_{II}$
			            \item $pOH_I < pOH_{II}$
			            \item $[OH^-]_{II} > [OH^-]_I$
			            \item $n {NO_3^-}_I = nI^-_{II}$
			            \item $n H_3O^+_{II} > n H_3O^+_I$
		            \end{enumerate}
	      \end{enumerate}

	\item[6.] Un recipiente contiene \SI{800}{\mL} de solución \SI{1,11}{\percent} m/V de $HClO_4$. Indiquen para la misma:
	      \begin{enumerate}
		      \item el número de aniones perclorato ($ClO$);
		      \item las fórmulas de todas las especies presentes;
		      \item el pH;
		      \item la concentración molar;
		      \item si su acidez es mayor, menor o igual a la de una solución de un ácido $HX$ \SI{0,0500}{\MR}.
	      \end{enumerate}

	\item[8.] Indiquen si las siguientes afirmaciones son correctas (C) o incorrectas (I). Justifiquen las
	      respuestas.
	      \begin{enumerate}
		      \item El pH de una solución de un ácido muy diluido puede ser mayor que 7.
		      \item El pOH de una solución de una base débil siempre es menor que el pOH de una solución de una base fuerte.
		      \item A mayor valor de $K_a$, mayor es la ionización del ácido.
		      \item El valor de $K_b$ solo se modifica si cambia la temperatura.
		      \item En una solución de una base débil, siempre se cumple que $pOH < pH$.
		      \item Cuanto mayor es el valor de $pK_b$ de una base débil, más débil es su ácido conjugado.
	      \end{enumerate}

	\item[9.] Se dispone de una solución \SI{1,00}{\MR} de HF (ácido débil).
	      \begin{enumerate}
		      \item Escriban:
		            \begin{enumerate}
			            \item la ecuación de ionización del ácido;
			            \item los pares ácido / base presentes;
			            \item la expresión de la constante de ionización del ácido (Ka);
			            \item las fórmulas de las especies presentes en la solución.
		            \end{enumerate}
		      \item Sin realizar cálculos, indiquen cuáles de las siguientes afirmaciones son correctas
		            (C); justifiquen las respuestas:
		            \begin{enumerate}
			            \item $[H_3O^+]_{eq} < \SI{1,00}{\MR}$
			            \item $pH = \SI{1,00}{\MR}$
			            \item $[H_3O^+]_{eq} = [HF]_{eq}$
			            \item $[H_3O^+]_{eq} = [F^-]_{eq}$
			            \item $pOH < \num{13,00}$
		            \end{enumerate}
	      \end{enumerate}

	\item[10.] Un recipiente contiene una solución \SI{7,46e-3}{\MR} de un ácido débil (HA), cuyo pH es de \num{3,67}.
	      \begin{enumerate}
		      \item Calculen el valor de:
		            \begin{enumerate}
			            \item $K_a$
			            \item $pK_a$
		            \end{enumerate}
		      \item Indiquen cuál de las siguientes perturbaciones disminuye el pH de la solución:
		            \begin{enumerate}
			            \item se agrega una sal de la base conjugada;
			            \item se agrega agua;
			            \item se agrega $KOH$;
			            \item se agrega $HI$.
		            \end{enumerate}
	      \end{enumerate}

	\item Se dispone de una solución acuosa de hidracina ($H_2NNH_2$) cuyo pH es de \num{8,25}. \\
	      Dato: $K_b(H_2NNH_2)$ = \num{3,02e-6}.
	      \begin{enumerate}
		      \item Escriban:
		            \begin{enumerate}
			            \item la ecuación de ionización de la base;
			            \item los pares ácido / base presentes;
			            \item la expresión de $K_b$;
			            \item las fórmulas de las especies presentes en la solución.
		            \end{enumerate}
		      \item Calculen la concentración molar en el equilibrio de:
		            \begin{enumerate}
			            \item $H_2NNH_2$
			            \item iones oxonio ($H_3O^+$)
		            \end{enumerate}
	      \end{enumerate}
\end{enumerate}
\subsection{Bloque 2}%4,6,11 y 16 
\end{document}
