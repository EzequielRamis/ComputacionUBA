\documentclass[../practica.root.tex]{subfiles}

% R = \frac{\SI{22,4}{\atm\dm\cubed}}{\SI{273,15}{\mole\kelvin}}

\begin{document}
\section{Unidad 8}
\subsection{Bloque 1}
\begin{enumerate}
    \item[4.] En un recipiente que contiene \SI{4,00}{\L} de solución acuosa de $HNO_3$ \SI{1,50}{\MR}, se coloca la
          cantidad de cobre suficiente para que reaccione el ácido nítrico presente. La ecuación que
          representa a la reacción es:
          \[ \sr{(0)}{Cu} \es + 4 \sr{(+1)(+5)(-2)}{HNO_3} \eac \lra \sr{(+2)(+5)(-2)}{Cu(NO_3)_2} \eac + 2 \sr{(+4)(-2)}{NO_2} \eg + 2 H_2O \el \]
          $HNO_3: \SI{1,50}{\MR} = \frac{\SI{6,00}{\mole}}{\SI{4,00}{\L}}$

          Indiquen:
          \begin{enumerate}
              \item la masa de cobre que reacciona;
                    \begin{center}
                        \begin{tabular}{c|c}
                            $Cu$                      & $HNO_3$            \\ \hline
                            $\SI{1}{\mole}$           & $\SI{4}{\mole}$    \\
                            $\blue{\SI{1,50}{\mole}}$ & $\SI{6,00}{\mole}$
                        \end{tabular}

                        \begin{tabular}{c|c}
                            $Cu$               & $Masa$                   \\ \hline
                            $\SI{1}{\mole}$    & $\SI{63,5}{\g}$          \\
                            $\SI{1,50}{\mole}$ & $\boxed{\SI{95,25}{\g}}$
                        \end{tabular}
                    \end{center}

              \item la masa de sal que se forma;
                    \[ \SI{1,50}{\mole} Cu \rightarrow \SI{1,50}{\mole} Cu(NO_3)_2 \]
                    \begin{center}
                        \begin{tabular}{c|c}
                            $Cu(NO_3)_2$       & $Masa$                    \\ \hline
                            $\SI{1}{\mole}$    & $\SI{187.5}{\g}$          \\
                            $\SI{1,50}{\mole}$ & $\boxed{\SI{281,25}{\g}}$
                        \end{tabular}
                    \end{center}

              \item la cantidad, expresada en moles, de $NO_2$ que se produce;
                    \begin{center}
                        \begin{tabular}{c|c}
                            $Cu$               & $NO_2$                  \\ \hline
                            $\SI{1}{\mole}$    & $\SI{2}{\mole}$         \\
                            $\SI{1,50}{\mole}$ & $\boxed{\SI{3}{\mole}}$
                        \end{tabular}
                    \end{center}

              \item el número de moléculas de agua que se obtienen;
                    \begin{center}
                        \begin{tabular}{c|c}
                            $Cu$               & $H_2O$          \\ \hline
                            $\SI{1}{\mole}$    & $\SI{2}{\mole}$ \\
                            $\SI{1,50}{\mole}$ & $\SI{3}{\mole}$
                        \end{tabular}
                    \end{center}
                    \[ \SI{3}{\mole}\cdot\SI[per-mode=reciprocal]{6,023e23}{\per\mole} \]
                    \[ \SI{3}{\cancel\mole}\cdot\SI[per-mode=reciprocal]{6,023e23}{\cancel\per\mole} \]
                    \[ \num{3}\cdot\num{6,023e23} \]
                    \[ \num{18,069e23} \]
                    \[ \boxed{\num{1,8069e24}} \]
              \item el tipo de reacción química que representa. \\
                    Redox. $Cu: 0 \rightarrow +2$; $N: +5 \rightarrow +4$
          \end{enumerate}

    \item[6.] En un recipiente se colocan \SI{280}{\g} de una muestra que contiene antimonio y \SI{42,0}{\g} de
          impurezas con exceso de ácido nitroso. La reacción que se produce se representa por la
          siguiente ecuación:
          \[ 5 HNO_2 \eac + Sb \es \lra 5 NO \eg + H_3SbO_4 \eac + H_2O \el \]
          Indiquen:
          \begin{enumerate}
              \item la pureza de la muestra;
                    \[ \frac{\SI{42,0}{\g}}{\SI{280}{\g}} = \SI{15}{\percent}\text{ de impureza} \]
                    \[ \SI{100}{\percent} - \SI{15}{\percent} = \boxed{\SI{85}{\percent}} \]
              \item el volumen mínimo de solución \SI{2,00}{\MR} de ácido nitroso necesario;
                    \begin{center}
                        \begin{tabular}{c|c}
                            Moles $Sb$        & Masa                         \\
                            \hline
                            1                 & \SI{122}{\g}                 \\
                            \blue{\num{1,95}} & $\SI{280}{\g} - \SI{42}{\g}$ \\
                        \end{tabular}
                    \end{center}
                    Moles de $HNO_2$ necesarios: $5 \cdot \num{1,95} = \num{9,75}$
                    \[ \SI{2,00}{\MR} = \SI{2,00}{\mole\per\L} \]
                    \[ \SI{2,00}{\mole\per\L}\cdot v = \SI{9,75}{\mole} \]
                    \[ \boxed{v = \SI{4,88}{\L}} \]
              \item la presión que ejerce el gas obtenido si se lo recoge en un recipiente de \SI{20,0}{\dm\cubed} a \SI{25,0}{\celsius};
                    \[ \text{Moles de $NO$}: \SI{9,75}{\mole} \]
                    \[ \SI{25,0}{\celsius} = \SI{298,15}{\kelvin} \]
                    \[ VP = nRT \]
                    \[
                        \SI{20,0}{\dm\cubed}\cdot P =
                        \SI{9,75}{\mole}
                        \cdot\SI{22,4/273,15}{\atm\dm\cubed\per\mole\per\kelvin}
                        \cdot\SI{298,15}{\kelvin}
                    \]
                    \[
                        \SI{20,0}{\cancel\dm\cubed}\cdot P =
                        \SI{9,75}{\cancel\mole}
                        \cdot\SI{22,4/273,15}{\atm\cancel\dm\cubed\per\cancel\mole\per\cancel\kelvin}
                        \cdot\SI{298,15}{\cancel\kelvin}
                    \]
                    \[
                        P = \frac{
                            \num{9,75 \cdot 22,4 \cdot 298,15}
                        }{
                            \num{273,15 \cdot 20,0}
                        }
                    \]
                    \[ \boxed{\SI{11,92}{\atm}} \]

              \item el tipo de reacción química que representa. \\
                    Redox. $Sb: 0 \rightarrow +5$; $N: +3 \rightarrow +2$
          \end{enumerate}

    \item[8.] En un recipiente se colocan \SI{500}{\mL} de solución acuosa \SI{2,00}{\MR} de hidróxido de sodio,
          \SI{30,0}{\g} de una muestra formada por carbono (\SI{10,0}{\percent} de impurezas inertes) y agua en
          exceso. La reacción se representa por la siguiente ecuación:
          \[ C \es + 2 NaOH \eac + H_2O \el \lra Na_2CO_3 \eac + 2 H_2 \eg \]
          Determinen:
          \begin{enumerate}
              \item la masa de sal que se produce;
                    \[ \text{Moles $NaOH$: } \SI{500}{\mL}\cdot\SI{2,00}{\MR} = \SI{1,00}{\mole} \]
                    \begin{center}
                        \begin{tabular}{c|c}
                            Moles $C$  & Masa                                  \\
                            \hline
                            1          & \SI{12}{\g}                           \\
                            \num{2,25} & $\SI{30,0}{\g}\cdot\SI{90}{\percent}$
                        \end{tabular}
                    \end{center}
                    Reactivo limitante: $NaOH$
                    \begin{center}
                        \begin{tabular}{c|c|c}
                            $NaOH$        & $Na_2CO_3$      & Masa $Na_2CO_3$     \\
                            \hline
                            \SI{2}{\mole} & \SI{1}{\mole}   & \SI{106}{\g}        \\
                            \SI{1}{\mole} & \SI{0,5}{\mole} & \boxed{\SI{53}{\g}}
                        \end{tabular}
                    \end{center}

              \item la cantidad de gas, expresada en moles, que se libera; \\
                    Por cada mol de $NaOH$ se produce otro de $H_2$
                    \[ \boxed{\SI{1}{\mole}} \]

              \item si la cantidad de gas que se forma, aumenta, disminuye o no cambia,
                    al repetir la experiencia con una muestra que contiene carbono con mayor
                    porcentaje de pureza. Justifiquen la respuesta. \\
                    \boxed{\text{RTA: No cambia ya que el reactivo limitante es el $NaOH$}}
          \end{enumerate}


    \item[9.] El sulfuro de zinc reacciona con el oxígeno formando óxido de zinc y dióxido de azufre.
          Si se utiliza un mineral que contiene $ZnS$, con una pureza del \SI{70,0}{\percent}, y se obtienen
          \SI{367,2}{\dm\cubed} de $SO_2$ medidos en CNPT:
          \begin{enumerate}
              \item escriban la ecuación química que representa el proceso;
              \item indiquen:
                    \begin{enumerate}
                        \item la masa de mineral utilizada;
                        \item la cantidad, expresada en moles, de ZnO que se produce.
                    \end{enumerate}
          \end{enumerate}

    \item[11.] Se hacen reaccionar \SI{44,8}{\g} de una muestra de cobre que contiene \SI{400}{\mg} de impurezas
          inertes con \SI{454}{\cm\cubed} de solución acuosa de $H_2SO_4$ \SI{3,30}{\MR}. La reacción se produce con un
          rendimiento del \SI{86,0}{\percent} y la ecuación que representa el proceso es:
          \[ Cu \es + 2 H_2SO_4 \eac \lra CuSO_4 \eac + SO_2 \eg + 2 H_2O \el \]
          El gas obtenido se recoge a \SI{25,0}{\celsius} en un recipiente rígido de \SI{12,5}{\dm\cubed}.
          \begin{enumerate}
              \item Indiquen cuál es el cambio que se produce en el número de oxidación del elemento
                    correspondiente a la especie que se oxida.
                    \[ \sr{(0)}{Cu} \es + 2 \sr{(+1)(+6)(-2)}{H_2SO_4} \eac
                        \lra \sr{(+2)(+6)(-2)}{CuSO_4} \eac + \sr{(+4)(-2)}{SO_2} \eg + 2 \sr{(+1)(-2)}{H_2O} \el \]
                    \boxed{Cu \text{: } 0 \rightarrow +2}
              \item Calculen:
                    \begin{enumerate}[label=\roman*)]
                        \item la presión que ejerce el gas obtenido en el recipiente;

                              \ii{Calcular la cantidad de cobre en moles: }
                              \begin{center}
                                  \begin{tabular}{c|c}
                                      $Cu$            & Masa                           \\
                                      \hline
                                      \SI{1}{\mole}   & \SI{63,5}{\g}                  \\
                                      \SI{0,7}{\mole} & $\SI{44,8}{\g}-\SI{0,400}{\g}$ \\
                                  \end{tabular}
                              \end{center}

                              \ii{Calcular la cantidad de moles de $N_2SO_4$: }
                              \[ \SI{3,30}{\MR}\cdot\SI{0,454}{\dm\cubed} = \SI{1,50}{\mole} \]

                              El reactivo limitante es el $Cu$ (\SI{0,7}{\mole})
                              \begin{center}
                                  \begin{tabular}{c|c}
                                      $Cu$            & $SO_2$                                  \\
                                      \hline
                                      \SI{1}{\mole}   & $\SI{1}{\mole}\cdot\SI{86,0}{\percent}$ \\
                                      \SI{0,7}{\mole} & \SI{0,6}{\mole}
                                  \end{tabular}
                              \end{center}
                              \[ PV = nRT \]
                              \[
                                  P\cdot\SI{12,5}{\dm\cubed}
                                  = \SI{0,6}{\mole}
                                  \cdot\SI{22,4/273,15}{\atm\dm\cubed\per\mole\per\kelvin}
                                  \cdot\SI{298.15}{\kelvin}
                              \]
                              \[
                                  P\cdot\SI{12,5}{\cancel\dm\cubed}
                                  = \SI{0,6}{\cancel\mole}
                                  \cdot\SI{22,4/273,15}{\atm\cancel\dm\cubed\per\cancel\mole\per\cancel\kelvin}
                                  \cdot\SI{298.15}{\cancel\kelvin}
                              \]
                              \[
                                  P\cdot\num{12,5}
                                  = \num{0,6}
                                  \cdot\SI{22,4/273,15}{\atm}
                                  \cdot\num{298.15}
                              \]
                              \[ \boxed{P = \SI{1,17}{\atm}} \]

                        \item el porcentaje de pureza de la muestra de cobre;
                              \[ \frac{\SI{44,8}{\g}-\SI{0,400}{\g}}{\SI{44,8}{\g}} \]
                              \[ \boxed{\SI{0,99}{\percent}} \]

                        \item el volumen de solución \SI{10,0}{\percent} m/V que se puede preparar con la masa de sal obtenida;
                              \begin{center}
                                  \begin{tabular}{c|c}
                                      $CuSO_4$        & Masa            \\
                                      \hline
                                      \SI{1}{\mole}   & \SI{159,5}{\g}  \\
                                      \SI{0,7}{\mole} & \SI{111,65}{\g} \\
                                  \end{tabular}
                              \end{center}
                              \[ \SI{111,65}{\g}\cdot\SI{86,0}{\percent} = \SI{96,019}{\g} \]
                              \[ \frac{\SI{96}{\g}}{v} = \SI{10,0}{\percent} m/V \]
                              \[ \boxed{v = \SI{960}{\cm\cubed}} \]

                        \item la cantidad de agua, expresada en moles, que se obtiene.
                              \begin{center}

                                  \begin{tabular}{c|c}
                                      $Cu$            & $H_2O$          \\
                                      \hline
                                      \SI{1}{\mole}   & \SI{2}{\mole}   \\
                                      \SI{0,7}{\mole} & \SI{1,4}{\mole} \\
                                  \end{tabular}
                              \end{center}
                              \[ \SI{1,4}{\mole}\cdot\SI{86,0}{\percent} = \boxed{\SI{1,2}{\mole}} \]
                    \end{enumerate}
          \end{enumerate}


\end{enumerate}
\subsection{Bloque 2}
\begin{enumerate}
    \item[7.] En un recipiente se colocan \SI{2,50}{\mole} de ácido propanoico y \SI{300}{\mL} de solución
          acuosa \SI{20,0}{\percent} m/V de hidróxido de sodio. La ecuación que representa el proceso es:
          \[ C_3H_6O_2 \el + NaOH \eac \lra C_3H_5O_2Na \eac + H_2O \el \]
          Indiquen:
          \begin{enumerate}
              \item la cantidad, expresada en moles, del reactivo que queda sin reaccionar;
              \item la masa de propanoato de sodio que se forma;
              \item si la masa de sal aumenta, disminuye o no cambia, si se utilizan \SI{400}{\mL} de la misma
                    solución, sin modificar la cantidad de ácido empleado. Justifiquen la respuesta;
              \item el tipo de reacción química que representa.
          \end{enumerate}

    \item[12.] Se hace reaccionar \SI{1,50}{\dm\cubed} de solución acuosa de hidróxido de sodio \SI{1,50}{\MR} con cantidad
          suficiente de silicio y de agua, según la siguiente ecuación:
          \[ Si + 2 NaOH (sc) + H_2O \lra Na_2SiO_3 + 2 H_2 \eg \] % "SC"????
          El hidrógeno obtenido se recoge en un recipiente rígido de \SI{22,0}{\dm\cubed} a \SI{25,0}{\celsius} ejerciendo
          una presión de \SI{2,00}{\atm}. Calculen:
          \begin{enumerate}
              \item el rendimiento de la reacción;
              \item la masa de agua que reacciona;
              \item la masa de silicato de sodio que se forma.
          \end{enumerate}

    \item[14.] En un recipiente se colocan \SI{200}{\g} de un mineral que contiene un \SI{80,0}{\percent} de cobre con
          \SI{2,50}{\liter} de solución acuosa de $HNO_3$ \SI{0,100}{\MR}. La reacción se representa por la siguiente
          ecuación:
          \[ 3 Cu + 8 HNO_3 \lra 3 Cu(NO_3)_2 + 2 NO \eg + 4 H_2O \]
          Si se obtienen \SI{1,60}{\g} de $NO \eg$, indiquen:
          \begin{enumerate}
              \item el cambio en el número de oxidación del elemento en la sustancia que se reduce;
              \item si el cobre de la muestra reacciona totalmente; justifiquen la respuesta;
              \item el rendimiento de la reacción;
              \item si la presión ejercida por el gas obtenido será mayor, igual o menor, al repetir la
                    experiencia con igual masa de otra muestra de cobre con \SI{40,0}{\percent} de impurezas inertes,
                    manteniendo todas las demás condiciones.
          \end{enumerate}

    \item[16.] En un recipiente se introducen \SI{1500}{\mL} de solución acuosa de ácido sulfúrico \SI{0,250}{\MR} con
          \SI{1500}{\mL} de una solución acuosa de ácido yodhídrico. La reacción tiene un rendimiento del
          \SI{75,0}{\percent} y se representa por la siguiente ecuación:
          \[ H_2SO_4 (sc) + 8 HI (sc) \lra H_2S + 4 I_2 \es + 4 H_2O \]
          Si se obtienen \num{6,77e23} moléculas de yodo, calculen:
          \begin{enumerate}
              \item la concentración de la solución de ácido yodhídrico, expresada en \% m/V;
              \item la masa de agua que se forma;
              \item la cantidad de $H_2S$, expresada en moles, que se obtiene.
          \end{enumerate}

    \item[19.] Al hacer reaccionar \SI{200}{\g} de una muestra de $PbO_2$ (\SI{85,0}{\percent} de pureza) con \SI{1,00}{\dm\cubed}
          de solución acuosa de $HCl$ \SI{1,80}{\MR}, se obtienen \SI{7,00}{\dm\cubed} de cloro gaseoso en
          CNPT. La reacción se representa por la siguiente ecuación:
          \[ PbO_2 + 4 HCl \lra PbCl 2 + Cl_2 \eg + 2 H_2O \]
          \begin{enumerate}
              \item Indiquen qué tipo de reacción química representa la ecuación dada. Justifiquen la respuesta.
              \item Determinen cuál es el reactivo en exceso y la masa del mismo que queda sin reaccionar.
              \item calculen:
                    \begin{enumerate}
                        \item el rendimiento de la reacción;
                        \item la masa de sal que se forma;
                        \item el número de moléculas de agua que se obtienen.
                    \end{enumerate}
          \end{enumerate}

\end{enumerate}
\end{document}
