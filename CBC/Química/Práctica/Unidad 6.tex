\documentclass[../Práctica.root.tex]{subfiles}
 
\begin{document}

\section{Unidad 6}
\subsection{Bloque 1}
\begin{enumerate}
    \item Un recipiente de tapa móvil contiene \SI{1,00}{\deci\m\cubed} de oxígeno gaseoso, a \SI{1520}{\torr} y a
          \SI{30,0}{\celsius}. Calculen la presión que ejercerá esa cantidad de oxígeno si el volumen se reduce
          hasta \SI{200}{\centi\m\cubed} y la temperatura a \SI{-20,0}{\celsius}.

    \item Indiquen cuál es el volumen que ocuparán \num{2,40} moles de una sustancia en estado gaseoso a
          \SI{127}{\celsius} y a presión normal.

    \item No necesario

    \item Calculen el volumen molar de un gas ideal, a \SI{27,0}{\celsius} y a una presión de \SI{2,00}{\atm}.

    \item No necesario

    \item No necesario

    \item No necesario

    \item Se dispone de un cilindro de \SI{100}{\liter} que contiene N_2 (g), a una presión de \SI{5,00}{\atm} y a una
          temperatura de \SI{22,0}{\celsius}. Calculen:
          \begin{enumerate}
              \item la masa de nitrógeno en el recipiente;
              \item el volumen que ocuparía esa cantidad de gas si a temperatura constante, la presión se
                    reduce \num{5} veces.
          \end{enumerate}

    \item No necesario

    \item No necesario

    \item No necesario

    \item No necesario

    \item No necesario

    \item No necesario

    \item No necesario

    \item No necesario

    \item No necesario

    \item No necesario

    \item No necesario

    \item Un recipiente rígido de \SI{10,0}{\dm\cubed} contiene cierta masa de CO_2 (g) en CNPT. Se agrega CO (g)
          hasta que la masa de la mezcla de gases es de \SI{60,0}{\gram}. Se produce una variación de la
          temperatura y un aumento en la presión de \SI{2,5}{\atm}. Indiquen:
          \begin{enumerate}
              \item la temperatura final que alcanza el sistema;
              \item si la presión parcial del dióxido de carbono en la mezcla es mayor, igual o menor que
                    la del monóxido de carbono;
              \item el número de átomos de oxígeno que hay en la mezcla;
              \item si la temperatura final alcanzada aumenta, disminuye o no cambia, si en lugar de CO
                    se hubiera agregado O_2 (g) hasta tener la misma masa final de \SI{60,0}{\gram} y el mismo
                    aumento de presión. Justifiquen la respuesta.
          \end{enumerate}
\end{enumerate}

\end{document}