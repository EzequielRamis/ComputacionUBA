\documentclass[../Práctica.root.tex]{subfiles}

\newcommand{\lra}{\ensuremath{\longrightarrow{}}}
\newcommand{\sr}[2]{\stackrel{#1}{#2}}

\begin{document}

\section{Unidad 7}
\subsection{Bloque 1}
\begin{enumerate}
    \item Lean atentamente y señalen cuáles de las opciones son correctas.
          Siempre que se produce una reacción química:
          \begin{enumerate}
              \item el tipo y número de átomos que intervienen no cambia; \checkmark
              \item se desprenden luz y/o calor;
              \item se produce siempre entre dos sustancias;
              \item cambian las sustancias presentes; \checkmark
              \item aparecen átomos nuevos;
              \item hay ruptura y formación de enlaces. \checkmark
          \end{enumerate}

    \item Escriban los coeficientes estequiométricos para balancear las ecuaciones:
          \begin{enumerate}
              \item $Ca(OH)_2$ (ac) + \blue{2}$HNO_3$ (ac) \lra \blue{}$Ca(NO_3)_2$ (ac) + \blue{2}$H_2O$ (l)
              \item $Fe$ (s) + $HCl$ (ac) \lra $FeCl_3$ (ac) + $H_2$ (g) \\
                    $Fe$ + \blue{3}$HCl$ \lra $FeCl_3$ + \blue{\num{1,5}}$H_2$ \\
                    ($Fe$ + \blue{3}$HCl$ \lra $FeCl_3$ + \blue{\num{1,5}}$H_2$)\cdot\blue{\num{2}} \\
                    \blue{2}$Fe$ (s) + \blue{6}$HCl$ (ac) \lra \blue{2}$FeCl_3$ (ac) + \blue{3}$H_2$ (g)
              \item $Al_2O_3$ (s) + $H_2$ (g) \lra $Al$ (s) + $H_2O$ (l) \\
                    $Al_2O_3$ + $H_2$ \lra \blue{2}$Al$ + \blue{3}$H_2O$ \\
                    $Al_2O_3$ (s) + \blue{3}$H_2$ (g) \lra \blue{2}$Al$ (s) + \blue{3}$H_2O$ (l)
              \item $H_2SO_4$ (ac) + $Fe(OH)_3$ (ac) \lra $Fe_2(SO_4)_3$ (ac) + $H_2O$ (l) \\
                    \blue{3}$H_2SO_4$ + \blue{2}$Fe(OH)_3$ \lra $Fe_2(SO_4)_3$ + $H_2O$ \\
                    \blue{3}$H_2SO_4$ (ac) + \blue{2}$Fe(OH)_3$ (ac) \lra $Fe_2(SO_4)_3$ (ac) + \blue{6}$H_2O$ (l)
              \item $KClO_3$ (s) \lra $KCl$ (s) + $O_2$ (g) \\
                    $KClO_3$ \lra $KCl$ + \blue{\num{1,5}}$O_2$ \\
                    ($KClO_3$ \lra $KCl$ + \blue{\num{1,5}}$O_2$)\cdot\blue{\num{2}} \\
                    \blue{2}$KClO_3$ (s) \lra \blue{2}$KCl$ (s) + \blue{3}$O_2$ (g)
          \end{enumerate}

    \item Para las siguientes ecuaciones:
          \begin{enumerate}
              \item $\sr{\blue{(+4)}(-1)}{TiCl_4}$ (ac) + 2 $\sr{\blue{(0)}}{Mg}$ (s)
                    \lra $\sr{\blue{(0)}}{Ti}$ (s) + 2 $\sr{\blue{(+2)}(-1)}{MgCl_2}$ (ac)
              \item $\sr{(+1)(-1)}{KBr}$ (ac) + $\sr{(+1)(+5)(-2)}{AgNO_3}$ (ac)
                    \lra $\sr{(+1)(-1)}{AgBr}$ (s) + $\sr{(+1)(+5)(-2)}{KNO_3}$ (ac)
              \item 2 $\sr{(+1)(-2)(+1)}{NaOH}$ (ac) + $\sr{(+1)(+4)(-2)}{H_2SO_4}$ (ac)
                    \lra $\sr{(+1)(+6)(-2)}{Na_2SO_4}$ (ac) + 2 $\sr{(+1)(-2)}{H_2O}$ (l)
              \item $C_3H_8$ (g) + 5 $O_2$ (g) \lra 3 $CO_2$ (g) + 4 $H_2O$ (g)
              \item 2 $SO_2$ (g) + $O_2$ (g) \lra 2 $SO_3$ (g)
              \item 6 $CoCl_2$ (ac) + 12 $KOH$ (ac) + $KClO_3$ (ac) \lra 3 $Co_2O_3$ (s) + 13 $KCl$ (ac) + 6 $H_2O$ (l)
              \item $CaCO_3$ (s) \lra $CaO$ (s) + $CO_2$ (g)
          \end{enumerate}
          Indiquen:
          \begin{enumerate}
              \item el tipo de reacción química que representa (síntesis, descomposición, combustión,
                    precipitación, neutralización o redox);
                    \begin{enumerate}[label=\alph*]
                        \item Redox
                        \item Intercambio
                        \item Neutralización
                        \item Combustión
                        \item Redox y síntesis
                        \item
                    \end{enumerate}
              \item el agente oxidante, el agente reductor y el cambio que se produce en los estados de
                    oxidación, en las que representan a reacciones redox;
              \item Justificar los coeficientes de las ecuaciones A y F utilizando el método del ionelectrón \\
                    $TiCl_4$ + 2 $Mg$ \lra $Ti$ + 2 $MgCl_2$ \\
                    $Ti^{4+}Cl^-_4$ + 2 $Mg^0$ \lra $Ti^0$ + 2 $Mg^{2+}Cl^-_2$
                    \begin{multicols}{2}
                        Oxidación: \\
                        $Mg^0$ \lra $Mg^{2+}Cl_2^-$ \\
                        \blue{$Cl_2^-$ +} $Mg^0$ \lra $Mg^{2+}Cl_2^-$ \blue{+ 2 $e^-$} \\
                        ($Cl_2^-$ + $Mg^0$ \lra $Mg^{2+}Cl_2^-$ + 2 $e^-$)\cdot\blue{2} \\
                        $Cl_4^-$ + 2 $Mg^0$ \lra 2 $Mg^{2+}Cl_2^-$ + 4 $e^-$

                        \columnbreak

                        Reducción: \\
                        $Ti^{4+}$ \lra $Ti^0$ \\
                        $Ti^{4+}$ + \blue{4 $e^-$} \lra $Ti^0$
                    \end{multicols}
                    \begin{tabular}{ r c l }
                        2 $Mg^0$              & \lra & 2 $Mg^{2+}$ + 4 $e^-$ \\
                        + $Ti^{4+}$ + 4 $e^-$ & \lra & $Ti^0$
                    \end{tabular}

          \end{enumerate}

    \item Ajustar las siguientes ecuaciones químicas por el método ion electrón en medio básico:
          \begin{enumerate}
              \item $Na_2SO_3$ + $NaOH$ + $I_2$ \lra $Na_2SO_4$ + $NaI$ + $H_2O$
              \item $KClO$ + $KAsO_2$ + $KOH$ \lra $K_3AsO_4$ + $KCl$ + $H_2O$
          \end{enumerate}

    \item Ajustar las siguientes ecuaciones químicas por el método ion electrón en medio ácido:
          \begin{enumerate}
              \item $Sn$ + $HNO_3$ \lra $SnO_2$ + $NO$ + $H_2O$
              \item $KIO_3$ + $KI$ + $H_2SO_4$ \lra $I_2$ + $H_2O$ + $K_2SO_4$
          \end{enumerate}

    \item Ajustar las siguientes ecuaciones químicas por el método del número de oxidación en vía
          seca.
          \begin{enumerate}
              \item $ZnS$ + $O_2$ \lra $ZnO$ + $SO_2$
              \item $AgNO_3$ + $Cu$ \lra $Cu(NO_3)_2$ + $Ag$
          \end{enumerate}
\end{enumerate}
\subsection{Bloque 2}
\begin{enumerate}
    \item Indiquen cuáles de los siguientes ejemplos de la vida cotidiana corresponden a reacciones
          químicas, y justifiquen su elección:
          \begin{enumerate}
              \item picar carne;
              \item putrefacción de una fruta;
              \item evaporación del alcohol;
              \item proceso de fotosíntesis;
              \item fundir un plástico;
              \item teñir el cabello;
              \item disolver sal en agua;
              \item pinchar un globo;
              \item quemar ramas;
              \item preparar un té;
              \item encender una hornalla;
              \item oxidación de un clavo.
          \end{enumerate}

    \item Para las siguientes ecuaciones:
          \begin{enumerate}
              \item 2 $K$ (s) + 2 $H_2O$ (l) \lra 2 $KOH$ (ac) + $H_2$ (g)
              \item $C_2H_4$ (g) + $HF$ (g) \lra $C_2H_5F$ (g)
              \item 3 $H_2SO_4$ (ac) + 2 $Ni(OH)_3$ (ac) \lra $Ni_2(SO_4)_3$ (ac) + 6 $H_2O$ (l)
              \item $AgNO_3$ (ac) + $NaI$ (ac) \lra $NaNO_3$ (ac) + $AgI$ (s)
              \item 2 $C_4H_{10}$ (g) + 13 $O_2$ (g) \lra 8 $CO_2$ (g) + 10 $H_2O$ (g)
              \item 4 $H_3PO_3$ (ac) + 2 $HNO_3$ (ac) \lra 4 $H_3PO_4$ (ac) + $N_2O$ (g) + $H_2O$ (l)
          \end{enumerate}
          Indiquen:
          \begin{enumerate}
              \item el tipo de reacción química que representa (síntesis, descomposición, combustión,
                    precipitación, neutralización o redox);
              \item el agente oxidante, el agente reductor y el cambio que se produce en los estados de
                    oxidación, en las que representan a reacciones redox.
          \end{enumerate}

    \item El cloruro de calcio, sólido, se produce a partir de la reacción entre el calcio y el cloro.
          \begin{enumerate}
              \item Escriban la ecuación química que representa el proceso.
              \item Indiquen el tipo de reacción química que representa.
          \end{enumerate}
\end{enumerate}
\end{document}