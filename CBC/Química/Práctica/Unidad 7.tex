\documentclass[../Práctica.root.tex]{subfiles}

\newcommand{\lra}{\ensuremath{\longrightarrow{}}}
\newcommand{\sr}[2]{\stackrel{#1}{#2}}
% Estados (ac) (s) (l) (g)
\newcommand{\eac}{\ \text{(ac)}}
\newcommand{\es}{\ \text{(s)}}
\newcommand{\el}{\ \text{(l)}}
\newcommand{\eg}{\ \text{(g)}}

\begin{document}

\section{Unidad 7}
\subsection{Bloque 1}
\begin{enumerate}
    \item Lean atentamente y señalen cuáles de las opciones son correctas.
          Siempre que se produce una reacción química:
          \begin{enumerate}
              \item el tipo y número de átomos que intervienen no cambia; \checkmark
              \item se desprenden luz y/o calor;
              \item se produce siempre entre dos sustancias;
              \item cambian las sustancias presentes; \checkmark
              \item aparecen átomos nuevos;
              \item hay ruptura y formación de enlaces. \checkmark
          \end{enumerate}

    \item Escriban los coeficientes estequiométricos para balancear las ecuaciones:
          \begin{enumerate}
              \item $Ca(OH)_2 \eac + \blue{2}HNO_3 \eac \lra Ca(NO_3)_2 \eac + \blue{2}H_2O \el$
              \item $Fe \es + HCl \eac \lra FeCl_3 \eac + H_2 \eg$ \\
                    $Fe + \blue{3}HCl \lra FeCl_3 + \blue{\num{1,5}}H_2$ \\
                    $(Fe + \blue{3}HCl \lra FeCl_3 + \blue{\num{1,5}}H_2)\cdot\blue{\num{2}}$ \\
                    $\blue{2}Fe \es + \blue{6}HCl \eac \lra \blue{2}FeCl_3 \eac + \blue{3}H_2 \eg$
              \item $Al_2O_3 \es + H_2 \eg \lra Al \es + H_2O \el$ \\
                    $Al_2O_3 + H_2 \lra \blue{2}Al + \blue{3}H_2O$ \\
                    $Al_2O_3 \es + \blue{3}H_2 \eg \lra \blue{2}Al \es + \blue{3}H_2O \el$
              \item $H_2SO_4 \eac + Fe(OH)_3 \eac \lra Fe_2(SO_4)_3 \eac + H_2O \el$ \\
                    $\blue{3}H_2SO_4 + \blue{2}Fe(OH)_3 \lra Fe_2(SO_4)_3 + H_2O$ \\
                    $\blue{3}H_2SO_4 \eac + \blue{2}Fe(OH)_3 \eac \lra Fe_2(SO_4)_3 \eac + \blue{6}H_2O \el$
              \item $KClO_3 \es \lra KCl \es + O_2 \eg$ \\
                    $KClO_3 \lra KCl + \blue{\num{1,5}}O_2$ \\
                    $(KClO_3 \lra KCl + \blue{\num{1,5}}O_2)\cdot\blue{\num{2}}$ \\
                    $\blue{2}KClO_3 \es \lra \blue{2}KCl \es + \blue{3}O_2 \eg$
          \end{enumerate}

    \item Para las siguientes ecuaciones: \\
          \ii{Nota: Las moléculas cuyos números de oxidación no están escritos no cambian entre reactivo y producto}
          \begin{enumerate}
              \item $\sr{\blue{(+4)}(-1)}{TiCl_4} \eac + 2 \sr{\blue{(0)}}{Mg} \es \lra \sr{\blue{(0)}}{Ti} \es + 2 \sr{\blue{(+2)}(-1)}{MgCl_2} \eac$
              \item $KBr \eac + AgNO_3 \eac \lra AgBr \es + KNO_3 \eac$
              \item $2 NaOH \eac + H_2SO_4 \eac \lra Na_2SO_4 \eac + 2 H_2O \el$
              \item $C_3H_8 \eg + 5 O_2 \eg \lra 3 CO_2 \eg + 4 H_2O \eg$
              \item $2 \sr{\blue{(+4)}(-2)}{SO_2} \eg + \sr{\blue{(0)}}{O_2} \eg \lra 2 \sr{\blue{(+6)(-2)}}{SO_3} \eg$
              \item $6 \sr{\blue{(+2)}(-1)}{CoCl_2} \eac + 12 KOH \eac + \sr{(+1)\blue{(+5)}(-2)}{KClO_3} \eac$
                    $\lra 3 \sr{\blue{(+3)}(-2)}{Co_2O_3} \es + 13 \sr{(+1)\blue{(-1)}}{KCl} \eac + 6 H_2O \el$
              \item $CaCO_3 \es \lra CaO \es + CO_2 \eg$
          \end{enumerate}
          Indiquen:
          \begin{enumerate}
              \item el tipo de reacción química que representa (síntesis, descomposición, combustión,
                    precipitación, neutralización o redox);
                    \begin{enumerate}[label=\alph*)]
                        \item Redox
                        \item Intercambio
                        \item Neutralización
                        \item Combustión
                        \item Redox y síntesis
                        \item Redox
                        \item Descomposición
                    \end{enumerate}
              \item el agente oxidante, el agente reductor y el cambio que se produce en los estados de
                    oxidación, en las que representan a reacciones redox;
                    \begin{enumerate}
                        \item[a)] Oxidanteg $TiCl_4$; Reductor: $Mg$
                        \item[e)] Oxidante: $O_2$; Reductor: $SO_2$
                        \item[f)] Oxidante: $KClO_3$; Reductor: $CoCl_2$
                    \end{enumerate}
              \item Justificar los coeficientes de las ecuaciones A y F utilizando el método del ionelectrón
                    \begin{itemize}
                        \item[a)] El método de ion-electrón no puede utilizarse en esta ecuación debido a que no tiene $H_2O$ (l)
                        \item[f)] $6 \sr{(+2)(-1)}{CoCl_2} + 12 KOH + \sr{(+1)(+5)(-2)}{KClO_3}$
                              $\lra 3 \sr{(+3)(-2)}{Co_2O_3} + 13 \sr{(+1)(-1)}{KCl} + 6 H_2O$ \\
                              $Co^{+2} + Cl_2^- + K^{+1} + OH^- + K^{+1} + ClO_3^- \lra Co_2O_3 + K^+ + Cl^- + H_2O$ \\ \\
                              \ii{Balance eléctrico}:
                              \begin{multicols}{2}
                                  Oxidación: \\
                                  $Co^{+2} \lra Co_2O_3$ \\
                                  $Co^{+2} \lra Co_2O_3 + \blue{e^-}$ \\
                                  $\blue{2} Co^{+2} \lra Co\blue{_2}O_3 + \blue{2}e^-$

                                  \columnbreak

                                  Reducción: \\
                                  $ClO_3^- \lra Cl^{-1}$ \\
                                  $ClO_3^- + \blue{6e^-} \lra Cl^{-1}$
                              \end{multicols}
                              \ii{Balance estequiométrico de O y H}:
                              \begin{multicols}{2}
                                  $2 Co^{+2} \lra Co_2O_3 + 2 e^-$

                                  \ii{Hay 3 átomos $O$ mas del lado der.}

                                  $2 Co^{+2} + \blue{6 OH^-} \lra Co_2O_3 + 2 e^- + \blue{3H_2O}$

                                  \ii{No hay diferencia de H}

                                  $2 Co^{+2} + 6 OH^- \lra Co_2O_3 + 2 e^- + 3 H_2O$

                                  \columnbreak

                                  $ClO_3^- + 6 e^- \lra Cl^-$

                                  \ii{Hay 3 átomos de O más del lado izq. }

                                  $\blue{3H_2O} + ClO_3^- + 6 e^- \lra Cl^- + \blue{6OH^-}$

                                  \ii{No hay diferencia de H}

                                  $3 H_2O + ClO_3^- + 6 e^- \lra Cl^- + 6 OH^-$
                              \end{multicols}
                              \ii{Multiplicación crusada para balancear $e^-$}
                              \begin{multicols}{2}
                                  $(2 Co^{+2} + 6 OH^- \lra Co_2O_3 + 2 e^- + 3 H_2O)\blue{\cdot\cancel{6}{3}}$ \\
                                  $6 Co^{+2} + 18 OH^- \lra 3 Co_2O_3 + \blue{6} e^- + 9 H_2O$

                                  \columnbreak

                                  $(3 H_2O + ClO_3^- + 6 e^- \lra Cl^{-1} + 6 OH^-)\blue{\cdot\cancel{2}{1}}$
                              \end{multicols}
                              Sumar:

                              \begin{tabular}{ r l } % Sí, estoy usando tablas para dar formato :P
                                  $6 Co^{+2} + 18 OH^-$                                                            & $\lra 3 Co_2O_3 + 6 e^- + 9 H_2O$                                               \\
                                  $3 H_2O + ClO_3^- + 6 e^-$                                                       & $\lra Cl^- + 6 OH^-$                                                            \\
                                  \hline % FUCK THIS SHIT
                                  $6 Co^{+2} + \cancel{18}{12} OH^- + \cancel{3 H_2O}  + ClO_3^- + \cancel{6 e^-}$ & $\lra 3 Co_2O_3 + \cancel{6 e^-} + \cancel{9}{6} H_2O + Cl^- + \cancel{6 OH^-}$
                              \end{tabular}
                              $6 Co^{+2} + 12 OH^- + ClO_3^- \lra 3 Co_2O_3 + 6 H_2O + Cl^-$ \\
                              Transladar los coeficientes a la ecuación original: \\
                              $\blue{6} CoCl_2 + \blue{12} KOH + \blue{1} KClO_3 \lra \blue{3} Co_2O_3 + \blue{1} KCl + \blue{6} H_2O$ \\
                              Igualar la cantidad de $K$. $12 + 1 \rightarrow 13$ \\
                              $6 CoCl_2 + 12 KOH + KClO_3 \lra 3 Co_2O_3 + \blue{13} KCl + 6 H_2O$ \\
                    \end{itemize}
          \end{enumerate}

    \item Ajustar las siguientes ecuaciones químicas por el método ion electrón en medio básico:
          \begin{enumerate}
              \item $Na_2SO_3 + NaOH + I_2 \lra Na_2SO_4 + NaI + H_2O$
              \item $KClO + KAsO_2 + KOH \lra K_3AsO_4 + KCl + H_2O$
          \end{enumerate}

    \item Ajustar las siguientes ecuaciones químicas por el método ion electrón en medio ácido:
          \begin{enumerate}
              \item $Sn + HNO_3 \lra SnO_2 + NO + H_2O$
              \item $KIO_3 + KI + H_2SO_4 \lra I_2 + H_2O + K_2SO_4$
          \end{enumerate}

    \item Ajustar las siguientes ecuaciones químicas por el método del número de oxidación en vía
          seca.
          \begin{enumerate}
              \item $ZnS + O_2 \lra ZnO + SO_2$
              \item $AgNO_3 + Cu \lra Cu(NO_3)_2 + Ag$
          \end{enumerate}
\end{enumerate}
\subsection{Bloque 2}
\begin{enumerate}
    \item Indiquen cuáles de los siguientes ejemplos de la vida cotidiana corresponden a reacciones
          químicas, y justifiquen su elección:
          \begin{enumerate}
              \item picar carne;
              \item putrefacción de una fruta;
              \item evaporación del alcohol;
              \item proceso de fotosíntesis;
              \item fundir un plástico;
              \item teñir el cabello;
              \item disolver sal en agua;
              \item pinchar un globo;
              \item quemar ramas;
              \item preparar un té;
              \item encender una hornalla;
              \item oxidación de un clavo.
          \end{enumerate}

    \item Para las siguientes ecuaciones:
          \begin{enumerate}
              \item $2 K \es + 2 H_2O \el \lra 2 KOH \eac + H_2 \eg$
              \item $C_2H_4 \eg + HF \eg \lra C_2H_5F \eg$
              \item $3 H_2SO_4 \eac + 2 Ni(OH)_3 \eac \lra Ni_2(SO_4)_3 \eac + 6 H_2O \el$
              \item $AgNO_3 \eac + NaI \eac \lra NaNO_3 \eac + AgI \es$
              \item $2 C_4H_{10} \eg + 13 O_2 \eg \lra 8 CO_2 \eg + 10 H_2O \eg$
              \item $4 H_3PO_3 \eac + 2 HNO_3 \eac \lra 4 H_3PO_4 \eac + N_2O \eg + H_2O \el$
          \end{enumerate}
          Indiquen:
          \begin{enumerate}
              \item el tipo de reacción química que representa (síntesis, descomposición, combustión,
                    precipitación, neutralización o redox);
              \item el agente oxidante, el agente reductor y el cambio que se produce en los estados de
                    oxidación, en las que representan a reacciones redox.
          \end{enumerate}

    \item El cloruro de calcio, sólido, se produce a partir de la reacción entre el calcio y el cloro.
          \begin{enumerate}
              \item Escriban la ecuación química que representa el proceso.
              \item Indiquen el tipo de reacción química que representa.
          \end{enumerate}
\end{enumerate}
\end{document}