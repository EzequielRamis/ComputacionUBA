\documentclass[../Teoría.root.tex]{subfiles}
 
\begin{document}

\section{Compuestos Químicos}
\label{sec:compuestos quimicos}
\subsection{Iónicos}
Están formados por dos átomos (metal con no metal) y/o moléculas, una es el catión\textsuperscript{1} y otra el anión. Cuando el compuesto es entre dos átomos, para que haya una unión iónica tiene que haber una diferencia de electronegatividad mayor o igual a \num{2}. La suma de las cargas de los iones tiene que ser igual a la carga total del compuesto.

\subsubsection{Tipos}
\begin{enumerate}
    \item Cationes metálicos monoatómicos: Un solo átomo de metal que pierde electrones
    
    \item Sales no oxigenadas: Anión no metálico (Excepto oxígeno) con un catión metálico
    
    \item Hidruros metálicos: Hidrógeno más un metal. Este es el único caso donde el \ch{H} funciona con Nº de oxidación \num{-1}. Ejemplos \ref{itm:nomenclatura.ej6} y \ref
    {itm:nomenclatura.ej7}

    \item Hidruros no metalicos: Hidrogeno mas un no metal
    
    \item Óxidos de metales: El oxígeno forma el anión \ch{O^2-} siempre, y se une a un catión metálico.
    
    \item Oxosales (Subsección \ref{subsec:oxosales})
    
\end{enumerate}
\textsuperscript{1} el único caso donde el catión es una molécula %(que yo sepa)
es en el \ch{NH4+}

\subsubsection{Nomenclatura}
\begin{enumerate}
    \item Para cationes metálicos monoatómicos:
          \begin{enumerate}
              \item Si tienen un solo número de oxidación, la fórmula es: “Ion [nombre]”
              
              \item Si tienen dos (o más) números de oxidación:
                    \begin{enumerate}
                        \item La nomenclatura moderna es: “Ion [nombre] ([numeral de Stock])”.
                        
                        \item La nomenclatura tradicional es: “Ion [nombre -oso/ico]”. “oso” para el menor nº de oxidación e “ico” para el mayor
                    \end{enumerate}
          \end{enumerate}

    \item Para sales no oxigenadas e hidruros metálicos se usa la fórmula “[no metal -uro] de [metal]”. El anión recibe la terminación “uro” y el catión sigue la regla 1 (excepto que sin “Ion”). Ejemplos \ref{itm:nomenclatura.ej4} y \ref{itm:nomenclatura.ej5} para sales, \ref{itm:nomenclatura.ej6} y \ref{itm:nomenclatura.ej7} para hidruros.
    
    \item Para óxidos de metales se usa “Óxido de [metal]”, donde el metal sigue la regla 1. Ejemplo \ref{itm:nomenclatura.ej8}.
    
\end{enumerate}
Ejemplos:
\begin{enumerate}
    \item \label{itm:nomenclatura.ej1} \ch{K+}: Ion potasio
    
    \item \label{itm:nomenclatura.ej2} \ch{Cu+}: Ion cobre (I) o Ion cuproso(“-oso” por ser el menor de los Nº de ox. del cobre: \num{1})
    
    \item \label{itm:nomenclatura.ej3} \ch{Cu^2+}: Ion cobre (II) o Ion cuprico(“-ico” por ser el mayor de los Nº de ox. del cobre: \num{2})
    
    \item \label{itm:nomenclatura.ej4} \ch{LiCl}: Cloruro de Litio (El litio tiene un solo Nº de ox., por lo tanto las dos nomenclaturas son iguales, y su nombre no recibe ni sufijo ni número de Stock)
    
    \item \label{itm:nomenclatura.ej5} \ch{CuF}: Fluoruro de Cobre (I) o Fluoruro Cuproso(El cobre tiene \num{2} Nº de ox., en CuF actúa con el menor: \num{1}, por lo que recibe el el Nº de Stock “I”)
    
    \item \label{itm:nomenclatura.ej6} \ch{CaH2}: Hidruro de Calcio
    
    \item \label{itm:nomenclatura.ej7} \ch{CuH}: Hidruro de Cobre (I) o Hidruro Cuproso
    
    \item \label{itm:nomenclatura.ej8} \ch{FeO}: Óxido de hierro (II) u Óxido ferroso
    
\end{enumerate}

\subsection{Moleculares}
% TODO: this

\subsection{Oxosales}
\label{subsec:oxosales}
% TODO: this

\end{document}