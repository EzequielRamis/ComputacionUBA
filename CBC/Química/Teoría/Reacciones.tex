\documentclass[../Teoría.root.tex]{subfiles}
 
\begin{document}

\section{Reacciones}
Cuando dos ó más moléculas o átomos interactúan para transformarse en otros compuestos se lo llama "Reacción química"; y dependiendo de las propiedades de dicha reacción, esta se puede clasificar en las siguientes maneras:

\subsection{Síntesis}
% Si esto esta mal, es culpa de Ramis
Cuando dos compuestos se unen para formar uno solo

Ej.:
\begin{equation}
    \ch{2 SO2} + \ch{O_2} \rightarrow\ \ch{2 SO3}
\end{equation}

\subsection{Descomposición}
Cuando un compuesto se divide en dos

Ej.:
\begin{equation}
    \ch{CaCO3} \rightarrow \ch{CaO} + \ch{CO_2}
\end{equation}

\subsection{Combustión}


\subsection{Precipitación}
% ?????????

\subsection{Neutralización}


\subsection{Redox}
``Oxido-Reducción'' (Ver \hyperref[sec:palabras clave]{Palabras clave})

Este tipo de reacción puede estar acompañada por cualquiera de las anteriores, ya que su propiedad es solo que haya un cambio en el número de oxidación entre reactivo y producto de al menos un átomo.
\\
\\
Ejemplo de una ecuación solo de redox:
\begin{equation}
    \sr{ \blue{(+4)}(-1) }{ \ch{TiCl4} }
     + 2 \sr{ \blue{(0)} }{ \ch{Mg} } 
    \ \lra \sr{ \blue{(0)} }{ \ch{Ti} }
     + 2 \sr{ \blue{(+2)}(-1) }{ \ch{MgCl2} }
\end{equation}
Ejemplo de una ecuación de síntesis y redox al mismo tiempo:
\begin{equation}
    \sr{ \blue{(+4)}(-2) }{ \ch{SO2} } + \sr{ \blue{(0)} }{ \ch{O2} } 
    \ \lra \sr{ \blue{(+6)(-2)} }{ \ch{SO3} }
\end{equation}


\end{document}