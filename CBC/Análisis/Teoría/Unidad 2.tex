% !TEX root = ../Teoría.root.tex

\documentclass[../Teoría.root.tex]{subfiles}

\begin{document}
    \section{Sucesiones}
    \subsection{Introducción}
    Las sucesiones son objetos matemáticos muy sencillos que se apoyan en la ordenación de un conjunto (finito o infinito) de números reales. Sirven, por ejemplo, para estudiar, representar y predecir los fenómenos que ocurren o se miden en el tiempo, en forma intermitente.
    \subsubsection{Un problema a modo de presentación}
    El problema consiste en encontrar un algoritmo que calcule la raíz cuadrada de un número dado (por ejemplo \(\sqrt{2}\)), utilizando sólo las cuatro operaciones básicas.\\
    Una solución al problema se basa en una idea geométrica:\\
    Se construyen sucesivos rectángulos todos de área 2. La base de cada uno de ellos es el promedio de la base y la altura del anterior.
    \begin{center}
        \begin{scaletikzpicturetowidth}{\linewidth}
            \begin{tikzpicture}[scale=\tikzscale]
                \node(rect)[draw=black,ultra thick, minimum width=2*1cm,minimum height=2*2cm,label=right:2,label=below:1](A){\(\textstyle1\cdot2=2\)};
                \node(rect)[right=of A,draw=black,ultra thick, minimum width=2*1.5cm,minimum height=2*1.333cm,label=right:\(\textstyle\frac{2}{1,5}\),label=below:\(\textstyle\text{\(\frac{1+2}{2}=1,5\)}\)](B){\(\textstyle1,5\cdot\frac{2}{1,5}=2\)};
                \node(rect)[right=of B,draw=black,ultra thick, minimum width=2*1.417cm,minimum height=2*1.412cm,label=right:\(\textstyle\text{1,4118}\),label=below:\(\textstyle\text{\(\frac{1,5+\frac{2}{1,5}}{2}\cong1,4167\)}\)](c){\(\scriptstyle1,4167\cdot1,4118\cong2\)};
            \end{tikzpicture}
        \end{scaletikzpicturetowidth}
    \end{center}
    Designamos con \(x_1\) a la medida de la base del primer rectángulo, que elegimos que fuera igual a 1, con \(x_2\) a lo que mide la base del segundo rectángulo, y así sucesivamente.\\
    Entonces resulta: \[x_1=1,x_2=\frac{x_1+\frac{2}{x_1}}{2}=1,5,x_3=\frac{x_2+\frac{2}{x_2}}{2}=1,4167,\dots,x_{n+1}=\frac{x_n+\frac{2}{x_n}}{2}\]
    Geométricamente se observa que los rectángulos se van aproximando a un cuadrado de área 2, por lo cual las bases \(x_n\) se van aproximando al lado del cuadrado de área 2, es decir \(x_n\rightarrow\sqrt{2}\) (\(x_n\) se aproxima a \(\sqrt{2}\)).\\
    \subsubsection{Ejemplos de sucesiones}
    Consideramos los siguientes ejemplos:
    \begin{enumerate}
        \item \(1,\frac{1}{2},\frac{1}{3},\frac{1}{4},\dots\)
        \item 1, 3, 5, 7, \dots
        \item \(-\frac{1}{2},\frac{1}{4},-\frac{1}{8},\frac{1}{16},\dots\)
        \item \(\frac{1}{2},\frac{2}{3},\frac{3}{4},\frac{4}{5},\dots\)
        \item 0, 1, 0, 1, \dots
        \item 2, -4, 6, -8, \dots
    \end{enumerate}
    Informalmente, una \textit{sucesión} es una \textit{lista ordenada} e infinita de números reales.\\
    Nos interesará el “comportamiento a la larga” de cada lista de números. En otras palabras, nos interesará saber si, a medida que avanzamos en la lista de números, éstos se parecen o aproximan a un número determinado. Habrá que dar más precisión a esta idea.\\
    Observemos por el momento, que una lista ordenada de números se puede describir con el lenguaje de las funciones que vimos en el primer módulo, usando como conjunto “ordenador” a los números naturales.\\
    Una \textit{sucesión} es una función \(a:\N\rightarrow\R\), se escribe \(a(n)=a_n\). Se lee “\(a\,sub\,n\)”. Indica el número real de la lista en la posición \(n\).\\
    Observemos la lista de sucesiones con las que comenzamos esta sección:\\
    En la sucesión \textbf{1.} \(a_3=\frac{1}{3}\) y \(a_{100}=\frac{1}{100}\); en la sección \textbf{4.} \(a_4=\frac{4}{5}\) y \(a_{1000}=\frac{1000}{1001}\).\\
    Lo que interesará es el comportamiento de \(a_n\) para “valores grandes” de \(n\).
    \subsubsection{Término general}
    Es la expresión de \(a_n\) para cada \(n\). Analizamos cada una de las sucesiones anteriores
    \begin{enumerate}
        \item \(a_n=\frac{1}{n}\)\tab\(1,\frac{1}{2},\frac{1}{3},\frac{1}{4},\dots\)
        \item \(a_n=2n-1\)\tab\(1,3,5,7,\dots\)
        \item \(a_n=(-1)^n\frac{1}{2^n}\)\tab\(-\frac{1}{2},\frac{1}{4},-\frac{1}{8},\frac{1}{16},\dots\)
        \item \(a_n=\frac{n}{n+1}\)\tab\(\frac{1}{2},\frac{2}{3},\frac{3}{4},\frac{4}{5},\dots\)
        \item \(a_n=\begin{cases}
            0 &\text{si \(n\) es impar}\\
            1 &\text{si \(n\) es par}
        \end{cases}\)\tab\(0,1,0,1,\dots\)
        \item \(a_n=(-1)^{n+1}2n\)\tab\(2,-4,6,-8,\dots\)
    \end{enumerate}
    \subsubsection{Representación gráfica}
    Como una sucesión es una función admite una representación gráfica, veamos alguno de los ejemplos anteriores:
    \begin{enumerate}
        \item \addtocounter{enumi}{1}\(1,\frac{1}{2},\frac{1}{3},\frac{1}{4},\dots\) 
        \begin{center}
            \begin{scaletikzpicturetowidth}{\linewidth}
                \begin{tikzpicture}[scale=\tikzscale]
                    \draw[->,thick] (0,-.1) -- (0,1.1);
                    \draw[->,thick] (-.1,0) -- (2.1,0);
                    \foreach \x/\xl/\y/\yl in {.2/1/1/1,.4/2/{1/2}/{1/2},.6/3/{1/3}/{1/3},.8/4/{1/4}/{1/4},1.4/\(n\)/{1/10}/{1/\(n\)}} {
                        \node[draw,shape=circle,fill=black,scale=0.4] at (\x,\y)(n){};
                        \node (yl) at (0,\y){};
                        \node (xl) at (\x,0){};
                        \draw[dashed] node[left=1mm of yl]{\(\scriptstyle\text{\yl}\)} (yl) -- (n) -- (\x,0) node[below=1mm of xl]{\xl};
                    }
                \end{tikzpicture}
            \end{scaletikzpicturetowidth}
        \end{center}
        \[a_n=\frac{1}{n}\rightarrow0\]
        \item \(-\frac{1}{2},\frac{1}{4},-\frac{1}{16},\dots\)
        \begin{center}
            \begin{scaletikzpicturetowidth}{\linewidth}
                \begin{tikzpicture}[scale=\tikzscale]
                    \draw[->,thick] (0,-1) -- (0,.6);
                    \draw[->,thick] (-.1,0) -- (2.1,0);
                    \foreach \x/\xl/\y/\yl in {.2/1/{-1/2}/{-1/2},.4/2/{1/4}/{1/4},.6/3/{-1/8}/{-1/8},.8/4/{1/16}/{1/16},1.4/\(n\)/0/ } {
                        \node[draw,shape=circle,fill=black,scale=0.4] at (\x,\y)(n){};
                        \node (yl) at (0,\y){};
                        \node (xl) at (\x,0){};
                        \draw[dashed] node[left=1mm of yl]{\(\scriptstyle\text{\yl}\)} (yl) -- (n) -- (\x,0) node[below=2mm of xl]{\xl};
                    }
                \end{tikzpicture}
            \end{scaletikzpicturetowidth}
        \end{center}
        \[a_n=(-1)^n\frac{1}{2^n}\rightarrow0\]
        \item \(\frac{1}{2},\frac{2}{3},\frac{3}{4},\frac{4}{5},\dots\)
        \begin{center}
            \begin{scaletikzpicturetowidth}{\linewidth}
                \begin{tikzpicture}[scale=\tikzscale]
                    \draw[->,thick] (0,-.1) -- (0,1.1);
                    \draw[->,thick] (-.1,0) -- (2.1,0);
                    \draw[thick] (0,1) -- node[above]{1} (2.1,1);
                    \foreach \x/\xl/\y/\yl in {.2/1/{1/2}/{1/2},.4/2/{2/3}/{2/3},.6/3/{3/4}/{3/4},.8/4/{4/5}/{4/5},1.4/\(n\)/{9/10}/\(a_n\)} {
                        \node[draw,shape=circle,fill=black,scale=0.4] at (\x,\y)(n){};
                        \node (yl) at (0,\y){};
                        \node (xl) at (\x,0){};
                        \draw[dashed] node[left=1mm of yl]{\(\scriptstyle\text{\yl}\)} (yl) -- (n) -- (\x,0) node[below=1mm of xl]{\xl};
                    }
                \end{tikzpicture}
            \end{scaletikzpicturetowidth}
        \end{center}
        \[a_n=\frac{n}{n+1}\rightarrow1\]
    \end{enumerate}
    \subsection{Noción de límite}
    Intuitivamente, una sucesión \textit{tiende} a un valor determinado \(L\) si los valores de \(a_n\) \textit{están cerca} de \(L\) cuando \(n\) es grande. Un poco más precisamente: el error que se comete al aproximar \(L\) con \(a_n\) es pequeño (menor que épsilon (\(\epsilon\))) si \(n\) es bastante grande (más que \(n_0\) en el gráfico)
    \begin{center}
        \begin{scaletikzpicturetowidth}{\linewidth}
            \begin{tikzpicture}[scale=\tikzscale]
                \draw[->,thick] (0,-.1) -- (0,1.1);
                \draw[->,thick] (-.1,0) -- (2,0);
                \draw[dashed] (0,.8) node[left]{\(L+\epsilon\)} -- (2,.8);
                \draw[dashed] (0,.5) node[left]{\(L\)} -- (2,.5);
                \draw[dashed] (0,.2) node[left]{\(L-\epsilon\)} -- (2,.2);
                \draw[<->,dashed,thick] (.5,.5) -- (.5,.8) node[above]{\(\epsilon\)};
                \draw[dashed] (1.2,0) node[below]{\(n_0\)} -- (1.2,1);
                \fill[green,opacity=0.2] (1.2,.2) -- (1.2,.8) -- (2,.8) -- (2,.2) -- cycle;
                \foreach \x/\y in {.2/.9,.3/.3,.6/.1,.8/.6,1/.9,1.3/.75,1.5/.4,1.7/.6,1.9/.45} {
                    \node[draw,shape=circle,fill=black,scale=0.4] at (\x,\y){};
                }
                \node[draw,text width=3.5cm] at (3.5,.5) {\textbf{Idea geométrica}\\A partir de \(n_0\) la franja verde capta a todos los \(a_n\)};
            \end{tikzpicture}
        \end{scaletikzpicturetowidth}
    \end{center}
    La definición precisa de límite es la siguiente:\\
    Se dice que \(a_n\) tiene \textit{límite L} si, cualquiera sea \(\epsilon>0\), existe un número natural \(n_0\) tal que si \(n\geq0\), entonces \[L-\epsilon<a_n<L+\epsilon\,\text{(o sea \(|a_n-L|<\epsilon\,\text{si}\,n\geq n_0\))}\]
    Se escribe \(\lim_{n\to\infty}a_n=L\) o \(a_n\rightarrow L\), se lee "\textit{el límite de a sub n cuando n tiende a infinito es L}".
    También se dice en tal caso que \(a_n\in(L-\epsilon,L+\epsilon)\) \textit{para casi todo n (pctn)}. En general, una propiedad vale \textit{para casi todo n} si vale para todo \(n\) salvo un número finito de valores de \(n\). Se pone \(pctn\).
    \subsubsection{Sucesiones divergentes}
    No todas las sucesiones convergen a un límite \(L\in\R\), por ejemplo:
    \begin{itemize}
        \item \(a_n=2n-1\)\tab\(1,3,5,7,\dots\)
        \begin{center}
            \begin{scaletikzpicturetowidth}{.5\linewidth}
                \begin{tikzpicture}[scale=\tikzscale]
                    \draw[->,thick] (0,-1) -- (0,10);
                    \draw[->,thick] (-1,0) -- (10,0);
                    \foreach \x/\xl/\y/\yl in {1/1/1/1,2/2/3/3,3/3/5/5,4/\(n_0\)/7/\(K\),5/\(n\)/9/\(2n-1\)} {
                        \node[draw,shape=circle,fill=black,scale=0.4] at (\x,\y)(n){};
                        \node (yl) at (0,\y){};
                        \node (xl) at (\x,0){};
                        \draw[dashed] node[left=1mm of yl]{\(\textstyle\text{\yl}\)} (yl) -- (n) -- (\x,0) node[below=1mm of xl]{\xl};
                    }
                    \draw[dashed,ultra thin] (1,1) -- (5,9);
                    \fill[green,opacity=0.2] (4,7) -- (4,10) -- (6,10) -- (6,7) -- cycle;
                \end{tikzpicture}
            \end{scaletikzpicturetowidth}
        \end{center}
        \[\lim_{n\to\infty}(2n-1)=+\infty\] Para todo \(K>0\) existe \(n_0\in\N\) tal que si \(n>n_0,2n-1>K\) (en el sector verde).
        \item \(b_0=\begin{cases}
            0 & \text{si \(n\) es impar}\\
            1 & \text{si \(n\) es par}
        \end{cases}\)\tab\(0,1,0,1,\dots\)
        \begin{center}
            \begin{scaletikzpicturetowidth}{.5\linewidth}
                \begin{tikzpicture}[scale=\tikzscale]
                    \draw[->,thick] (0,-1) -- (0,6);
                    \draw[->,thick] (-.5,0) -- (10,0);
                    \foreach \y [count=\x] in {0,1,0,1,0,1} {
                        \node[draw,shape=circle,fill=black,scale=0.4] at (\x,\y*5)(n){};
                        \node (yl) at (0,\y*5){};
                        \node (xl) at (\x,0){};
                        \draw node[left=1mm of yl]{\y} (yl);
                        \draw[dashed] (n) -- (\x,0) node[below=1mm of xl]{\x};
                    }
                    \draw[dashed] (0,5) -- (10,5);
                \end{tikzpicture}
            \end{scaletikzpicturetowidth}
        \end{center}
        \[\lim_{n\to\infty}b_n\,\text{no existe}\] En este caso se dice que \(a_n\) oscila finitamente.
        \item \(c_n=(-1)^{n-1}2n\)\tab\(2,-4,6,-8,\dots\)
        \begin{center}
            \begin{scaletikzpicturetowidth}{.5\linewidth}
                \begin{tikzpicture}[scale=\tikzscale]
                    \draw[->,thick] (0,-11) -- (0,11);
                    \draw[->,thick] (-1,0) -- (20,0);
                    \foreach \y [count=\x] in {2,-4,6,-8} {
                        \node[draw,shape=circle,fill=black,scale=0.4] at (\x*2,\y)(n){};
                        \node (yl) at (0,\y){};
                        \node (xl) at (\x*2,0){};
                        \draw[dashed] node[left=1mm of yl]{\y} (yl) -- (n) -- (\x*2,0) node[below=1mm of xl]{\x};
                    }
                \end{tikzpicture}
            \end{scaletikzpicturetowidth}
        \end{center}
        Se dice que \(c_n\) tiende a infinito (sin especificar el signo) o que oscila infinitamente.
    \end{itemize}
    \subsection{Propiedades del límite}
    La mayoría de las veces, el problema consistirá en calcular el valor de \(\lim_{n\to\infty}a_n\). La definición no será útil para ello porque presupone conocer el valor de \(L\), de modo que nos valdremos de propiedades y diversos recursos algebraicos para poder determinar el valor del límite en los ejemplos que estudiemos. La definición de límite es imprescindible para poder obtener esas propiedades y para introducir casi todos los conceptos de la materia que se basan en esta noción. En la práctica no haremos un uso directo de dicha definición.\\
    Las siguientes propiedades se deducen de la definición de límite y nos servirán para desarrollar técnicas que nos permitan calcular algunos límites.
    \subsubsection{Unicidad del límite}
    \begin{center}
        \begin{scaletikzpicturetowidth}{.5\linewidth}
            \begin{tikzpicture}[scale=\tikzscale]
                \draw[->,thick] (0,-1) -- (0,5);
                \draw[->,thick] (-1,0) -- (10,0);
                \foreach \x/\xl in {1/1,2/2,3/3,4/4,6/\(n\)} {
                    \draw[thick] (\x,.25) -- (\x,-.25) node[below]{\xl};
                }
                \draw[dashed, red] (6,0) -- (6,5);
                \foreach \y [count=\i] in {1.5,3} {
                    \draw[thick] (0,\y+.5) -- (10,\y+.5);
                    \draw[dashed, red] (0,\y) node[left]{\(L_\i\)} -- (10,\y);
                    \draw[thick] (0,\y-.5) -- (10,\y-.5);
                }
            \end{tikzpicture}
        \end{scaletikzpicturetowidth}
    \end{center}
    Una sucesión no puede converger a dos límites distintos. Si así fuera todos los \(a_n\) a partir de \(n=n_0\) tendrían que estar simultáneamente en las dos franjas y eso no es posible.
    \subsubsection{Acotación de las sucesiones convergentes}
    Si \(a_n\) es convergente, entonces el conjunto \(A=\{a_n:n\in\N\}\) es acotado.
    \begin{center}
        \begin{scaletikzpicturetowidth}{.5\linewidth}
            \begin{tikzpicture}[scale=\tikzscale]
                \draw[->,thick] (0,-1) -- (0,7);
                \draw[->,thick] (-1,0) -- (11,0);
                \draw[thick] (5.5,.25) -- (5.5,-.25) node[below]{\(n_0\)};
                \draw[dashed] (5.5,0) -- (5.5,7);
                \draw[red] (0,1) node[left]{\(m\)} -- (11,1);
                \draw[dashed] (0,2) node[left]{\(L-1\)} -- (11,2);
                \draw[dashed] (0,3) node[left]{\(L\)} -- (11,3);
                \draw[dashed] (0,4) node[left]{\(L+1\)} -- (11,4);
                \draw[red] (0,5) node[left]{\(M\)} -- (11,5);
                \fill[green,opacity=0.2] (5.5,2) -- (5.5,4) -- (11,4) -- (11,2) -- cycle;
                \foreach \y [count=\x] in {4.5,2.2,1.2,3.5,4.6,3.8,2.6,3.3,2.7,3.2} {
                        \node[draw,shape=circle,fill=black,scale=0.4] at (\x,\y){};
                }
            \end{tikzpicture}
        \end{scaletikzpicturetowidth}
    \end{center}
    Se elige \(\epsilon=1\) o cualquier otro. Una vez que \(a_n\) queda dentro de la franja roja para \(n\geq n_0\), es fácil encontrar cotas superiores e inferiores de \(A\). \(M\) es cota superior y \(m\) es cota inferior de \(A\).
    \subsubsection{Conservación de signo}
    Si \(a_n\) converge a un límite \(L\) mayor que cero, entonces la sucesión \(a_n\) es mayor que cero para casi todo \(n\). Es decir: \[\text{Si }\lim_{n\to\infty}a_n=L>0\text{ entonces }a_n>0\,pctn\]
    \begin{center}
        \begin{scaletikzpicturetowidth}{.5\linewidth}
            \begin{tikzpicture}[scale=\tikzscale]
                \draw[->,thick] (0,-1) -- (0,7);
                \draw[->,thick] (-1,0) -- (11,0);
                \draw[thick] (5.5,.25) -- (5.5,-.25) node[below]{\(n_0\)};
                \draw[dashed] (5.5,0) -- (5.5,7);
                \draw[dashed] (0,2) node[left]{\(L-L/2\)} -- (11,2);
                \draw[dashed] (0,3) node[left]{\(L\)} -- (11,3);
                \draw[dashed] (0,4) node[left]{\(L+L/2\)} -- (11,4);
                \fill[green,opacity=0.2] (5.5,2) -- (5.5,4) -- (11,4) -- (11,2) -- cycle;
                \foreach \y [count=\x] in {4.5,-.8,1.2,3.5,-.4,3.8,2.6,3.3,2.7,3.2} {
                        \node[draw,shape=circle,fill=black,scale=0.4] at (\x,\y){};
                }
            \end{tikzpicture}
        \end{scaletikzpicturetowidth}
    \end{center}
    Se elige \(\epsilon=L/2\), de modo que \(L-L/2=L/2>0\). Esto asegura que la franja verde esté por encime del eje de las \(x\). Así \(a_n>0\) si \(n\geq n_0\).
    \subsubsection{Álgebra de límites}
    Si \(a_n\rightarrow a\in\R\) y \(b_n\rightarrow b\in\R\), entonces
    \begin{itemize}
        \item \(a_n+b_n\rightarrow a+b\)
        \item \(a_n\cdot b_n\rightarrow ab\). En particular \(k\cdot a_n\rightarrow ka\) si \(k\in\R\)
        \item Si \(b\neq0\) entonces \(\frac{a_n}{b_n}\rightarrow\frac{a}{b}\)
        \item \(|a_n|\rightarrow|a|\)
        \item Si \(a>0\) entonces \((a_n)^{b_n}\rightarrow a^b\)
    \end{itemize}
    Ejemplo. Calcular \(\lim_{n\to\infty}=\left(\frac{1}{n}+\frac{2n}{n+1}\right)^3\).\\
    Tenemos que \[\lim_{n\to\infty}\frac{1}{n}=0\] \[\lim_{n\to\infty}\frac{2n}{n+1}=2\cdot\lim_{n\to\infty}\frac{n}{n+1}=2\cdot1=2\]
    Entonces: \[\lim_{n\to\infty}=\left(\frac{1}{n}+\frac{2n}{n+1}\right)^3=(0+2)^3=8\]
    \subsubsection{Intederminaciones}
    El álgebra de límites requiere que las sucesiones involucradas sean convergentes a un número real. Por ello, no podemos aplicar el álgebra de límites en forma directa en, por ejemplo \(\lim_{a\to\infty}\frac{3n^2+2}{2n^2+5n}\), ya que un primer análisis de la sucesión nos dice que tanto numerador como denominador tienden a más infinito y el teorema de álgebra de límites se refiere a valores numéricos del límite.\\
    Se suele decir que estamos en presencia de una \textit{indeterminación} en este caso, del tipo \(\frac{\infty}{\infty}\) entendiendo este símbolo como el cociente de sucesiones que tienden ambas a infinito. El nombre de indeterminación es porque no hay una propiedad general que nos indique el valor del límite en una situación como esta.\\
    Es necesario, en cada caso, aplicar alguna técnica algebraica que permita “salvar” la indeterminación y calcular el límite. No es el único tipo de indeterminación con el que nos vamos a encontrar. Analicemos los siguientes ejemplos:
    \begin{itemize}
        \item \(a_n=\frac{1}{n}\rightarrow0,b_n=n\rightarrow+\infty\):\tab\(a_nb_n=\frac{1}{n}n=1\rightarrow1\)
        \item \(a_n=\frac{1}{n}\rightarrow0,b_n=n^2\rightarrow+\infty\):\tab\(a_nb_n=\frac{1}{n}n^2=n\rightarrow+\infty\)
        \item \(a_n=\frac{1}{n^2}\rightarrow0,b_n=n\rightarrow+\infty\):\tab\(a_nb_n=\frac{1}{n^2}n=\frac{1}{n}\rightarrow0\)
    \end{itemize}
    En estos observamos que no existe una propiedad que pueda predecir sobre un límite del tipo \(0\cdot\infty\).\\
    Los límites de los siguientes “tipos”, aunque no son todos, constituyen indeterminaciones:
    \begin{itemize}
        \item \(\frac{\infty}{\infty}\)
        \item \(+\infty-\infty\)
        \item \(0\cdot\infty\)
        \item \(\frac{0}{0}\)
        \item \((+\infty)^0\)
        \item \(0^0\)
    \end{itemize}
    En todos los casos, hay que entender estos símbolos como el límite de la operación aritmética indicada en cada caso, entre dos sucesiones.\\
    Como vimos, el álgebra de límites requiere que las sucesiones involucradas sean convergentes a un número real. Cuando esto no ocurre, a veces se presentan indeterminaciones. En cada caso hay que usar algún recurso algebraico que permita salvar la indeterminación y calcular el valor del límite.\\
    En ocasiones no es posible aplicar el álgebra de límites porque los límites involucrados no son finitos, sin embargo no estamos ante una indeterminación. A continuación damos algunas situaciones donde se puede saber el límite a pesar de que los límites involucrados no sean todos números reales.
    \begin{itemize}
        \item \((+\infty)\cdot L=+\infty\) si \(L>0\)\tab Ej: \(\lim_{n\to\infty}\sqrt{n}\left(\frac{n}{3n+1}\right)=+\infty\)
        \item \((+\infty)+\infty=+\infty\)\tab Ej: \(\lim_{n\to\infty}\frac{n^2}{n+1}+\sqrt{n}=+\infty\)
        \item \((+\infty)+\) oscila finitamente \(=+\infty\)\tab Ej: \(\lim_{n\to\infty}(n+\cos(n))=+\infty\)
        \item \(\frac{0}{\infty}=0\)\tab Ej: \(\lim_{n\to\infty}\frac{\frac{1}{n}}{n^2+1}=0\)
        \item \((0)^{+\infty}=0\)\tab Ej: \(\lim_{n\to\infty}\left(\frac{n}{n^2+1}\right)^n=0\)
    \end{itemize}
    Cada una de estas afirmaciones se puede demostrar a partir de la definición de límite.
\end{document}