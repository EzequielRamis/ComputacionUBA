% !TEX root = ../Teoría.root.tex

\documentclass[../Teoría.root.tex]{subfiles}

\begin{document}
    \section{Sucesiones}
    \subsection{Introducción}
    Las sucesiones son objetos matemáticos muy sencillos que se apoyan en la ordenación de un conjunto (finito o infinito) de números reales. Sirven, por ejemplo, para estudiar, representar y predecir los fenómenos que ocurren o se miden en el tiempo, en forma intermitente.
    \subsubsection{Un problema a modo de presentación}
    El problema consiste en encontrar un algoritmo que calcule la raíz cuadrada de un número dado (por ejemplo \(\sqrt{2}\)), utilizando sólo las cuatro operaciones básicas.\\
    Una solución al problema se basa en una idea geométrica:\\
    Se construyen sucesivos rectángulos todos de área 2. La base de cada uno de ellos es el promedio de la base y la altura del anterior.
    \begin{center}
        \begin{scaletikzpicturetowidth}{\linewidth}
            \begin{tikzpicture}[scale=\tikzscale]
                \node(rect)[draw=black,ultra thick, minimum width=2*1cm,minimum height=2*2cm,label=right:2,label=below:1](A){\(\textstyle1\cdot2=2\)};
                \node(rect)[right=of A,draw=black,ultra thick, minimum width=2*1.5cm,minimum height=2*1.333cm,label=right:\(\textstyle\frac{2}{1,5}\),label=below:\(\textstyle\text{\(\frac{1+2}{2}=1,5\)}\)](B){\(\textstyle1,5\cdot\frac{2}{1,5}=2\)};
                \node(rect)[right=of B,draw=black,ultra thick, minimum width=2*1.417cm,minimum height=2*1.412cm,label=right:\(\textstyle\text{1,4118}\),label=below:\(\textstyle\text{\(\frac{1,5+\frac{2}{1,5}}{2}\cong1,4167\)}\)](c){\(\scriptstyle1,4167\cdot1,4118\cong2\)};
            \end{tikzpicture}
        \end{scaletikzpicturetowidth}
    \end{center}
    Designamos con \(x_1\) a la medida de la base del primer rectángulo, que elegimos que fuera igual a 1, con \(x_2\) a lo que mide la base del segundo rectángulo, y así sucesivamente.\\
    Entonces resulta: \[x_1=1,x_2=\frac{x_1+\frac{2}{x_1}}{2}=1,5,x_3=\frac{x_2+\frac{2}{x_2}}{2}=1,4167,\dots,x_{n+1}=\frac{x_n+\frac{2}{x_n}}{2}\]
    Geométricamente se observa que los rectángulos se van aproximando a un cuadrado de área 2, por lo cual las bases \(x_n\) se van aproximando al lado del cuadrado de área 2, es decir \(x_n\rightarrow\sqrt{2}\) (\(x_n\) se aproxima a \(\sqrt{2}\)).\\
    \subsubsection{Ejemplos de sucesiones}
    Consideramos los siguientes ejemplos:
    \begin{enumerate}
        \item \(1,\frac{1}{2},\frac{1}{3},\frac{1}{4},\dots\)
        \item 1, 3, 5, 7, \dots
        \item \(-\frac{1}{2},\frac{1}{4},-\frac{1}{8},\frac{1}{16},\dots\)
        \item \(\frac{1}{2},\frac{2}{3},\frac{3}{4},\frac{4}{5},\dots\)
        \item 0, 1, 0, 1, \dots
        \item 2, -4, 6, -8, \dots
    \end{enumerate}
    Informalmente, una \textit{sucesión} es una \textit{lista ordenada} e infinita de números reales.\\
    Nos interesará el “comportamiento a la larga” de cada lista de números. En otras palabras, nos interesará saber si, a medida que avanzamos en la lista de números, éstos se parecen o aproximan a un número determinado. Habrá que dar más precisión a esta idea.\\
    Observemos por el momento, que una lista ordenada de números se puede describir con el lenguaje de las funciones que vimos en el primer módulo, usando como conjunto “ordenador” a los números naturales.\\
    Una \textit{sucesión} es una función \(a:\N\rightarrow\R\), se escribe \(a(n)=a_n\). Se lee “\(a\,sub\,n\)”. Indica el número real de la lista en la posición \(n\).\\
    Observemos la lista de sucesiones con las que comenzamos esta sección:\\
    En la sucesión \textbf{1.} \(a_3=\frac{1}{3}\) y \(a_{100}=\frac{1}{100}\); en la sección \textbf{4.} \(a_4=\frac{4}{5}\) y \(a_{1000}=\frac{1000}{1001}\).\\
    Lo que interesará es el comportamiento de \(a_n\) para “valores grandes” de \(n\).
    \subsubsection{Término general}
    Es la expresión de \(a_n\) para cada \(n\). Analizamos cada una de las sucesiones anteriores
    \begin{enumerate}
        \item \(a_n=\frac{1}{n}\)\tab\(1,\frac{1}{2},\frac{1}{3},\frac{1}{4},\dots\)
        \item \(a_n=2n-1\)\tab\(1,3,5,7,\dots\)
        \item \(a_n=(-1)^n\frac{1}{2^n}\)\tab\(-\frac{1}{2},\frac{1}{4},-\frac{1}{8},\frac{1}{16},\dots\)
        \item \(a_n=\frac{n}{n+1}\)\tab\(\frac{1}{2},\frac{2}{3},\frac{3}{4},\frac{4}{5},\dots\)
        \item \(a_n=\begin{cases}
            0 &\text{si \(n\) es impar}\\
            1 &\text{si \(n\) es par}
        \end{cases}\)\tab\(0,1,0,1,\dots\)
        \item \(a_n=(-1)^{n+1}2n\)\tab\(2,-4,6,-8,\dots\)
    \end{enumerate}
    \subsubsection{Representación gráfica}
    Como una sucesión es una función admite una representación gráfica, veamos alguno de los ejemplos anteriores:
    \begin{enumerate}
        \item \(1,\frac{1}{2},\frac{1}{3},\frac{1}{4},\dots\) 
        \begin{center}
            \begin{scaletikzpicturetowidth}{\linewidth}
                \begin{tikzpicture}[scale=\tikzscale]
                    \draw[->,thick] (0,-.1) -- (0,1.1);
                    \draw[->,thick] (-.1,0) -- (2.1,0);
                    \foreach \x/\xl/\y/\yl in {.2/1/1/1,.4/2/{1/2}/{1/2},.6/3/{1/3}/{1/3},.8/4/{1/4}/{1/4},1.4/\(n\)/{1/10}/{1/\(n\)}} {
                        \node[draw,shape=circle,fill=black,scale=0.4] at (\x,\y)(n){};
                        \node (yl) at (0,\y){};
                        \node (xl) at (\x,0){};
                        \draw[dashed] node[left=1mm of yl]{\yl} (yl) -- (n) -- (\x,0) node[below=1mm of xl]{\xl};
                    }
                \end{tikzpicture}
            \end{scaletikzpicturetowidth}
        \end{center}
    \end{enumerate}
\end{document}