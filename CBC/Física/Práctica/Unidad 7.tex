\documentclass[Física - Práctica.root.tex]{subfiles}

\newcommand{\gravity}[1][per-mode=fraction]{\SI[#1]{9,8}{\meter\per\second\squared}}

\begin{document}

\section{Unidad 7}
\subsection{Parte 1}
\begin{enumerate}
  \item
        \begin{multicols}{2}
          Guillermo entrena arrastrando \SI{20}{\meter} una bigornia que
          está vinculada a su cintura por medio de una soga como
          muestra la figura. El ángulo que forma la soga con el piso es
          de \ang{35} y la fuerza que ejerce Guillermo es de \SI{420}{\newton}. Si una
          fuerza de fricción de \SI{320}{\newton} se opone al movimiento.
          Calcule el trabajo total realizado. \\
          \begin{center}
            \begin{tikzpicture}[scale=1.2]
              \draw (-0.5,-0.5) rectangle ++(1,1);
              \filldraw[red] (0,0) circle [radius=0.05];
              \begin{scope}[very thick, red, ->, near end] % Vectores
                \draw (0,0) -- node[above]{$\vec{F}_a$}(35:2) node (F);
                \draw (0,0) -- node[above]{$\vec{f}_k$}(-1.5,0);
              \end{scope}
              % Angulo
              \draw[help lines] (0,0) -- (F |- 2,0);
              \draw[help lines] (0.7,0) arc[start angle=0, end angle=35, radius=0.7] node[pos=0.5, right]{$\theta$};
              % Desplazamiento
              \draw[gray!70, very thin] (-3,0) ++(-0.5,-0.5) rectangle ++(1,1);
              \draw[dashed, |->] (-3,-1) -- node[below]{$\vec{s}$}(0,-1);
            \end{tikzpicture}
          \end{center}
        \end{multicols}
        \begin{center}
          \[ W = \vec{s}\cdot\vec{F} \]
          \[ F_x = F_a\cos\theta - f_s \]
          \[ F_x = \SI{344}{\newton} - \SI{320}{\newton} \]
          \[ F_x = \SI{24}{\newton} \]
          \[ W_x = \SI{20}{\meter}\cdot\SI{24}{\newton} \]
          \[ \boxed{W_x = \SI{480}{\joule}} \]
        \end{center}

  \item
        \begin{multicols}{2}
          Dos bloques están unidos por una cuerda muy ligera que pasa por una polea sin masa y
          sin fricción, como muestra la figura. Los bloques se están desplazando a rapidez
          constante, el bloque A de \SI{20}{\newton} se mueve \SI{75,0}{\centi\meter} hacia la derecha y
          el bloque B de \SI{12,0}{\newton} se mueve \SI{75,0}{\centi\meter} hacia abajo. Durante el proceso, \\

          \begin{center}
            \begin{tikzpicture}
              \draw (-0.5,0.5) rectangle ++(1,1) node[midway](A){$A$};
              \draw (A) ++(0.5,0) -- ++(1.5,0) arc(90:0:0.3) node[midway](c) -- ++(0,-0.5) node(B);
              \draw (B) ++(0.5,0) rectangle ++(-1,-1) node[midway]{$B$};
              \draw (c) ++(-0.15,-0.15) circle[radius=0.15];
            \end{tikzpicture}
          \end{center}
        \end{multicols}
        \begin{enumerate}
          \item ¿Cuánto trabajo efectúan sobre el bloque B: la gravedad y la tensión de la cuerda?
                \begin{center}
                  \[ W_g = s_y\cdot F_g \]
                  \[ W_g = \SI{0,75}{\meter}\cdot\SI{12,0}{\newton} \]
                  \[ \boxed{W_g = \SI{9,0}{\joule}} \]
                  \[ \sum F_y = \SI{0}{\newton} \]
                  \[ \sum F_y = F_g + T_c = \SI{0}{\newton} \]
                  \[ \SI{12,0}{\newton} + T_c = \SI{0}{\newton} \]
                  \[ T_c = \SI{-12,0}{\newton} \]
                  \[ W_c = s_y\cdot T_C \]
                  \[ W_c = \SI{0,75}{\meter}\cdot\SI{-12,0}{\newton} \]
                  \[ \boxed{W_c = \SI{-9,0}{\joule}} \]
                \end{center}
          \item ¿Cuánto trabajo efectúan sobre el bloque A: la gravedad, la tensión de la cuerda, la fricción y la fuerza normal?
                \begin{center}

                  \[ W_g = s_y\cdot F_g \]
                  \[ W_g = \SI{0}{\meter}\cdot\SI{20}{\newton} \]
                  \[ \boxed{W_g = \SI{0}{\joule}} \]

                  \[ W_c = s_x\cdot T_c \]
                  \[ W_c = \SI{0,75}{\meter}\cdot\SI{12,0}{\newton} \]
                  \[ \boxed{W_c = \SI{9,0}{\joule}} \]

                  \[ a_x = 0 \Rightarrow F_x = \SI{0}{\newton} \]
                  \[ \sum F_x = \SI{0}{\newton} \]
                  \[ \sum F_x = T_c + f_k = \SI{0}{\newton} \]
                  \[ \SI{12,0}{\newton} + f_k = \SI{0}{\newton} \]
                  \[ f_k = \SI{-12,0}{\newton} \]
                  \[ W_f = s_x\cdot\SI{-12,0}{\newton} \]
                  \[ W_f = \SI{0,75}{\meter}\cdot\SI{-12,0}{\newton} \]
                  \[ \boxed{W_f = \SI{-9,0}{\joule}} \]

                  \[ W_n = s_y\cdot F_n \]
                  \[ W_n = \SI{0,0}{\meter}\cdot\SI{20}{\newton} \]
                  \[ \boxed{W_n = \SI{0}{\joule}} \]
                \end{center}
          \item Obtenga el trabajo total efectuado sobre cada bloque. \\
                Para el bloque B:
                \begin{center}
                  \[ \sum F_x = \SI{0}{\newton} \]
                  \[ \sum F_y = F_g - T_c \]
                  \[ \sum F_y = \SI{12,0}{\newton} - \SI{12,0}{\newton} \]
                  \[ \sum F_y = \SI{0,0}{\newton} \]
                  \[ W = \vec{s}\cdot\vec{F} \]
                  \[ W = \SI{0,75}{\meter}\vec{i}\cdot\SI{0,0}{\newton}\vec{i} \]
                  \[ \boxed{W = \SI{0,0}{\joule}} \]
                \end{center}
                Para el bloque A:
                \begin{center}
                  \[ \sum F_x = T_c + f_k \]
                  \[ \sum F_x = \SI{12,0}{\newton} + \SI{-12,0}{\newton} \]
                  \[ \sum F_x = \SI{0,0}{\newton} \]
                  \[ \sum F_y = F_g + F_n \]
                  \[ \sum F_y = \SI{20}{\newton} + \SI{-20}{\newton} \]
                  \[ \sum F_y = \SI{0}{\newton} \]
                  \[ W = \vec{s}\cdot\vec{F} \]
                  \[ 
                      W = \SI{0,75}{\meter}\vec{i}\cdot\SI{0,0}{\newton}\vec{i}
                        + \SI{0}{\meter}\vec{j}\cdot\SI{0}{\newton}\vec{j}
                   \]
                  \[ 
                      W = \SI{0,0}{\joule}
                        + \SI{0}{\joule}
                   \]
                  \[ \boxed{W = \SI{0}{\joule}} \]
                \end{center}
        \end{enumerate}
  
  \item Considere una situación igual a la del problema anterior, pero suponga ahora que no hay
        fuerza de rozamiento sobre el bloque A de \SI{20,0}{\newton} que descansa sobre la mesa. La polea
        es ligera y sin fricción.
        \begin{enumerate}
          \item Calcule la tensión T en la cuerda ligera que une los bloques
          \item Para un desplazamiento en el cual el bloque de \SI{12,0}{\newton} desciende \SI{1,20}{\meter}, 
                calcule el trabajo total realizado sobre: el bloque A y el bloque B.
          \item Para el desplazamiento del inciso b), calcule el trabajo total realizado sobre el sistema
                de dos bloques. ¿Cómo se compara su respuesta con el trabajo realizado sobre el bloque
                de \SI{12,0}{\newton} por la gravedad?
          \item Si el sistema se libera del reposo, ¿cuál es la rapidez del bloque de \SI{12,0}{\newton} cuando ha
                descendido \SI{1,20}{\meter}?
          
        \end{enumerate}

  \item Un vagón de juguete con masa de \SI{7,00}{\kilo\gram} se mueve en línea recta sobre una superficie
        horizontal sin fricción. Tiene una rapidez inicial de \SI{4,00}{\meter\per\second} y luego es empujado a lo
        largo de \SI{3}{\meter}, en la dirección de la velocidad inicial, por una fuerza cuya magnitud es de
        \SI{10,0}{\newton}.
        \begin{enumerate}
          \item Use el teorema de trabajo y energía para calcular la rapidez final del vagón
          \item Calcule la aceleración producida por la fuerza y úsela para calcular la rapidez final
                del vagón con la fórmula utilizada en cinemática. Compare este resultado con el del
                inciso a).
        \end{enumerate}
\end{enumerate}

\subsection{Parte 2}
\begin{enumerate}

\end{enumerate}

\end{document}