\documentclass[Física - Práctica.root.tex]{subfiles}

\newcommand{\gravity}[1][per-mode=fraction]{\SI[#1]{9,8}{\meter\per\second\squared}}

\begin{document}

\section{Unidad 7}
\subsection{Parte 1}
\begin{enumerate}
  \item
        \begin{multicols}{2}
          Guillermo entrena arrastrando \SI{20}{\meter} una bigornia que
          está vinculada a su cintura por medio de una soga como
          muestra la figura. El ángulo que forma la soga con el piso es
          de \ang{35} y la fuerza que ejerce Guillermo es de \SI{420}{\newton}. Si una
          fuerza de fricción de \SI{320}{\newton} se opone al movimiento.
          Calcule el trabajo total realizado. \\
          \begin{center}
            \begin{tikzpicture}[scale=1.2]
              \draw (-0.5,-0.5) rectangle ++(1,1);
              \filldraw[red] (0,0) circle [radius=0.05];
              \begin{scope}[very thick, red, ->, near end] % Vectores
                \draw (0,0) -- node[above]{$\vec{F}_a$}(35:2) node (F);
                \draw (0,0) -- node[above]{$\vec{f}_k$}(-1.5,0);
              \end{scope}
              % Angulo
              \draw[help lines] (0,0) -- (F |- 2,0);
              \draw[help lines] (0.7,0) arc[start angle=0, end angle=35, radius=0.7] node[pos=0.5, right]{$\theta$};
              % Desplazamiento
              \draw[gray!70, very thin] (-3,0) ++(-0.5,-0.5) rectangle ++(1,1);
              \draw[dashed, |->] (-3,-1) -- node[below]{$\vec{s}$}(0,-1);
            \end{tikzpicture}
          \end{center}
        \end{multicols}
        \begin{center}
          \[ W = \vec{s}\cdot\vec{F} \]
          \[ F_x = F_a\cos\theta - f_s \]
          \[ F_x = \SI{344}{\newton} - \SI{320}{\newton} \]
          \[ F_x = \SI{24}{\newton} \]
          \[ W_x = \SI{20}{\meter}\cdot\SI{24}{\newton} \]
          \[ \boxed{W_x = \SI{480}{\joule}} \]
        \end{center}

\end{enumerate}

\subsection{Parte 2}
\begin{enumerate}

\end{enumerate}

\end{document}