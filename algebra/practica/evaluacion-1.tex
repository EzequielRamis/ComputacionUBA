\documentclass[../practica.root.tex]{subfiles}

\begin{document}

\section{Primera Evaluación}
\begin{enumerate}
    \item El conjunto de todos los \( z \in \C \) tales que \(\overline{z}(\Re(z) + 1) = 2 + 6i\)
          \[ (a - bi)(a + 1) = 2 + 6i \]
          \[ a^2 + a - abi - bi = 2 + 6i \]
          \[
              \begin{cases}
                  a^2 + a = 2 \\
                  -abi - bi = 6i
              \end{cases}
          \] \[
              \begin{cases}
                  a^2 + a -2 = 0 \\
                  (-abi - bi = 6i)i
              \end{cases}
          \] \[
              \begin{cases}
                  a^2 + a -2 = 0 \\
                  ab + b = -6
              \end{cases}
          \] \[
              \begin{cases}
                  (a + 2)(a - 1) = 0 \\
                  b(a + 1) = -6
              \end{cases}
          \] \[
              a = \{ -2, 1 \}
          \]
          Si \(a = -2\)
          \[ b(-2 + 1) = -6 \]
          \[ -b = -6 \]
          \[ b = 6 \]
          Si \(a = 1\)
          \[ b(1 + 1) = -6 \]
          \[ 2b = -6 \]
          \[ b = -3 \]
          \[ \boxed{\{ -2 + 6i, 1 -3i \}} \]

    \item Sea \(k \in \R\) y \(P(x) = 3x^3 - x^2 + kx - 4 \in \R[x]\). Si \(P\) tiene una raiz imaginaria pura, encontrar las raices.


    \item Sea \(\mathbb{L}: \lambda(2, 1, -2) + (3,1,4)\), \(A = (-1,1,1)\), \(B = (0,2,3)\) y \(C = (3,1,0)\). El plano \(\Pi\) que pasa por \(C\) y es paralelo a \(\mathbb{L}\) y \(\overline{AB}\)
          \[ \overline{AB}: \lambda(A-B) + B \]
          \[ \overline{AB}: \lambda((-1,1,1)-(0,2,3)) + (0,2,3) \]
          \[ \overline{AB}: \lambda(1, 1, 2) + (0, 2, 3) \]
          Para que \(\Pi\) sea paralelo a \(\mathbb{L}\) y \(\overline{AB}\), la normal tiene que ser perpendicular a sus vectores directores
          \[ N = (1, 1, 2)\x(2, 1, -2) = (-4, 6, -1) \]
          \[ \Pi : -4x + 6y - z = d \]
          \[ -4\cdot3 + 6\cdot1 - 0 = d \]
          \[ -12 + 6 = -6 = d \]
          \[ -4x + 6y - z = -6 \iff 4x - 6y + z = 6  \]
          \[ \boxed{\Pi : 4x - 6y + z = 6} \]

    \item Sean \(A = \begin{pmatrix}
              1  & 1 & -1 \\
              -2 & 0 & 3  \\
              1  & 1 & -1
          \end{pmatrix}\) y \(B = \begin{pmatrix}
              0 & -1 & 0 \\
              2 & 1  & 0 \\
              2 & 0  & 1
          \end{pmatrix}\) encontrar \(X \in \R^{3\x3}\) tales que \(AX - B = A - BX\)
          \[ AX - B = A - BX \]
          \[ AX + BX - A - B = \0 \]
          \[ X(A + B) - (A + B) = \0 \]
          \[ (A + B)(X - I) = \0 \]
          \[
              A + B = \begin{pmatrix}
                  1 & 0 & -1 \\
                  0 & 1 & 3  \\
                  3 & 1 & 0
              \end{pmatrix}
          \] \[
              X - I = \begin{pmatrix}
                  x_{1,1}-1 & x_{1,2}   & x_{1,3}   \\
                  x_{2,1}   & x_{2,2}-1 & x_{2,3}   \\
                  x_{3,1}   & x_{3,2}   & x_{3,3}-1 \\
              \end{pmatrix}
          \] \[
              \begin{pmatrix}
                  1 & 0 & -1 \\
                  0 & 1 & 3  \\
                  3 & 1 & 0  \\
              \end{pmatrix}
              \x
              \begin{pmatrix}
                  x_{1,1}-1 & x_{1,2}   & x_{1,3}   \\
                  x_{2,1}   & x_{2,2}-1 & x_{2,3}   \\
                  x_{3,1}   & x_{3,2}   & x_{3,3}-1 \\
              \end{pmatrix}
          \] \[
              =
              \begin{pmatrix}
                  x_{1,1} - x_{3,1} - 1    & x_{1,2} - x_{3,2}      & x_{1,3} - x_{3,3} + 1    \\
                  x_{2,1} + 3x_{3,1}       & x_{2,2} + 3x_{3,2} - 1 & x_{2,3} + 3(x_{3,3} - 1) \\
                  3(x_{1,1} - 1) + x_{2,1} & 3x_{1,2} + x_{2,2} - 1 & 3x_{1,3} + x_{2,3}       \\
              \end{pmatrix}
              =
              \0
          \] \[
              \begin{array}{rl}
                  F_3-3F_1-F_2
              \end{array}
              \begin{pmatrix}
                  x_{1,1} - x_{3,1} - 1 & x_{1,2} - x_{3,2}      & x_{1,3} - x_{3,3} + 1    \\
                  x_{2,1} + 3x_{3,1}    & x_{2,2} + 3x_{3,2} - 1 & x_{2,3} + 3(x_{3,3} - 1) \\
                  0                     & 0                      & 0                        \\
              \end{pmatrix}
          \]
          Notese que cada columna es independiente del resto, cada una usa sus propias variables, y por lo tanto las podemos resolver independientemente
          \[
              \begin{cases}
                  x_{1,1} - x_{3,1} - 1 = 0 \\
                  x_{2,1} + 3x_{3,1} = 0    \\
              \end{cases}
              =
              \begin{cases}
                  x_{1,1} = 1 + x_{3,1} \\
                  x_{2,1} = -3x_{3,1}   \\
              \end{cases}
          \] \[
              \begin{cases}
                  x_{1,2} - x_{3,2} = 0      \\
                  x_{2,2} + 3x_{3,2} - 1 = 0 \\
              \end{cases}
              =
              \begin{cases}
                  x_{1,2} = x_{3,2}      \\
                  x_{2,2} = 1 - 3x_{3,2} \\
              \end{cases}
          \] \[
              \begin{cases}
                  x_{1,3} - x_{3,3} + 1 = 0    \\
                  x_{2,3} + 3(x_{3,3} - 1) = 0 \\
              \end{cases}
              =
              \begin{cases}
                  x_{1,3} = -1 + x_{3,3} \\
                  x_{2,3} = 3 - 3x_{3,3} \\
              \end{cases}
          \]
          \[
              \begin{pmatrix}
                  x_{1,1} & x_{1,2} & x_{1,3} \\
                  x_{2,1} & x_{2,2} & x_{2,3} \\
                  x_{3,1} & x_{3,2} & x_{3,3} \\
              \end{pmatrix}
              =
              \begin{pmatrix}
                  1 + x_{3,1} & x_{3,2}      & -1 + x_{3,3} \\
                  -3x_{3,1}   & 1 - 3x_{3,2} & 3 - 3x_{3,3} \\
                  x_{3,1}     & x_{3,2}      & x_{3,3}      \\
              \end{pmatrix}
          \] \[
              \boxed{
                  \left\{
                  \begin{pmatrix}
                      1 + r & s      & -1 + t \\
                      -3r   & 1 - 3s & 3 - 3t \\
                      r     & s      & t      \\
                  \end{pmatrix}
                  :r, s, t \in \R
                  \right\}
              }
          \]

    \item Sea \(A = \begin{pmatrix}
              1 & 2  & 0 \\
              0 & -3 & 2 \\
              0 & 4  & a \\
          \end{pmatrix}\) y \(B = \begin{pmatrix}
              1 & 2 & -1 \\
              0 & 1 & 1  \\
              0 & 3 & 2  \\
          \end{pmatrix}\). Encontrar el conjunto de todos los \( a \in \R \) tales que \( Ax + 3x = B^{-1}x \) tiene solución unica.
          \[
              B^{-1} = \begin{pmatrix}
                  1 & 7  & -3 \\
                  0 & -2 & 1  \\
                  0 & 3  & -1 \\
              \end{pmatrix}
          \] \[
              x(A + 3 = B^{-1})
          \] \[
              A + 3 - B^{-1} = \0
          \] \[
              \begin{pmatrix}
                  3 & -2 & 6   \\
                  3 & 2  & 4   \\
                  3 & 4  & a+4 \\
              \end{pmatrix}
          \]

    \item Sean \(\mathbb{L}_1 : \lambda_1(0,1,1)+(2,3,-1)\) y \(\mathbb{L}_2\) la recta que pasa por \(A = (4,0,-3)\) y \(B = (6,-1,-3)\). Si \(P : \mathbb{L}_1 \cap \mathbb{L}_2\) y \(\Pi : x + 2y - 2z = 4\), encontrar \(d(P,\Pi)\)
          \[ \mathbb{L}_2 : \lambda_2(A-B) + B \]
          \[ \lambda_2((4,0,-3)-(6,-1,-3))+(6,-1,-3) \]
          \[ \mathbb{L}_2 : \lambda_2(-2,1,0)+(6,-1,-3) \]
          \[ P : \mathbb{L}_1 \cap \mathbb{L}_2 \]
          \[ \lambda_1(0,1,1)+(2,3,-1) = \lambda_2(-2,1,0)+(6,-1,-3) \]
          \[ \lambda_1(0,1,1)-\lambda_2(-2,1,0) = (6,-1,-3)-(2,3,-1) \]
          \[ (0,\lambda_1,\lambda_1)+(2\lambda_2,-\lambda_2,0) = (4,-4,-2) \]
          \[ (2\lambda_2,\lambda_1-\lambda_2,\lambda_1) = (4,-4,-2) \]
          \[
              \begin{cases}
                  2\lambda_2 = 4             \\
                  \lambda_1 - \lambda_2 = -4 \\
                  \lambda_1 = -2
              \end{cases}
          \] \[
              \lambda_1 = -2, \lambda_2 = 2
          \]
          \[ P = -2(0,1,1)+(2,3,-1) \]
          \[ P = (2,1,-3) \]
          \[
              d(P,\Pi) = \frac{
                  |ax + by + cz - d|
              }{
                  \sqrt{a^2 + b^2 + c^2}
              } = \frac{
                  |2 + 2 + 6 - 4|
              }{
                  \sqrt{1 + 4 + 4}
              } = \frac{6}{\sqrt{9}} = \boxed{2}
          \]


    \item Sean \(A = (1,2,0), B = (0,1,1), C = (2,0,1)\). Sea \(\Pi\) el plano que pasa por \(A\), \(B\) y \(C\) y sea \(\mathbb{L} = \lambda(-k^2 + 1, k - 2, k - 1) + (k, 3, -1)\). El conjunto de los valores de \(k \in \R\) tal que \(\mathbb{L}\cap\Pi = \emptyset\)
          \[ \Pi: ax + by + cz = B\cdot N \]
          \[ N = (a,b,c) = (A-B)\x(C-B) = (1,1,-1)\x(2,-1,0) = (-1,-2,-3) \equiv (1,2,3) \]
          \[ B\cdot N = 5 \]
          \[ \Pi: x + 2y + 3z = 5 \]
          \[ \mathbb{L} = (-k^2\lambda + \lambda + k, k\lambda - 2\lambda + 3, k\lambda - \lambda - 1)  \]
          \[ -k^2\lambda + \lambda + k + 2(k\lambda - 2\lambda + 3) + 3(k\lambda - \lambda - 1) \neq 5 \]
          \[ -k^2\lambda + \lambda + k + 2k\lambda - 4\lambda + 6 + 3k\lambda - 3\lambda - 3 \neq 5 \]
          \[ -k^2\lambda + \lambda - 4\lambda - 3\lambda + k + 2k\lambda + 3k\lambda + 6 - 3 - 5 \neq 0 \]
          \[ -k^2\lambda - 6\lambda + k + 5k\lambda -2 \neq 0 \]
          (Una de las posibles respuestas del multiple-choice es \(k = 3\), asi que probamos con esa)
          \[ k = 3 \]
          \[ -9\lambda - 6\lambda + 3 + 15\lambda -2 \neq 0 \]
          \[ 1 \neq 0 \]
          \[ \boxed{S = \{3\}} \]


    \item Todos los valores de \(a, b \in \R\) para los cuales el conjunto solución del sistema \(\begin{cases}
              x_1 - x_2 + 2x_3 + x_4 = 0 \\
              x_2 + 2x_3 - 3x_4 = 0      \\
              2ax_2 + bx_3 - x_4 = 0     \\
          \end{cases}\) es \(S = \{\alpha(4,2,-1,0), \alpha \in \R\}\)
          \[
              \begin{cases}
                  4\alpha - 2\alpha - 2\alpha = 0 \\
                  2\alpha - 2\alpha = 0           \\
                  4a\alpha -b\alpha = 0           \\
              \end{cases}
          \] \[
              \begin{cases}
                  0 = 0  \\
                  0 = 0  \\
                  4a = b \\
              \end{cases}
          \] \[
              \begin{cases}
                  x_1 - x_2 + 2x_3 + x_4 = 0 \\
                  x_2 + 2x_3 - 3x_4 = 0      \\
                  2ax_2 + 4ax_3 - x_4 = 0    \\
              \end{cases}
              =
              \begin{pmatrix}
                  1 & -1 & 2  & 1  \\
                  0 & 1  & 2  & -3 \\
                  0 & 2a & 4a & -1 \\
              \end{pmatrix}
          \]

    \item El polinomio \(P \in \R[x]\) de grado minimo tal que las soluciones de \(z^3 + 4z\overline{z} = 0\) son raises de \(P\) y \(P(-1) = -15\)
          \[ z^3 + 4z\overline{z} = 0 \]
          \[ S = \{ -4, 0, 2-2i\sqrt{3}, 2+2i\sqrt{3} \} \]
          \[ (x+4)(x)(x-2+2i\sqrt{3})(x-2-2i\sqrt{3}) = x^4 + 64x \]
          \[ (-1)+64(-1) = -65 \]
          \[ -65\cdot\frac{5}{21} = -15 \]
          \[ \boxed{P(x) = \frac{5}{21}(x^4 + 64x)} \]


    \item Si \(A = \begin{pmatrix}
              1  & 1  & -1 \\
              2  & 1  & -2 \\
              -2 & -3 & 4
          \end{pmatrix}\). Calcular \( \det\left(\frac{1}{8}(A^6 - 2A^5)\right) \)
          \[ \det\left(\frac{1}{8}(A^6 - 2A^5)\right) \]
          \[ (\frac{1}{8})^3\det(A^6 - 2A^5) \]
          \[ \frac{1}{512}\det(A^5)\det(A - 2I) \]
          \[ \frac{1}{512}\det(A)^5\det(A - 2I) \]
          \[
              \det(A) = -2, \det(A - 2I) = 16
          \]
          \[ \frac{1}{512}\cdot(-32)\cdot 16 \]
          \[ \boxed{\det\left(\frac{1}{8}(A^6 - 2A^5)\right) = -1} \]

\end{enumerate}
\end{document}