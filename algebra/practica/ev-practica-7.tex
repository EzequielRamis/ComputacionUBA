\documentclass[../practica.root.tex]{subfiles}

\begin{document}

\section{Autoevaluación de la Práctica 7}

\begin{enumerate}
    \item[] \dots
    \item   \dots
    \item   \dots
    \item[] Sea \( A = \begin{pmatrix}
              0 & -1 & k  \\
              3 & 1  & -1 \\
              2 & 1  & -2
          \end{pmatrix} \) y \( v = (1,2,1) \) un autovector de esta. Hallar \( k \in \R \), \( C \in \R^{3\x3} \) que sea inversible y \( D \in \R^{3\x3} \) que sea diagonal tal que \( A = CDC^{-1} \).
    \item   El valor de \( k \in \R \) para el cual el vector \( v = (1,2,1) \) es un autovector de \( A \) es:
          \[
              \begin{pmatrix}
                  0 & -1 & k  \\
                  3 & 1  & -1 \\
                  2 & 1  & -2
              \end{pmatrix}
              \begin{pmatrix}
                  1 \\ 2 \\ 1
              \end{pmatrix}
              =
              \lambda
              \begin{pmatrix}
                  1 \\ 2 \\ 1
              \end{pmatrix}
          \] \[
              \begin{pmatrix}
                  k - 2 \\ 4 \\ 2
              \end{pmatrix}
              =
              \lambda
              \begin{pmatrix}
                  1 \\ 2 \\ 1
              \end{pmatrix}
          \]
          Si \( k = 4 \), entonces \( v \) es un autovector asociado al autovalor \( \lambda = 2 \)
          \[ \boxed{ k = 4 } \]
    \item Para el valor de \( k \) encontrado, hallar todos los autovalores
          \[
              A - \lambda I=
              \begin{pmatrix}
                  -\lambda & -1        & 4          \\
                  3        & 1-\lambda & -1         \\
                  2        & 1         & -2-\lambda
              \end{pmatrix}
          \]
          \begin{align*}
              \det(A - \lambda I)              & = 0                                          \\
              -\lambda^3 -\lambda^2 + 6\lambda & = 0                                          \\
              -\lambda(\lambda^2 + \lambda -6) & = 0 \implies \lambda_1 = 0                   \\
              \lambda^2 + \lambda -6           & = 0                                          \\
              (\lambda - 2)(\lambda + 3)       & = 0 \implies \lambda_2 = 2 \  \lambda_3 = -3
          \end{align*}
          \[ \boxed{\lambda_1 = 0 \ \lambda_2 = 2 \ \lambda_3 = -3} \]
    \item Encontrar una base \( \R^3 \) formada por autovectores de \( A \):
          \[
              \begin{pmatrix}
                  -\lambda & -1        & 4          \\
                  3        & 1-\lambda & -1         \\
                  2        & 1         & -2-\lambda
              \end{pmatrix}
              \begin{pmatrix}
                  x \\ y \\ z
              \end{pmatrix}
              =
              \0
          \] \[
              \begin{cases}
                  -\lambda x - y + 4z = 0       \\
                  (1 - \lambda) y + 3x - z = 0  \\
                  (-\lambda - 2) z + 2x + y = 0 \\
              \end{cases}
          \]
          \begin{minipage}[t]{0.5\textwidth}
              Si \( \lambda = 0 \)
              \[
                  \begin{cases}
                      -y + 4z = 0 \implies y = 4z    \\
                      y + 3x - z = 0 \implies x = -z \\
                      -2z + 2x + y = 0               \\
                  \end{cases}
              \] \[
                  (x, y, z) = (-z,4z,z) = z(-1,4,1)
              \]
          \end{minipage}
          \begin{minipage}[t]{0.5\textwidth}
              Si \( \lambda = 2 \)
              \[
                  \begin{cases}
                      -2x - y + 4z = 0 \\
                      3x - y -  z = 0  \\
                      2x + y - 4z = 0  \\
                  \end{cases}
                  \implies
                  \begin{array}{rl}
                      x & = z  \\
                      y & = 2z
                  \end{array}
              \] \[
                  (x, y, z) = (z,2z,z) = z(1,2,1)
              \]
          \end{minipage}
          \begin{minipage}[t]{0.5\textwidth}
              Si \( \lambda = -3 \)
              \[
                  \begin{cases}
                      3x -  y + 4z = 0 \\
                      3x + 4y -  z = 0 \\
                      2x +  y +  z = 0 \\
                  \end{cases}
                  \implies
                  \begin{array}{rl}
                      y & = -x \\
                      z & = -x
                  \end{array}
              \] \[
                  (x, y, z) = (x,-x,-x) = x(1,-1,-1)
              \]
          \end{minipage}
          \[ \boxed{B = \{ (-1,4,1),(1,2,1),(1,-1,-1) \}} \]
    \item Una matriz inversible \( C \in \R^{3\x3} \) y una diagonal \( D \in \R^{3\x3} \) tal que \( A = CDC^{-1} \) son:
          \begin{enumerate}
              \item \( C = \begin{pmatrix}
                        1 & -1 & -1 \\
                        2 & 1  & 4  \\
                        1 & 1  & 1
                    \end{pmatrix}
                    D = \begin{pmatrix}
                        0 & 0 & 0  \\
                        0 & 2 & 0  \\
                        0 & 0 & -3
                    \end{pmatrix} \)
              \item \( C = \begin{pmatrix}
                        1 & -1 & -1 \\
                        2 & 1  & 4  \\
                        1 & 1  & 1
                    \end{pmatrix}
                    D = \begin{pmatrix}
                        -3 & 0 & 0 \\
                        0  & 0 & 0 \\
                        0  & 0 & 2
                    \end{pmatrix} \)
              \item \( C = \begin{pmatrix}
                        1  & 2 & 1 \\
                        -1 & 1 & 1 \\
                        -1 & 4 & 1
                    \end{pmatrix}
                    D = \begin{pmatrix}
                        0 & 0 & 0  \\
                        0 & 2 & 0  \\
                        0 & 0 & -3
                    \end{pmatrix} \)
              \item \boxed{
                        C = \begin{pmatrix}
                            -1 & -1 & 1 \\
                            1  & 4  & 2 \\
                            1  & 1  & 1
                        \end{pmatrix}
                        D = \begin{pmatrix}
                            -3 & 0 & 0 \\
                            0  & 0 & 0 \\
                            0  & 0 & 2
                        \end{pmatrix}
                    }
          \end{enumerate}
          Respuesta: \href{https://www.wolframalpha.com/input/?i=%7B%7B-1%2C-1%2C1%7D%2C%7B1%2C4%2C2%7D%2C%7B1%2C1%2C1%7D%7D%7B%7B-3%2C0%2C0%7D%2C%7B0%2C0%2C0%7D%2C%7B0%2C0%2C2%7D%7D(%7B%7B-1%2C-1%2C1%7D%2C%7B1%2C4%2C2%7D%2C%7B1%2C1%2C1%7D%7D%5E-1)}{[WolframAlpha]}
    \item
\end{enumerate}

\end{document}