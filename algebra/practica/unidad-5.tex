\documentclass[../practica.root.tex]{subfiles}

\begin{document}

\section{Transformaciones Lineales}

\begin{enumerate}
    \item Determinar si la función $f$ es t. l.
          \begin{enumerate}
              \item $ f: \R^3 \to \R^2, f(x_1, x_2, x_3) = (x_1 - x_2, 2x_1) $
                    \[ v, u \in \R^3 \]
                    \[ f(v + u) = f(v) + f(u) \]
                    \[ f(v_1 + u_1,v_2 + u_2,v_3 + u_3) = f(v_1,v_2,v_3) + f(u_1,u_2,u_3) \]
                    \[ ((v_1 + u_1) - (v_2 + u_2), 2(v_1 + u_1)) = (v_1 - v_2, 2v_1) + (u_1 - u_2, 2u_1) \]
                    \[ (v_1 - v_2 + u_1 - u_2, 2v_1 + 2u_1) = (v_1 - v_2 + u_1 - u_2, 2v_1 + 2u_1) \]
                    $f$ verifica $(v + u) = f(v) + f(u)$
                    \[ v \in \R^3, a \in \R \]
                    \[ af(v) = f(av) \]
                    \[ a(v_1 - v_2, 2v_1) = (av_1 - av_2, 2av_1) \]
                    \[ \boxed{f \text{ es t. l.}} \]
              \item $ f: \R^2 \to \R^3, f(x_1,x_2) = (x_1\cdot x_2,0,0) $
                    \[ v, u \in \R^2 \]
                    \[ f(v + u) = f(v) + f(u) \]
                    \[ f(v_1 + u_1, v_2 + u_2) = f(v_1, v_2) + f(u_1, u_2) \]
                    \[ ((v_1 + u_1)\cdot(v_2 + u_2),0,0) = (v_1v_2,0,0) + (u_1u_2,0,0) \]
                    \[ (v_1v_2 + v_1u_2 + u_1v_2 + u_1u_2,0,0) = (v_1v_2 + u_1u_2,0,0) \]
                    \[ v_1v_2 + v_1u_2 + u_1v_2 + u_1u_2 \neq v_1v_2 + u_1u_2 \]
                    \[ \boxed{$f$ \text{ no es t. l. No verifica } $f(v + u) = f(v) + f(u)$} \]
              \item $ f: \R^2 \to \R^{3\x2}, f(x_1,x_2) =
                    \begin{pmatrix}
                        x_1  & x_1 - x_2 \\
                        0    & 0         \\
                        -x_1 & 0
                    \end{pmatrix}
                    $
                    \[ v, u \in \R^2 \]
                    \[ f(v + u) = f(v) + f(u) \]
                    \[ f(v_1 + u_1, v_2 + u_2) = f(v_1, v_2) + f(u_1, u_2) \]
                    \[
                        \begin{pmatrix}
                            (v_1 + u_1)  & (v_1 + u_1) - (v_2 + u_2) \\
                            0            & 0                         \\
                            -(v_1 + u_1) & 0
                        \end{pmatrix}
                        =
                        \begin{pmatrix}
                            v_1  & v_1 - v_2 \\
                            0    & 0         \\
                            -v_1 & 0
                        \end{pmatrix}
                        +
                        \begin{pmatrix}
                            u_1  & u_1 - u_2 \\
                            0    & 0         \\
                            -u_1 & 0
                        \end{pmatrix}
                    \] \[
                        \begin{pmatrix}
                            v_1 + u_1  & v_1 + u_1 - v_2 - u_2 \\
                            0          & 0                     \\
                            -v_1 - u_1 & 0
                        \end{pmatrix}
                        =
                        \begin{pmatrix}
                            v_1 + u_1 & v_1 + u_1 - v_2 - u_2 \\
                            0         & 0                     \\
                            -v_1-u_1  & 0
                        \end{pmatrix}
                    \]
                    \[ v \in \R^{3\x2}, a \in \R \]
                    \[ af(v) = f(av) \]
                    \[
                        a
                        \begin{pmatrix}
                            v_1  & v_1 - v_2 \\
                            0    & 0         \\
                            -v_1 & 0
                        \end{pmatrix}
                        =
                        \begin{pmatrix}
                            av_1  & av_1 - av_2 \\
                            0     & 0           \\
                            -av_1 & 0
                        \end{pmatrix}
                    \]
                    \[ \boxed{f \text{ es t. l.}} \]
              \item $ f: \R^{2\x3} \to \R^{3\x2}, f(A) = A^t $
                    \[ v, u \in \R^2 \]
                    \[ f(v + u) = f(v) + f(u) \]
                    \[
                        f
                        \begin{pmatrix}
                            v_{1,1} + u_{1,1} & v_{1,2} + u_{1,2} & v_{1,3} + u_{1,3} \\
                            v_{2,1} + u_{2,1} & v_{2,2} + u_{2,2} & v_{2,3} + u_{2,3} \\
                        \end{pmatrix}
                        =
                        f
                        \begin{pmatrix}
                            v_{1,1} & v_{1,2} & v_{1,3} \\
                            v_{2,1} & v_{2,2} & v_{2,3} \\
                        \end{pmatrix}
                        +
                        f
                        \begin{pmatrix}
                            u_{1,1} & u_{1,2} & u_{1,3} \\
                            u_{1,1} & u_{1,1} & u_{1,1} \\
                        \end{pmatrix}
                    \] \[
                        \begin{pmatrix}
                            (vu)_{1,1} & (vu)_{2,1} \\
                            (vu)_{1,2} & (vu)_{2,2} \\
                            (vu)_{1,3} & (vu)_{2,3} \\
                        \end{pmatrix}
                        =
                        \begin{pmatrix}
                            v_{1,1} & v_{2,1} \\
                            v_{1,2} & v_{2,2} \\
                            v_{1,3} & v_{2,3} \\
                        \end{pmatrix}
                        +
                        \begin{pmatrix}
                            u_{1,1} & u_{2,1} \\
                            u_{1,2} & u_{2,2} \\
                            u_{1,3} & u_{2,3} \\
                        \end{pmatrix}
                    \]
                    $f$ verifica $(v + u) = f(v) + f(u)$
                    \[ v \in \R^{2\x3}, a \in \R \]
                    \[ af(v) = f(av) \]
                    \[
                        a\cdot f
                        \begin{pmatrix}
                            v_{1,1} & v_{1,2} & v_{1,3} \\
                            v_{2,1} & v_{2,2} & v_{2,3} \\
                        \end{pmatrix}
                        =
                        \begin{pmatrix}
                            av_{1,1} & av_{1,2} & av_{1,3} \\
                            av_{2,1} & av_{2,2} & av_{2,3} \\
                        \end{pmatrix}
                    \] \[
                        a\cdot
                        \begin{pmatrix}
                            v_{1,1} & v_{2,1} \\
                            v_{1,2} & v_{2,2} \\
                            v_{1,3} & v_{2,3} \\
                        \end{pmatrix}
                        =
                        \begin{pmatrix}
                            av_{1,1} & av_{2,1} \\
                            av_{1,2} & av_{2,2} \\
                            av_{1,3} & av_{2,3} \\
                        \end{pmatrix}
                    \]
                    \[ \boxed{f \text{ es t. l.}} \]
          \end{enumerate}
    \item \dots
    \item \dots
    \item Hallar una base y la dimensión de $\Nu f$ y de $\Img f$
          \begin{enumerate}
              \item $ f : \R^3 \to \R^3, f(x_1,x_2,x_3) = (x_1 + x_2 + x_3, x_1 - x_2, 2x_2 + x_3) $
                    Calcular $\Nu f$:
                    \[ f(x) = \0 \]
                    \[
                        \begin{cases}
                            x_1 + x_2 + x_3 = 0 \\
                            x_1 - x_2 = 0       \\
                            2x_2 + x_3 = 0
                        \end{cases}
                    \] \[
                        \begin{pmatrix}
                            1 & 1  & 1 \\
                            1 & -1 & 0 \\
                            0 & 2  & 1
                        \end{pmatrix}
                        \begin{array}{rl}
                            F_2 - F_1 + F_3 & \to F_2             \\
                            \frac{1}{2}F_3  & \to F_3             \\
                            F_2             & \leftrightarrow F_3
                        \end{array}
                        \begin{pmatrix}
                            1 & 1 & 1   \\
                            0 & 1 & 0.5 \\
                            0 & 0 & 0
                        \end{pmatrix}
                    \] \[
                        \boxed{\dim\Nu{f} = 3 - 2 = 1}
                    \] \[
                        \begin{cases}
                            x_1 + x_2 + x_3 = 0 \implies x_1 = -\frac{1}{2}x_3 \\
                            x_2 + \frac{1}{2}x_3 = 0 \implies x_2 = -\frac{1}{2}x_3
                        \end{cases}
                    \] \[
                        (x_1, x_2, x_3) = \left( -\frac{1}{2}x_3, -\frac{1}{2}x_3, x_3 \right) = x_3\left(-\frac{1}{2}, -\frac{1}{2}, 1\right)
                    \] \[
                        \boxed{B_{\Nu f} = \left\{\left(-\frac{1}{2}, -\frac{1}{2}, 1\right)\right\}}
                    \]
                    Calcular $\Img f$:
                    \[ \Img f = \langle f(\hat{i}), f(\hat{j}), f(\hat{k}) \rangle = \langle (1,1,0),(1,-1,2),(1,0,1) \rangle\]
                    Verificar que los vectores sean l. i.:
                    \[
                        \alpha(1,1,0) + \beta(1,-1,2) + \gamma(1,0,1) = (\alpha + \beta + \gamma, \alpha - \beta, 2\beta + \gamma) = 0
                    \] \[
                        \begin{pmatrix}
                            1 & 1  & 1 \\
                            1 & -1 & 0 \\
                            0 & 2  & 1
                        \end{pmatrix}
                    \]
                    (Este sistema es el mismo que el del nucleo)
                    \[
                        \begin{pmatrix}
                            1 & 1 & 1   \\
                            0 & 1 & 0.5 \\
                            0 & 0 & 0
                        \end{pmatrix}
                    \]
                    El sistema es l. d. por lo que hay que hay que extraer los 2 primeros vectores para crear una base:
                    \[ \boxed{B_{\Img f} = \{ (1,0,0),(1,1,0) \}} \]
                    \[ \boxed{\dim B_{\Img f} = 2} \]

              \item $ f : \R^4 \to \R^4, f(x_1,x_2,x_3,x_4) = (x_1 + x_3, 0, x_2 + 2x_3, -x_1 + x_2 + x_3) $
                    Calcular $\Nu f$:
                    \[ f(x) = \0 \]
                    \[
                        \begin{cases}
                            x_1 + x_3 = 0        \\
                            0 = 0                \\
                            x_2 + 2x_3 = 0       \\
                            -x_1 + x_2 + x_3 = 0 \\
                        \end{cases}
                    \] \[
                        \begin{pmatrix}
                            1  & 0 & 1 & 0 \\
                            0  & 0 & 0 & 0 \\
                            0  & 1 & 2 & 0 \\
                            -1 & 1 & 1 & 0
                        \end{pmatrix}
                        \begin{array}{rl}
                            F_4 + F_1 - F_3 & \to F_4             \\
                            F_2             & \leftrightarrow F_3 \\
                        \end{array}
                        \begin{pmatrix}
                            1 & 0 & 1 & 0 \\
                            0 & 1 & 2 & 0 \\
                            0 & 0 & 0 & 0 \\
                            0 & 0 & 0 & 0
                        \end{pmatrix}
                    \] \[
                        \boxed{\dim \Nu f = 4 - 2 = 2}
                    \] \[
                        \begin{cases}
                            x_1 + x_3 = 0 \implies x_1 = -x_3 \\
                            x_2 + 2x_3 = 0 \implies x_2 = -2x_3
                        \end{cases}
                    \] \[
                        (x_1, x_2, x_3, x_4) = (-x_3, -2x_3, x_3, x_4) = x_3(-1,-2,1,0) + x_4(0,0,0,1)
                    \] \[
                        \boxed{B_{\Nu f} = \{ (-1,-2,1,0),(0,0,0,1) \}}
                    \]
                    Calcular $ \Img f $:
                    \[
                        \begin{array}{rl}
                            \Img f & = \langle f(1,0,0,0),f(0,1,0,0),f(0,0,1,0),f(0,0,0,1) \rangle \\
                                   & = \langle (1,0,0,-1),(0,0,1,1),(1,0,2,1),(0,0,0,0) \rangle
                        \end{array}
                    \]
                    Verificar que los vectores sean l. i.:
                    \[
                        \begin{pmatrix}
                            1  & 0 & 1 & 0 \\
                            0  & 0 & 0 & 0 \\
                            0  & 1 & 2 & 0 \\
                            -1 & 1 & 1 & 0 \\
                        \end{pmatrix}
                    \]
                    (Este sistema es el mismo que el del nucleo)
                    \[
                        \begin{pmatrix}
                            1 & 0 & 1 & 0 \\
                            0 & 1 & 2 & 0 \\
                            0 & 0 & 0 & 0 \\
                            0 & 0 & 0 & 0
                        \end{pmatrix}
                    \]
                    El sistema es l. d. por lo que hay que hay que extraer los 2 primeros vectores para crear una base:
                    \[ \boxed{B_{\Img f} = \{ (1,0,0,0),(0,1,0,0) \}} \]
                    \[ \boxed{\dim B_{\Img f} = 2} \]

              \item $ f : \R^3 \to \R^{2\x2}, f(x_1, x_2, x_3) =
                    \begin{pmatrix}
                        x_2 - x_3 & x_1 + x_3 \\
                        x_1 + x_2 & x_2 - x_3
                    \end{pmatrix} $
                    Calcular $\Nu f$:
                    \[ f(x) = \0 \]
                    \[
                        \begin{cases}
                            x_2 - x_3 = 0 \\
                            x_1 + x_3 = 0 \\
                            x_1 + x_2 = 0 \\
                            x_2 - x_3 = 0 \\
                        \end{cases}
                    \] \[
                        \begin{pmatrix}
                            0 & 1 & -1 \\
                            1 & 0 & 1  \\
                            1 & 1 & 0  \\
                            0 & 1 & -1 \\
                        \end{pmatrix}
                        \begin{array}{rl}
                            F_3 - F_2 - F_1 & \to F_3             \\
                            F_4 - F_1       & \to F_4             \\
                            F_1             & \leftrightarrow F_2
                        \end{array}
                        \begin{pmatrix}
                            1 & 0 & 1  \\
                            0 & 1 & -1 \\
                            0 & 0 & 0  \\
                            0 & 0 & 0  \\
                        \end{pmatrix}
                    \] \[
                        \boxed{\dim\Nu f = 3 - 2 = 1}
                    \] \[
                        \begin{cases}
                            x_1 + x_3 = 0 \implies x_1 = -x_3 \\
                            x_2 - x_3 = 0 \implies x_2 = x_3
                        \end{cases}
                    \] \[
                        (x_1, x_2, x_3) = (-x_3, x_3, x_3) = x_3(-1,1,1)
                    \] \[
                        \boxed{B_{\Nu f} = \{ (-1,1,1) \}}
                    \]
                    Calcular $ \Img f $:
                    \[
                        \begin{array}{rl}
                            \Img f & = \langle f(1,0,0),f(0,1,0),f(0,0,1) \rangle \\
                                   & =
                            \left\langle
                            \begin{pmatrix} 0 & 1 \\ 1 & 0 \end{pmatrix},
                            \begin{pmatrix} 1 & 0 \\ 1 & 1 \end{pmatrix},
                            \begin{pmatrix} -1 & 1 \\ 0 & -1 \end{pmatrix}
                            \right\rangle
                        \end{array}
                    \]
                    Los vectores no son l. i., por lo que tomamos los 2 primeros
                    \[
                        \boxed{
                            B_{\Img f} =
                            \left\{
                            \begin{pmatrix} 0 & 1 \\ 1 & 0 \end{pmatrix},
                            \begin{pmatrix} 1 & 0 \\ 1 & 1 \end{pmatrix}
                            \right\}
                        }
                    \] \[
                        \boxed{\dim B_{\Img f} = 2}
                    \]

          \end{enumerate}
    \item \dots
    \item Calcular $\dim\Nu f$ y $\dim\Img f$
          \begin{enumerate}
              \item $f: \R^3 \to \R^5$ monomorfismo \\
                    Por ser monomorfismo:
                    \[ \boxed{\dim\Nu f = 0} \]
                    \[ \dim f = \dim\Nu f + \dim\Img f \]
                    \[ 3 = 0 + \dim\Img f \]
                    \[ \boxed{\dim\Img f = 3} \]

              \item $f: \R^4 \to \R^3$ epimorfismo \\
                    Por ser epimorfismo:
                    \[ \Img f = \R^3 \implies \boxed{\Img f = 3}  \]
                    \[ \dim f = \dim\Nu f + \dim\Img f \]
                    \[ 4 = \dim\Nu f + 3 \]
                    \[ \boxed{\dim\Nu = 1} \]

              \item $f: \R \to \R \  f(x) = 2x $ \\
                    La función $ f(x) = 2x $ es isomorfica. \\ % anda a saber como probar algo asi
                    Por ser monomorfica:
                    \[ \boxed{\dim\Nu f = 0} \]
                    \[ \dim f = \dim\Nu f + \dim\Img f \]
                    \[ 4 = 0 + \dim\Img f \]
                    \[ \boxed{\dim\Img f = 4 } \]
          \end{enumerate}
    \item \dots
    \item \dots
    \item \dots
    \item Hallar $h = g \circ f$, $t = f \circ g$ y determinar el núcleo y la imagen de los 4
          \begin{enumerate}
              \item $ f : \R^3 \to \R^2 $, $ f(x_1,x_2,x_3) = (x_1,x_1+x_2-x_3) $, $ g : \R^2 \to \R^3 $, $ g(x_1, x_2) = (x_1 - x_2, x_1, x_2) $ \\ \\
                    \begin{minipage}[t]{0.5\textwidth}
                        Calcular nucleo de $f$:
                        \[ f(x) = \0 \]
                        \[
                            \begin{cases}
                                x_1 = 0 \\
                                x_1 + x_2 - x_3 = 0 \implies x_2 = x_3
                            \end{cases}
                        \] \[
                            (x_1, x_2, x_3) = (0, x_3, x_3) = x_3(0,1,1)
                        \] \[
                            \boxed{\Nu f = \langle (0,1,1) \rangle}
                        \]
                    \end{minipage}
                    \begin{minipage}[t]{0.5\textwidth}
                        Calcular imagen de $f$:
                        \[
                            \Img f = \langle f(\hat{i}),f(\hat{j}),f(\hat{k}) \rangle
                        \] \[
                            \Img f = \langle (1,1),(0,1),(0,-1) \rangle
                        \] \[
                            \boxed{\Img f = \langle (1,1),(0,1) \rangle}
                        \]
                    \end{minipage}

                    \begin{minipage}[t]{0.5\textwidth}
                        Calcular nucleo de $g$:
                        \[
                            g(x) = 0
                        \] \[
                            \begin{cases}
                                x_1 - x_2 = 0 \implies x_1 = x_2 \\
                                x_1 = 0                          \\
                                x_2 = 0
                            \end{cases}
                        \] \[
                            (x_1, x_2) = (x_2, x_2) = x_2(1,1)
                        \] \[
                            \boxed{\Nu g = \langle (1,1) \rangle}
                        \]
                    \end{minipage}
                    \begin{minipage}[t]{0.5\textwidth}
                        Calcular imagen de $g$:
                        \[ \Img g = \langle g(\hat{i}),g(\hat{j}) \rangle \]
                        \[ \boxed{\Img g = \langle (1,1,0),(-1,0,1) \rangle} \]
                    \end{minipage}
                    \[ h = g(f(x)) = g(x_1, x_1 + x_2 - x_3) = (-x_2 + x_3, x_1, x_1 + x_2 - x_3) \]
                    \begin{minipage}[t]{0.5\textwidth}
                        Calcular nucleo de $h$:
                        \[ h(x) = \0 \]
                        \[
                            \begin{cases}
                                -x_2 + x_3 = 0 \implies x_3 = x_2 \\
                                x_1 = 0                           \\
                                x_1 + x_2 - x_3 = 0
                            \end{cases}
                        \] \[
                            (x_1,x_2,x_3) = (0,x_2,x_2) = x_2(0,1,1)
                        \] \[
                            \boxed{\Nu h = \langle (0,1,1) \rangle \supset \Nu f}
                        \]
                    \end{minipage}
                    \begin{minipage}[t]{0.5\textwidth}
                        Calcular imagen de $h$:
                        \[ \Img h = \langle h(\hat{i}),h(\hat{j}),h(\hat{k}) \rangle \]
                        \[ \Img h = \langle (0,1,1),(-1,0,1),(1,0,-1) \rangle \]
                        \[ \boxed{\Img h = \langle (0,1,1),(1,0,-1) \rangle \subset \Img g} \]
                    \end{minipage}
                    \[ t = f(g(x)) = f(x_1 - x_2, x_1, x_2) = (x_1 - x_2, 2x_1 - 2x_2) \]
                    \begin{minipage}[t]{0.5\textwidth}
                        Calcular nucleo de $t$:
                        \[ t(x) = \0 \]
                        \[
                            \begin{cases}
                                x_1 - x_2 = 0 \implies x_1 = x_2 \\
                                2x_1 - 2x_2 = 0
                            \end{cases}
                        \] \[
                            (x_1, x_2) = (x_2, x_2) = x_2(1,1)
                        \] \[
                            \boxed{\Nu t = \langle (1,1) \rangle \supset \Nu g}
                        \]
                    \end{minipage}
                    \begin{minipage}[t]{0.5\textwidth}
                        Calcular imagen de $t$:
                        \[ \Img t = \langle t(\hat{i}),t(\hat{j}) \rangle \]
                        \[ \Img t = \langle (1,2),(-1,-2) \rangle \]
                        \[ \boxed{\Img t = \langle (1,2) \rangle \subset \Img f } \]
                    \end{minipage}
              \item $ f : \R^3 \to \R^3 $ la transformación lineal tal que: \\
                    $ f(0,0,1)=(0,-1,1) $, $ f(0,1,1)=(1,0,1) $ y $ f(1,1,1)=(1,1,0) $ \\
                    $ g : \R^3 \to \R^3 $, $g(x_1, x_2, x_3) = (2x_1 + x_3, x_2 - x_3, 2x_1 + x_2)$ \\
                    Calcular imagen de $f$:
                    \[
                        \begin{cases}
                            f(0,0,1) = (0, -1, 1) \\
                            f(0,1,1) = (1,  0, 1) \\
                            f(1,1,1) = (1,  1, 0)
                        \end{cases}
                        \implies
                        \begin{cases}
                            f(\hat{i}) = (0, -1, 1)                      \\
                            f(\hat{j}) = f(0,1,1)-f(\hat{i}) = (1, 1, 0) \\
                            f(\hat{k}) = f(1,1,1)-f(0,1,1) = (0,1,-1)
                        \end{cases}
                    \] \[
                        \Img f = \langle (0,-1,1),(1,1,0),(0,1,-1) \rangle
                    \] \[
                        \boxed{\Img f = \langle (0,-1,1),(1,1,0) \rangle}
                    \]
                    Calcular nucleo de $f$:
                    \[ f(x) = \0 \]
                    \[ x_1(0,-1,1) + x_2(1,1,0) + x_3(0,1,-1) = (x_2,-x_1+x_2+x_3,x_1-x_3) = \0 \]
                    \[
                        \begin{cases}
                            x_2 = 0              \\
                            -x_1 + x_2 + x_3 = 0 \\
                            x_1 - x_3 = 0 \implies x_1 = x_3
                        \end{cases}
                    \]
                    \[ (x_1,x_2,x_3) = (x_3,0,x_3) = x_3(1,0,1) \]
                    \[ \boxed{\Nu f = \langle (1,0,1) \rangle} \]
                    Calcular imagen de $g$:
                    \[ \Img g = \langle g(\hat{i}),g(\hat{j}),g(\hat{k}) \rangle \]
                    \[ \Img g = \langle (2,0,2),(0,1,1),(1,-1,0) \rangle \]
                    \[ \boxed{\Img g = \langle (0,1,1),(1,-1,0) \rangle} \]
                    Calcular nucleo de $g$:
                    \[
                        g(x) = \0
                    \] \[
                        \begin{cases}
                            2x_1 + x_3 = 0 \implies x_1 = -\frac{1}{2}x_3 \\
                            x_2 - x_3 = 0 \implies x_2 = x_3              \\
                            2x_1 + x_2 = 0
                        \end{cases}
                    \] \[
                        (x_1, x_2, x_3) = \left(-\frac{1}{2}x_3, x_3, x_3\right) = x_3\left(-\frac{1}{2},1,1\right)
                    \] \[
                        \boxed{\Img f = \left\langle \left(-\frac{1}{2},1,1\right) \right\rangle }
                    \]
                    \begin{align*}
                        h & = g(f(x))                                                        \\
                          & = g( f_1,f_2,f_3 ) \text{ (Estos vectores son la base de $f$)}   \\
                          & = (2f_1 + f_2, f_2 - f_3,2f_1 + f_2)                             \\
                          & = (2(0,-1,1) + (1,1,0), (1,1,0) - (0,1,-1), 2(0,-1,1) + (1,1,0)) \\
                          & = ( (1,-1,2), (1,0,1), (1,-1,2) )                                \\
                          & = h_1(1,-1,2) + h_2(1,0,1) + h_3(1,-1,2)                         \\
                          & = (h_1,-h_1,2h_1) + (h_2,0,h_2) + (h_3,-h_3,2h_3)                \\
                          & = (h_1 + h_2 + h_3, -h_1 -h_3, 2h_1 + h_2 + 2h_3)                \\
                    \end{align*}
                    \begin{minipage}[t]{0.5\textwidth}
                        Calcular la imagen de $h$:
                        \[ \Img h = \langle h(\hat{i}),h(\hat{j}),h(\hat{k}) \rangle \]
                        \[ \Img h = \langle (1,-1,2),(1,0,1),(1,-1,2) \rangle \]
                        \[ \boxed{\Img h = \langle (1,-1,2),(1,0,1) \rangle} \]
                    \end{minipage}
                    \begin{minipage}[t]{0.5\textwidth}
                        Calcular el nucleo de $h$:
                        \[ h(x) = \0 \]
                        \[
                            \begin{cases}
                                x_1 + x_2 + x_3 = 0               \\
                                -x_1 -x_3 = 0 \implies x_1 = -x_3 \\
                                2x_1 + x_2 + 2x_3 = 0 \implies x_2 = 0
                            \end{cases}
                        \] \[
                            (x_1, x_2, x_3) = (-x_3,0,x_3) = x_3(-1,0,1)
                        \] \[
                            \boxed{\Nu h = \langle (-1,0,1) \rangle}
                        \]
                    \end{minipage}
                    \begin{align*}
                        t & = f(g(x))                                                               \\
                          & = f(2x_1 + x_3, x_2 - x_3, 2x_1 + x_2)                                  \\
                          & = (2x_1 + x_3)(0,-1,1) + (x_2 - x_3)(1,1,0) + (2x_1 + x_2)(0,1,-1)      \\
                          & = (0,-2x_1-x_3,2x_1+x_3) + (x_2-x_3,x_2-x_3,0) + (0,2x_1+x_2,-2x_1-x_2) \\
                          & = (x_2 -x_3, -2x_1 -x_3 + x_2 -x_3 + 2x_1 + x_2, 2x_1 + x_3 -2x_1 -x_2) \\
                          & = (x_2 - x_3, 2x_2 -2x_3, -x_2 + x_3)
                    \end{align*}
                    \begin{minipage}[t]{0.5\textwidth}
                        Calcular imagen de $t$:
                        \[ \Img t = \langle t(\hat{i}),t(\hat{j}),t(\hat{k}) \rangle \]
                        \[ \Img t = \langle (0,0,0),(1,2,-1),(-1,-2,1) \rangle \]
                        \[ \boxed{\Img t = \langle (1,2,-1) \rangle} \]
                    \end{minipage}
                    \begin{minipage}[t]{0.5\textwidth}
                        Calcular nucleo de $t$:
                        \[ t(x) = \0 \]
                        \[
                            \begin{cases}
                                x_2 - x_3 = 0 \implies x_2 = x_3 \\
                                2x_2 - 2x_3 = 0                  \\
                                -x_2 + x_3 = 0
                            \end{cases}
                        \]
                        \begin{align*}
                            (x_1, x_2, x_3) & = (x_1, x_3, x_3)         \\
                                            & = x_1(1,0,0) + x_3(0,1,1)
                        \end{align*}
                        \[ \boxed{\Nu t = \langle (1,0,0),(0,1,1) \rangle} \]
                    \end{minipage}
          \end{enumerate}
    \item Hallar la función inversa del isomorfismo $f$
          \begin{enumerate}
              \item $ f : \R^3 \to \R^3 $ $ f(1,1,-1) = (1,-1,1) $,$ f(2,0,1) = (1,1,0) $,$ f(0,1,0) = (0,0,1) $
                    \[
                        \boxed{
                            f^{-1} =
                            \begin{cases}
                                f^{-1}(1,-1,1) = (1,1,-1) \\
                                f^{-1}(1,1,0) = (2,0,1)   \\
                                f^{-1}(0,0,1) = (0,1,0)   \\
                            \end{cases}
                        }
                    \]
              \item $ f : \R^2 \to \R^2 $ $ f(x_1,x_2) = (x_1,x_1-x_2) $
                    \[ f^{-1}(f(x_1,x_2)) = f^{-1}(x_1, x_1 - x_2) = (x_1, x_2) \]
                    \[ \boxed{f^{-1}(x_1, x_2) = (x_1, -x_2 + x_1) } \]
              \item $ f : \R^{2\x3} \to \R^{3\x2} $ $ f(A) = A^t $
          \end{enumerate}
    \item \dots
    \item \dots
    \item \dots
    \item Definir un proyector $p$ tal que:
          \begin{enumerate}
              \item $ p : \R^2 \to \R^2 $, $ \Nu p = \langle (-1,2) \rangle $ e $ \Img p \langle (-1,1) \rangle $
                    \[
                        \boxed{
                            \begin{cases}
                                p(-1,2) = (0,0) \\
                                p(-1,1) = (-1,1)
                            \end{cases}
                        }
                    \]
              \item $ p : \R^3 \to \R^3 $, $ \Nu p = \langle (1,1,-2) \rangle $ ¿Es único?
                    \[
                        \boxed{
                            \begin{cases}
                                p(1,1,-2) = (0,0,0)      \\
                                p(\Img p_1) = (\Img p_1) \\
                                p(\Img p_2) = (\Img p_2)
                            \end{cases}
                        }
                    \]
                    Existen infinitas proyecciones ya que los vectores generadores de la imagen pueden tener cualquier valor (Siempre y cuando no sean multiplos del nucleo)
              \item $ p : \R^4 \to \R^4 $, $ \Nu p = \langle (1,1,1,1),(-1,0,1,1),(1,2,3,3) \rangle $, $ \Img p = \langle (1,2,0,1),(-1,1,4,2) \rangle $ \\
                    El nucleo no es l. i.:
                    \[ 2(1,1,1,1) + (-1,0,1,1) = (1,2,3,3) \]
                    Tomamos los 2 primeros
                    \[ \Nu = \langle (1,1,1,1),(-1,0,1,1) \rangle \]
                    \[
                        \boxed{
                            \begin{cases}
                                p(1,1,1,1) = (0,0,0,0)  \\
                                p(-1,0,1,1) = (0,0,0,0) \\
                                p(1,2,0,1) = (1,2,0,1)  \\
                                p(-1,1,4,2) = (-1,1,4,2)
                            \end{cases}
                        }
                    \]
          \end{enumerate}
    \item \dots
    \item Sea $ f : \R^3 \to \R^3 $ tal que $ M(f) = \begin{pmatrix}
              1  & 2 & -1 \\
              3  & 1 & 2  \\
              -2 & 0 & -2
          \end{pmatrix} $
          \begin{enumerate}
              \item Calcular $ f(1,2,1) $, $ f(5,7,2) $ y $ f(0,0,1) $
                    \[
                        M(f)\cdot
                        \begin{pmatrix}
                            1 \\ 2 \\ 1
                        \end{pmatrix}
                        =
                        \boxed{
                            \begin{pmatrix}
                                4 \\ 7 \\ -4
                            \end{pmatrix}
                        }
                    \] \[
                        M(f)\cdot
                        \begin{pmatrix}
                            5 \\ 7 \\ 2
                        \end{pmatrix}
                        =
                        \boxed{
                            \begin{pmatrix}
                                17 \\ 26 \\ -14
                            \end{pmatrix}
                        }
                    \] \[
                        M(f)\cdot
                        \begin{pmatrix}
                            0 \\ 0 \\ 1
                        \end{pmatrix}
                        =
                        \boxed{
                            \begin{pmatrix}
                                -1 \\ 2 \\ -2
                            \end{pmatrix}
                        }
                    \]
              \item Hallar bases de $ \Nu f $ e $ \Img f $
                    \[
                        B_{\Img f} = \left\{
                        M(f) \cdot \begin{pmatrix} 1 \\ 0 \\ 0 \end{pmatrix},
                        M(f) \cdot \begin{pmatrix} 0 \\ 1 \\ 0 \end{pmatrix},
                        M(f) \cdot \begin{pmatrix} 0 \\ 0 \\ 1 \end{pmatrix},
                        \right\}
                        =
                    \] \[
                        \boxed{
                            B_{\Img f} = \left\{
                            \begin{pmatrix} 1 \\ 3 \\ -2 \end{pmatrix},
                            \begin{pmatrix} 2 \\ 1 \\ 0 \end{pmatrix},
                            \begin{pmatrix} -1 \\ 2 \\ -2 \end{pmatrix},
                            \right\}
                        }
                    \] \[
                        M(f)v = \0
                    \] \[
                        \begin{cases} % {{a + 2 b - c}, {3 a + b + 2 c}, {-2 a - 2 c}}
                            v_1 + 2v_2 - v_3  = 0                     \\
                            3v_1 + v_2 + 2v_3 = 0 \implies v_1 = -v_3 \\
                            -2v_1 - 2v_2  = 0 \implies v_1 = -v_2
                        \end{cases}
                    \] \[
                        (v_1, v_2, v_3) = (v_1, -v_1, -v_1) = v_1(1,-1,-1)
                    \] \[
                        \boxed{
                            B_{\Nu f} = \{ (1,-1,-1) \}
                        }
                    \]
          \end{enumerate}
    \item En cada caso hallar $ M_{BB'}(f) $
          \begin{enumerate}
              \item $ f : \R^3 \to \R^3 \ f(x_1, x_2, x_3) = (2x_1 + x_2, x_1 - x_3, x_1 + 2x_2 + x_3) $
                    \\ $ B = B' = E $
                    \[
                        \boxed{
                            M_{BB'}(f) = \begin{pmatrix}
                                2 & 1 & 0  \\
                                1 & 0 & -1 \\
                                1 & 2 & 1
                            \end{pmatrix}
                        }
                    \]
              \item $ f : \R^2 \to \R^2 \ f(x_1, x_2) = (x_1 - 2x_2, x_1) $ \\
                    $ B = \{(-1,0),(1,1)\} \quad B' = \{ (1,1),(0,1) \} $
%                    \[
%                        \begin{cases}
%                            f(B_1) = (-1,-1)_B \\
%                            f(B_2) = (-1,1)_B
%                        \end{cases}
%                    \] \[
%                        \left(
%                        \begin{array}{cc|c|c}
%                                -1 & 1 & -1 & -1 \\
%                                0  & 1 & -1 & 1
%                            \end{array}
%                        \right)
%                        -F_1 + F_2 \to F_1
%                        \left(
%                        \begin{array}{cc|c|c}
%                                1 & 0 & 0  & 2 \\
%                                0 & 1 & -1 & 1
%                            \end{array}
%                        \right)
%                    \] \[
%                        \boxed{
%                            M_{BB'}(f) = \begin{pmatrix}
%                                0  & 2 \\
%                                -1 & 1
%                            \end{pmatrix}
%                        }
%                    \]
              \item $ f : \R^3 \to \R^2 \ f(x_1, x_2, x_3) = (x_1 - x_2, x_1 + x_2 + x_3) $ \\
                    $ B = \{ (1,-1,2),(0,2,-1),(0,0,1) \} \quad B = \{ (2,1),(1,-1) \} $
              \item $ f : \R^4 \to \R^3 \ f(x_1, x_2, x_3, x_4) = (x_1 - x_3, 2x_4, x_2 + x_3) $ \\
                    $ B = \{ (1,-1,2,0),(0,2,-1,1),(0,0,2,1),(0,0,0,-1) \} \quad B' = E $
          \end{enumerate}
    \item \dots
\end{enumerate}
\end{document}
