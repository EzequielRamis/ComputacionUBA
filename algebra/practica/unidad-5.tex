\documentclass[../practica.root.tex]{subfiles}

\begin{document}

\section{Transformaciones Lineales}

\begin{enumerate}
    \item Determinar si la función $f$ es t. l.
          \begin{enumerate}
              \item \( f: \R^3 \to \R^2, f(x_1, x_2, x_3) = (x_1 - x_2, 2x_1) \)
                    \[ v, u \in \R^3 \]
                    \[ f(v + u) = f(v) + f(u) \]
                    \[ f(v_1 + u_1,v_2 + u_2,v_3 + u_3) = f(v_1,v_2,v_3) + f(u_1,u_2,u_3) \]
                    \[ ((v_1 + u_1) - (v_2 + u_2), 2(v_1 + u_1)) = (v_1 - v_2, 2v_1) + (u_1 - u_2, 2u_1) \]
                    \[ (v_1 - v_2 + u_1 - u_2, 2v_1 + 2u_1) = (v_1 - v_2 + u_1 - u_2, 2v_1 + 2u_1) \]
                    $f$ verifica $(v + u) = f(v) + f(u)$
                    \[ v \in \R^3, a \in \R \]
                    \[ af(v) = f(av) \]
                    \[ a(v_1 - v_2, 2v_1) = (av_1 - av_2, 2av_1) \]
                    \[ \boxed{f \text{ es t. l.}} \]
              \item \( f: \R^2 \to \R^3, f(x_1,x_2) = (x_1\cdot x_2,0,0) \)
                    \[ v, u \in \R^2 \]
                    \[ f(v + u) = f(v) + f(u) \]
                    \[ f(v_1 + u_1, v_2 + u_2) = f(v_1, v_2) + f(u_1, u_2) \]
                    \[ ((v_1 + u_1)\cdot(v_2 + u_2),0,0) = (v_1v_2,0,0) + (u_1u_2,0,0) \]
                    \[ (v_1v_2 + v_1u_2 + u_1v_2 + u_1u_2,0,0) = (v_1v_2 + u_1u_2,0,0) \]
                    \[ v_1v_2 + v_1u_2 + u_1v_2 + u_1u_2 \neq v_1v_2 + u_1u_2 \]
                    \[ \boxed{$f$ \text{ no es t. l. No verifica } $f(v + u) = f(v) + f(u)$} \]
              \item \( f: \R^2 \to \R^{3\x2}, f(x_1,x_2) =
                    \begin{pmatrix}
                        x_1  & x_1 - x_2 \\
                        0    & 0         \\
                        -x_1 & 0
                    \end{pmatrix}
                    \)
                    \[ v, u \in \R^2 \]
                    \[ f(v + u) = f(v) + f(u) \]
                    \[ f(v_1 + u_1, v_2 + u_2) = f(v_1, v_2) + f(u_1, u_2) \]
                    \[
                        \begin{pmatrix}
                            (v_1 + u_1)  & (v_1 + u_1) - (v_2 + u_2) \\
                            0            & 0                         \\
                            -(v_1 + u_1) & 0
                        \end{pmatrix}
                        =
                        \begin{pmatrix}
                            v_1  & v_1 - v_2 \\
                            0    & 0         \\
                            -v_1 & 0
                        \end{pmatrix}
                        +
                        \begin{pmatrix}
                            u_1  & u_1 - u_2 \\
                            0    & 0         \\
                            -u_1 & 0
                        \end{pmatrix}
                    \] \[
                        \begin{pmatrix}
                            v_1 + u_1  & v_1 + u_1 - v_2 - u_2 \\
                            0          & 0                     \\
                            -v_1 - u_1 & 0
                        \end{pmatrix}
                        =
                        \begin{pmatrix}
                            v_1 + u_1 & v_1 + u_1 - v_2 - u_2 \\
                            0         & 0                     \\
                            -v_1-u_1  & 0
                        \end{pmatrix}
                    \]
                    \[ v \in \R^{3\x2}, a \in \R \]
                    \[ af(v) = f(av) \]
                    \[
                        a
                        \begin{pmatrix}
                            v_1  & v_1 - v_2 \\
                            0    & 0         \\
                            -v_1 & 0
                        \end{pmatrix}
                        =
                        \begin{pmatrix}
                            av_1  & av_1 - av_2 \\
                            0     & 0           \\
                            -av_1 & 0
                        \end{pmatrix}
                    \]
                    \[ \boxed{f \text{ es t. l.}} \]
              \item \( f: \R^{2\x3} \to \R^{3\x2}, f(A) = A^t \)
                    \[ v, u \in \R^2 \]
                    \[ f(v + u) = f(v) + f(u) \]
                    \[
                        f
                        \begin{pmatrix}
                            v_{1,1} + u_{1,1} & v_{1,2} + u_{1,2} & v_{1,3} + u_{1,3} \\
                            v_{2,1} + u_{2,1} & v_{2,2} + u_{2,2} & v_{2,3} + u_{2,3} \\
                        \end{pmatrix}
                        =
                        f
                        \begin{pmatrix}
                            v_{1,1} & v_{1,2} & v_{1,3} \\
                            v_{2,1} & v_{2,2} & v_{2,3} \\
                        \end{pmatrix}
                        +
                        f
                        \begin{pmatrix}
                            u_{1,1} & u_{1,2} & u_{1,3} \\
                            u_{1,1} & u_{1,1} & u_{1,1} \\
                        \end{pmatrix}
                    \] \[
                        \begin{pmatrix}
                            (vu)_{1,1} & (vu)_{2,1} \\
                            (vu)_{1,2} & (vu)_{2,2} \\
                            (vu)_{1,3} & (vu)_{2,3} \\
                        \end{pmatrix}
                        =
                        \begin{pmatrix}
                            v_{1,1} & v_{2,1} \\
                            v_{1,2} & v_{2,2} \\
                            v_{1,3} & v_{2,3} \\
                        \end{pmatrix}
                        +
                        \begin{pmatrix}
                            u_{1,1} & u_{2,1} \\
                            u_{1,2} & u_{2,2} \\
                            u_{1,3} & u_{2,3} \\
                        \end{pmatrix}
                    \]
                    $f$ verifica $(v + u) = f(v) + f(u)$
                    \[ v \in \R^{2\x3}, a \in \R \]
                    \[ af(v) = f(av) \]
                    \[
                        a\cdot f
                        \begin{pmatrix}
                            v_{1,1} & v_{1,2} & v_{1,3} \\
                            v_{2,1} & v_{2,2} & v_{2,3} \\
                        \end{pmatrix}
                        =
                        \begin{pmatrix}
                            av_{1,1} & av_{1,2} & av_{1,3} \\
                            av_{2,1} & av_{2,2} & av_{2,3} \\
                        \end{pmatrix}
                    \] \[
                        a\cdot
                        \begin{pmatrix}
                            v_{1,1} & v_{2,1} \\
                            v_{1,2} & v_{2,2} \\
                            v_{1,3} & v_{2,3} \\
                        \end{pmatrix}
                        =
                        \begin{pmatrix}
                            av_{1,1} & av_{2,1} \\
                            av_{1,2} & av_{2,2} \\
                            av_{1,3} & av_{2,3} \\
                        \end{pmatrix}
                    \]
                    \[ \boxed{f \text{ es t. l.}} \]
          \end{enumerate}
    \item \dots
    \item \dots
    \item Hallar una base y la dimensión de \(\Nu f\) y de \(\Img f\)
          \begin{enumerate}
              \item \( f : \R^3 \to \R^3, f(x_1,x_2,x_3) = (x_1 + x_2 + x_3, x_1 - x_2, 2x_2 + x_3) \)
                    Calcular \(\Nu f\):
                    \[ f(x) = 0 \]
                    \[
                        \begin{cases}
                            x_1 + x_2 + x_3 = 0 \\
                            x_1 - x_2 = 0       \\
                            2x_2 + x_3 = 0
                        \end{cases}
                    \] \[
                        \begin{pmatrix}
                            1 & 1  & 1 \\
                            1 & -1 & 0 \\
                            0 & 2  & 1
                        \end{pmatrix}
                        \begin{array}{rl}
                            F_2 - F_1 + F_3 & \to F_2             \\
                            \frac{1}{2}F_3  & \to F_3             \\
                            F_2             & \leftrightarrow F_3
                        \end{array}
                        \begin{pmatrix}
                            1 & 1 & 1   \\
                            0 & 1 & 0.5 \\
                            0 & 0 & 0
                        \end{pmatrix}
                    \] \[
                        \boxed{\dim\Nu{f} = 3 - 2 = 1}
                    \] \[
                        \begin{cases}
                            x_1 + x_2 + x_3 = 0 \implies x_1 = -\frac{1}{2}x_3 \\
                            x_2 + \frac{1}{2}x_3 = 0 \implies x_2 = -\frac{1}{2}x_3
                        \end{cases}
                    \] \[
                        (x_1, x_2, x_3) = \left( -\frac{1}{2}x_3, -\frac{1}{2}x_3, x_3 \right) = x_3\left(-\frac{1}{2}, -\frac{1}{2}, 1\right)
                    \] \[
                        \boxed{B_{\Nu f} = \left\{\left(-\frac{1}{2}, -\frac{1}{2}, 1\right)\right\}}
                    \]
                    Calcular \(\Img f\):
                    \[ \Img f = \langle f(\hat{i}), f(\hat{j}), f(\hat{k}) \rangle = \langle (1,1,0),(1,-1,2),(1,0,1) \rangle\]
                    Verificar que los vectores sean l. i.:
                    \[
                        \alpha(1,1,0) + \beta(1,-1,2) + \gamma(1,0,1) = (\alpha + \beta + \gamma, \alpha - \beta, 2\beta + \gamma) = 0
                    \] \[
                        \begin{pmatrix}
                            1 & 1  & 1 \\
                            1 & -1 & 0 \\
                            0 & 2  & 1
                        \end{pmatrix}
                    \]
                    (Este sistema es el mismo que el del nucleo)
                    \[
                        \begin{pmatrix}
                            1 & 1 & 1   \\
                            0 & 1 & 0.5 \\
                            0 & 0 & 0
                        \end{pmatrix}
                    \]
                    El sistema es l. d. por lo que hay que hay que extraer los 2 primeros vectores para crear una base:
                    \[ \boxed{B_{\Img f}} = \{ (1,0,0),(1,1,0) \} \]
                    \[ \boxed{\dim B_{\Img f} = 2} \]

              \item \( f : \R^4 \to \R^4, f(x_1,x_2,x_3,x_4) = (x_1 + x_3, 0, x_2 + 2x_3, -x_1 + x_2 + x_3) \)
                    Calcular \(\Nu f\):
                    \[ f(x) = 0 \]
                    \[
                        \begin{cases}
                            x_1 + x_3 = 0        \\
                            0 = 0                \\
                            x_2 + 2x_3 = 0       \\
                            -x_1 + x_2 + x_3 = 0 \\
                        \end{cases}
                    \] \[
                        \begin{pmatrix}
                            1  & 0 & 1 & 0 \\
                            0  & 0 & 0 & 0 \\
                            0  & 1 & 2 & 0 \\
                            -1 & 1 & 1 & 0
                        \end{pmatrix}
                        \begin{array}{rl}
                            F_4 + F_1 - F_3 & \to F_4             \\
                            F_2             & \leftrightarrow F_3 \\
                        \end{array}
                        \begin{pmatrix}
                            1 & 0 & 1 & 0 \\
                            0 & 1 & 2 & 0 \\
                            0 & 0 & 0 & 0 \\
                            0 & 0 & 0 & 0
                        \end{pmatrix}
                    \] \[
                        \boxed{\dim \Nu f = 4 - 2 = 2}
                    \] \[
                        \begin{cases}
                            x_1 + x_3 = 0 \implies x_1 = -x_3 \\
                            x_2 + 2x_3 = 0 \implies x_2 = -2x_3
                        \end{cases}
                    \] \[
                        (x_1, x_2, x_3, x_4) = (-x_3, -2x_3, x_3, x_4) = x_3(-1,-2,1,0) + x_4(0,0,0,1)
                    \] \[
                        \boxed{B_{\Nu f} = \{ (-1,-2,1,0),(0,0,0,1) \}}
                    \]
                    Calcular \( \Img f \):
                    \[
                        \begin{array}{rl}
                            \Img f & = \langle f(1,0,0,0),f(0,1,0,0),f(0,0,1,0),f(0,0,0,1) \rangle \\
                                   & = \langle (1,0,0,-1),(0,0,1,1),(1,0,2,1),(0,0,0,0) \rangle
                        \end{array}
                    \]
                    Verificar que los vectores sean l. i.:
                    \[
                        \begin{pmatrix}
                            1  & 0 & 1 & 0 \\
                            0  & 0 & 0 & 0 \\
                            0  & 1 & 2 & 0 \\
                            -1 & 1 & 1 & 0 \\
                        \end{pmatrix}
                    \]
                    (Este sistema es el mismo que el del nucleo)
                    \[
                        \begin{pmatrix}
                            1 & 0 & 1 & 0 \\
                            0 & 1 & 2 & 0 \\
                            0 & 0 & 0 & 0 \\
                            0 & 0 & 0 & 0
                        \end{pmatrix}
                    \]
                    El sistema es l. d. por lo que hay que hay que extraer los 2 primeros vectores para crear una base:
                    \[ \boxed{B_{\Img f} = \{ (1,0,0,0),(0,1,0,0) \}} \]
                    \[ \boxed{\dim B_{\Img f} = 2} \]

              \item \( f : \R^3 \to \R^{2\x2}, f(x_1, x_2, x_3) =
                    \begin{pmatrix}
                        x_2 - x_3 & x_1 + x_3 \\
                        x_1 + x_2 & x_2 - x_3
                    \end{pmatrix} \)
                    Calcular \(\Nu f\):
                    \[ f(x) = 0 \]
                    \[
                        \begin{cases}
                            x_2 - x_3 = 0 \\
                            x_1 + x_3 = 0 \\
                            x_1 + x_2 = 0 \\
                            x_2 - x_3 = 0 \\
                        \end{cases}
                    \] \[
                        \begin{pmatrix}
                            0 & 1 & -1 \\
                            1 & 0 & 1  \\
                            1 & 1 & 0  \\
                            0 & 1 & -1 \\
                        \end{pmatrix}
                        \begin{array}{rl}
                            F_3 - F_2 - F_1 & \to F_3             \\
                            F_4 - F_1       & \to F_4             \\
                            F_1             & \leftrightarrow F_2
                        \end{array}
                        \begin{pmatrix}
                            1 & 0 & 1  \\
                            0 & 1 & -1 \\
                            0 & 0 & 0  \\
                            0 & 0 & 0  \\
                        \end{pmatrix}
                    \] \[
                        \boxed{\dim\Nu f = 3 - 2 = 1}
                    \] \[
                        \begin{cases}
                            x_1 + x_3 = 0 \implies x_1 = -x_3 \\
                            x_2 - x_3 = 0 \implies x_2 = x_3
                        \end{cases}
                    \] \[
                        (x_1, x_2, x_3) = (-x_3, x_3, x_3) = x_3(-1,1,1)
                    \] \[
                        \boxed{B_{\Nu f} = \{ (-1,1,1) \}}
                    \]
                    Calcular \( \Img f \):
                    \[
                        \begin{array}{rl}
                            \Img f & = \langle f(1,0,0),f(0,1,0),f(0,0,1) \rangle \\
                                   & =
                            \left\langle
                            \begin{pmatrix} 0 & 1 \\ 1 & 0 \end{pmatrix},
                            \begin{pmatrix} 1 & 0 \\ 1 & 1 \end{pmatrix},
                            \begin{pmatrix} -1 & 1 \\ 0 & -1 \end{pmatrix}
                            \right\rangle
                        \end{array}
                    \]
                    Los vectores no son l. i., por lo que tomamos los 2 primeros
                    \[
                        \boxed{
                            B_{\Img f} =
                            \left\{
                            \begin{pmatrix} 0 & 1 \\ 1 & 0 \end{pmatrix},
                            \begin{pmatrix} 1 & 0 \\ 1 & 1 \end{pmatrix}
                            \right\}
                        }
                    \] \[
                        \boxed{\dim B_{\Img f} = 2}
                    \]

          \end{enumerate}
    \item \dots
    \item \dots
    \item \dots
    \item \dots
    \item \dots
    \item \dots
    \item \dots
    \item \dots
\end{enumerate}
\end{document}