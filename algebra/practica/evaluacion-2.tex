\documentclass[../practica.root.tex]{subfiles}

\begin{document}

\section{Simulacro de la Segunda Evaluación}
\begin{enumerate}
    \item Sean \( \S = \langle (1,0,k,k),(3,−2,−1,k) \rangle \) y \( \H = \{ x \in \R^4 / 3x_1 + x_2 -2x_3 -kx_4 = 0 \} \). Encontrar todos los valores de \( k \in \R \) tales que \( \S \subset \H \) \\
          Reemplazar los vectores generadores en \( \H \):
          \[
              \begin{cases}
                  3 + -2k -k^2 = 0 \\
                  9 -k^2 = 0 \implies 9 = k^2
              \end{cases}
          \] \[
              3 + -2k -9 = 0
          \] \[
              3 + k = 0 \implies \boxed{k = -3}
          \]

    \item Sean \( \S = \langle (1,1,1,1),(0,6,2,3) \rangle \) y \( \T = \{ x \in \R^4 / x_1 + x_2 − 4x_3 = 0; −x_2 + x_3 + 2x_4 = 0 \} \). Una base de \( \R^4 \) que contiene una base de 𝕊 y una base de 𝕋 es: \\
        %   Encontrar conjunto generador de \( \T \):
        %   \[
        %       \begin{pmatrix}
        %           1 & 1  & −4 & 0 \\
        %           0 & −1 & 1  & 2
        %       \end{pmatrix}
        %       \begin{array}{rl}
        %           F_1 + F_2 & \to F_1 \\
        %           -F_2      & \to F_2
        %       \end{array}
        %       \begin{pmatrix}
        %           1 & 0 & -3 & 2  \\
        %           0 & 1 & -1 & -2
        %       \end{pmatrix}
        %   \] \[
        %       \begin{cases}
        %           x_1 - 3x_3 + 2x_4 = 0 \\
        %           x_2 - x_3 - 2x_4 = 0
        %       \end{cases}
        %       \implies
        %       \begin{cases}
        %           x_1 = 3x_3 -2x_4 \\
        %           x_2 = x_3 + 2x_4
        %       \end{cases}
        %   \] \[
        %       (x_1, x_2, x_3, x_4) = (3x_3 - 2x_4, x_3 + 2x_4, x_3, x_4) = (3x_3, x_3, x_3, 0) + (-2x_4, + 2x_4,0,1)
        %   \] \[
        %       \T = \langle (3,1,1,0),(-2,2,0,1) \rangle
        %   \] \[
        %       \S + \T = \boxed{\{(1,1,1,1),(0,6,2,3),(3,1,1,0),(-2,2,0,1)\}}
        %   \]
    \item Sean \( \S = \langle(1,1,0,1),(0,0,2,3)\rangle \), \( \langle \T = \{ x \in \R^4 / x_3 - x_4 = 0; x_1 + 2x_3 = 0 \} \) y \( \H = \{ x \in \R^4 / x_1 - x_2 = 0 \} \). Un subespacio \( \W \) de \( \R^4 \) tal que \( \W \subset \T \) y \( \S + \W = \H \) es
          \begin{enumerate}
              \item \( \langle (1,1,−2,−2) \rangle \) \xmark
              \item \( \langle (−2,−2,1,1) \rangle \)
              \item \( \langle (1,1,−2,−2),(0,0,1,1) \rangle \) \xmark
              \item \( \langle (−2,−2,1,1),(1,1,0,2) \rangle \) \xmark
          \end{enumerate}
          Comprobar cuales de las respuestas cumplen \( \W \subset \T \) (Solo c y d):
          \begin{align*}
              x_3 − x_4 & = 0 & x_1 + 2x_3 & = 0 \\
          \end{align*}
          \begin{align*}
              (-2) − (-2) & = 0 & 1 + 2(-2) & = 0    \\
              0           & = 0 & -3        & \neq 0 \\
          \end{align*}
          El vector \( (1,1,-2,-2) \) no pertenece a \( \T \), por lo que a y c no son validas
          \begin{align*}
              1 - 1 & = 0 & -2 + 2 & = 0 \\
              0     & = 0 & 0      & = 0 \\
          \end{align*}
          \begin{align*}
              0 - 2 & = 0    & 1 + 0 & = 0    \\
              -2    & \neq 0 & 1     & \neq 0 \\
          \end{align*}
          El vector \( (1,1,0,2) \) no pertenece a \( \T \), por lo que d no es valida
    \item Sea \( B = \{(1,-2,1),(-1,3,0),(2,-1,4) \}\) base de \( \R^3 \) y sea \( \S = \{ x \in \R^3 / x_1 + 3x_2 - x_3 = 0 \} \). El conjunto de todos los vectores \( v \in \S \) cuyas coordenadas en la base \(B\) son \( (v)_b = (a,2a,b) \) con \( a,b \in \R \) es:
          \begin{itemize}
              \item \(\langle (3,2,9) \rangle\)
              \item \(\langle (-1,4,1),(2,-1,4) \rangle\)
              \item \(\langle (1,2,2) \rangle\)
              \item \(\langle (1,2,7) \rangle\)
          \end{itemize}
    \item Sea \( f : \R^3 \to \R^3 \) una transformación lineal tal que \( f^{-1}(1,-1,2) = \{ (2,4,1) + \lambda(1,3,1),\lambda \in \R \} \) y \( f(\hat{i}) = (0,1,1) \). Encontrar \( f(0,-1,2) \) \\ \\
          Sea un vector \( v \in \{ (2,4,1) + \lambda(1,3,1), \lambda \in \R \} \), \( f(v) = (1,-1,2) \). \\
          Sea \( \lambda = -1 \implies v = (1,1,0) \)
          \begin{align*}
              f(v - \hat{i}) & = (1,-1,2) - (0,1,1) \\
              f(\hat{j})     & = (1,-2,1)
          \end{align*}
          Sea \( \lambda = 0 \implies v = (2,4,1) \)
          \begin{align*}
              f(v - 2\hat{i} - 4\hat{j}) & = (1,-1,2) - 2(0,1,1) - 4(1,-2,1) \\
              f(\hat{k})                 & = (-3,5,-4)
          \end{align*}
          \[
              \begin{cases}
                  f(\hat{i}) = (0,1,1)  \\
                  f(\hat{j}) = (1,-2,1) \\
                  f(\hat{k}) = (-3,5,-4)
              \end{cases}
          \]
          \begin{align*}
              f(0,-1,2) & = f(-\hat{j} + 2\hat{k})    \\
                        & = -f(\hat{j}) + 2f(\hat{k}) \\
                        & = -(1,-2,1) + 2(-3,5,-4)    \\
                        & = \boxed{(-7, 12, -9)}      \\
          \end{align*}
    \item Sea \( f : \R^3 \to \R^4 \) y \( M(f) = \begin{pmatrix}
              1  & 4 & -1 \\
              2  & 1 & k  \\
              -2 & 2 & -8 \\
              3  & 0 & 9
          \end{pmatrix} \) Encontrar todos los valores de \( k \in \R \) para los que \( f \) es monomorfica \\ \\
          Calcular \( \Nu f \):
          \[
              f(v) = \0
          \] \[
              \begin{pmatrix}
                  1  & 4 & -1 \\
                  2  & 1 & k  \\
                  -2 & 2 & -8 \\
                  3  & 0 & 9
              \end{pmatrix}
              \begin{pmatrix}
                  x \\ y \\ z
              \end{pmatrix}
              =
              \0
          \] \[
              \begin{pmatrix}
                  1  & 4 & -1 \\
                  2  & 1 & k  \\
                  -2 & 2 & -8 \\
                  3  & 0 & 9
              \end{pmatrix}
              \begin{array}{rl}
                  F_4 + F_3      & \to F_4 \\
                  F_2 + F_3      & \to F_2 \\
                  F_1 + F_3      & \to F_1 \\
                  \frac{1}{2}F_3 & \to F_3 \\
              \end{array}
              \begin{pmatrix}
                  -1 & 6 & -9  \\
                  0  & 3 & k-8 \\
                  -1 & 1 & -4  \\
                  1  & 2 & 1
              \end{pmatrix}
          \] \[
              \begin{array}{rl}
                  \frac{1}{8}(F_1 + F_4) & \to F_1 \\
                  \frac{1}{3}(F_3 + F_4) & \to F_3
              \end{array}
              \begin{pmatrix}
                  0 & 1 & -1  \\
                  0 & 3 & k-8 \\
                  0 & 1 & -1  \\
                  1 & 2 & 1
              \end{pmatrix}
              \begin{array}{rl}
                  F_2 - 3F_1 & \to F_2 \\
                  F_3 - F_1  & \to F_3 \\
                  F_4 - 2F_1 & \to F_4
              \end{array}
          \]  \[
              \begin{pmatrix}
                  0 & 1 & -1  \\
                  0 & 0 & k-5 \\
                  0 & 0 & 0   \\
                  1 & 0 & 3
              \end{pmatrix}
              \iff
              \begin{pmatrix}
                  1 & 0 & 3
                  0 & 1 & -1  \\
                  0 & 0 & k-5 \\
                  0 & 0 & 0   \\
              \end{pmatrix}
          \] \[
              \begin{cases}
                  x + 3z = 0 \implies x = -3z \\
                  y - z = 0  \implies y = z   \\
                  (k-5)z = 0
              \end{cases}
          \]
          Para que \( \Nu f = \{ 0 \} \) tiene que cumplirse \( z = 0 \). Para que esto suceda \( k-5 \) tiene que ser diferente a 0, ergo:
          \[ \boxed{ k \in \R - {5} } \]
    \item Sean \( B = \{ (1,1,1),(0,1,1),(0,0,1) \} \) bases de \(\{ \R^3 \}\), \( f : \R^3 \to \R^3 \) la transformación lineal tal que \( M_{BE}(f) = \begin{pmatrix}
              1 & -1 & 0 \\
              1 & 0  & 1 \\
              0 & 2  & 2
          \end{pmatrix} \) y \( g : \R^3 \to \R^3 \), \( g(x) = (-2x_1 + 2x_x - x_3, -x_1 + x_2, -3x_1 + 3x_2 - 3x_3) \). La imagen de \( g \circ f \) es: \\ \\
          Obtener \( M(f) \):
          \[ M(f) = C_{EE} * M_{BE}(f) * C_{EB} \]
          \[ M(f) = M_{BE}(f) * (C_{BE})^{-1} \]
          \[
              M(f) =
              \begin{pmatrix}
                  1 & -1 & 0 \\
                  1 & 0  & 1 \\
                  0 & 2  & 2
              \end{pmatrix}
              \begin{pmatrix}
                  1 & 0 & 0 \\
                  1 & 1 & 0 \\
                  1 & 1 & 1
              \end{pmatrix}^{-1}
          \] \[
              M(f) =
              \begin{pmatrix}
                  1 & -1 & 0 \\
                  1 & 0  & 1 \\
                  0 & 2  & 2
              \end{pmatrix}
              \begin{pmatrix}
                  1  & 0  & 0 \\
                  -1 & 1  & 0 \\
                  0  & -1 & 1
              \end{pmatrix}
          \] \[
              M(f) = \begin{pmatrix}
                  2  & -1 & 0 \\
                  1  & -1 & 1 \\
                  -2 & 0  & 2
              \end{pmatrix}
          \] \[
              M(g) = \begin{pmatrix}
                  -2 & 2 & -1 \\
                  -1 & 1 & 0  \\
                  -3 & 3 & -3
              \end{pmatrix}
          \] \[
              M(g \circ f) = M(g)M(f) = \begin{pmatrix}
                  0  & 0 & 0  \\
                  -1 & 0 & 1  \\
                  3  & 0 & -3
              \end{pmatrix}
          \] \[
              gof(x_1, x_2, x_3) = (0, -x_1 + x_3, 3x_1 - 3x_3)
          \] \[
              \Img gof = \langle g\circ f(\hat{i}),g\circ f(\hat{j}),g\circ f(\hat{k}) \rangle
          \] \[
              \Img gof = \langle (0,-1,3),(0,0,0),(0,1,-3) \rangle
          \] \[
              \boxed{\Img g\circ f = \langle (0,1,-3) \rangle}
          \]
    \item Sean \( B = \{ (0,1,0),(1,1,0),(0,-2,1) \} \) y \( B' = \{ (1,0,2),(-1,1,1),w \} \) bases de \(\R^3\) y sea \( f : \R^3 \to \R^3 \) la t. l. \( M_{BB'}(f) = \begin{pmatrix}
              1  & 0  & -1 \\
              -2 & 1  & 0  \\
              0  & -2 & 3
          \end{pmatrix} \). Si \( f(1,1,1) = (2,-2,1) \) entonces \(w\) es:
          \begin{itemize}
              \item \((0,0,1)\)
              \item \((1,-1,0)\)
              \item \((1,-3,1)\)
              \item \((-2,1,2)\)
          \end{itemize}
    \item Sea \( A = \begin{pmatrix}
              0 & 1  & -2 \\
              1 & 0  & 2  \\
              1 & -1 & 3
          \end{pmatrix} \). Encontrar un autovector de \( A \) que pertenezca a \( \S = \{ x \in \R^3 / 2x_1 + x_2 + x_3 = 0 \} \) \\ \\
          Encontrar autovalores de \( A \):
          \[ A - \lambda I \]
          \[
              \begin{pmatrix}
                  -\lambda & 1        & -2        \\
                  1        & -\lambda & 2         \\
                  1        & -1       & 3-\lambda
              \end{pmatrix}
          \]
          \begin{align*}
              \det(A - \lambda I) & = 0 \\
              -(\lambda - 1)^3    & = 0
          \end{align*}
          \[ \lambda = 1 \]
          Encontrar autovectores:
          \[
              \begin{pmatrix}
                  -1 & 1  & -2 \\
                  1  & -1 & 2  \\
                  1  & -1 & 2
              \end{pmatrix}
              v
              =
              \0
          \] \[
              \begin{cases}
                  -x + y - 2z = 0 \implies x = y - 2z
                  x - y + 2z = 0
                  x - y + 2z = 0
              \end{cases}
          \] \[
              (x, y, z) = (y - 2z, y, z) = y(1,1,0) + z(-2,0,1)
          \]
          Buscar autovectores que entren en \( \S \):
          \begin{align*}
              2x_1 + x_2 + x_3    & = 0 \\
              2(y-2z) + (y) + (z) & = 0 \\
              2y + y - 2z + z     & = 0 \\
              y = z
          \end{align*}
          \[ k((1,1,0) + (-2,0,1)) = k(-1,1,1) \ k \in \R \]
          \[ \text{Un posible valor: } \boxed{(-1,1,1)} \]
    \item Sea \( B = \{ v_1, v_2, v_3 \} \) una base de un E.V. \( \V \) y sea \( f : \V \to \V \) la t. l. tal que:
          \begin{enumerate*}
              \item \( f(v_1) = 4v_1 + kv_3 \)
              \item \( f(v_2) = v_1 + 4v_2 \)
              \item \( f(v_3) = v_1 - 4v_3 \)
          \end{enumerate*}
          y 5 es autovalor de \( f \). Entonces:
          \begin{itemize}
              \item \( k = -16 \) y \( f \) no es diagonalizable
              \item \( k = 9 \) y \( f \) es diagonalizable
              \item \( k = -16 \) y \( f \) es diagonalizable
              \item \( k = 9 \) y \( f \) no es diagonalizable
          \end{itemize}
          \[
              M(f) = \begin{pmatrix}
                  4 & 0 & k  \\
                  1 & 4 & 0  \\
                  1 & 0 & -4
              \end{pmatrix}
          \] \[
              \begin{array}{rl}
                  \det(M(f) - \lambda I)      & = 0 \\
                  \det(M(f) - 5I)      & = 0 \\
                  k - 9 & = 0
              \end{array}
          \] \[
              \boxed{k = 9}
          \]
\end{enumerate}

\end{document}
