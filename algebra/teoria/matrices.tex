\documentclass[../teoria.root.tex]{subfiles}

\begin{document}

\section{Matrices}

Una matriz es una tabla rectangular de números reales. El conjunto
$\mathbb{R}^{n\times m}$ contiene todas las matrices de $n$ filas y $m$
columnas.\footnote{Es importante recordar que es filas $\times$ columnas, no
viceversa, algo que a menudo me confundo.}

En cada conjunto $\mathbb{R}^{n\times m}$ existe una matriz cero,
$\mathbb{O}_{m\times n}$:
\[\mathbb{O}_{3\times 4}=\begin{pmatrix}
	0 & 0 & 0 & 0 \\
	0 & 0 & 0 & 0 \\
	0 & 0 & 0 & 0
\end{pmatrix}\]

En los conjuntos de matrices cuadradas, $\mathbb{R}^{n\times n}$, existe
también una matriz identidad, $\mathbb{I}_n$, que tiene $1$ en su diagonal
principal y $0$ en todos los demás lugares:
\[\mathbb{I}_3=\begin{pmatrix}
	1 & 0 & 0 \\
	0 & 1 & 0 \\
	0 & 0 & 1
\end{pmatrix}\]

Para referirse al valor de una matriz $A$ en la fila $i$ y la columna $j$, se
escribe $A_{i,j}$:
\begin{gather*}
	A=\begin{pmatrix}
		6 & 2 & 8 \\
		4 & 1 & 3 \\
		9 & 5 & 7
	\end{pmatrix}\\
	A_{2,3}=8
\end{gather*}

\subsection{Suma}

La suma y resta de matrices solo puede ocurrir entre matrices con la misma
cantidad de filas y columnas, y funciona igualmente a la suma de vectores:
\[\begin{pmatrix}
	1 & 2 \\
	3 & 4
\end{pmatrix}+\begin{pmatrix}
	5 & 6 \\
	7 & 8
\end{pmatrix}=\begin{pmatrix}
	 6 &  8 \\
	10 & 12
\end{pmatrix}\]

\subsection{Producto por escalar}

El producto de una matriz por un escalar funciona igual a la multiplicación de
vectores por escalares:
\[5\cdot\begin{pmatrix}
	1 & 2 \\
	3 & 4
\end{pmatrix}=\begin{pmatrix}
	 5 & 10 \\
	15 & 20
\end{pmatrix}\]

\subsection{Producto entre matrices}

El producto $A\cdot B$ de dos matrices $A$ y $B$ es posible únicamente si la
cantidad de columnas de $A$ es igual a la cantidad de filas de $B$. El
resultado va a ser una matriz con la cantidad de filas de $A$ y la cantidad de
columnas de $B$:
\[
	A\in\mathbb{R}^{n\times k}\land
	B\in\mathbb{R}^{k\times m}\implies
	(A\cdot B)\in\mathbb{R}^{n\cdot m}
\]

Para calcular $C=A\cdot B$, cada valor $C_{i,j}$ va a ser igual al producto
interno de la fila $i$ de $A$ por la columna $j$ de $B$. Por ejemplo:
\begin{gather*}
	\begin{pmatrix}[rrrr]
		1 & -1 & 0 &  2 \\
		\rowcolor{red!50} 3 &  0 & 4 & -2
	\end{pmatrix}\cdot
	\begin{pmatrix}[r>{\columncolor{red!50}}rr]
		 1 & 0 & 1 \\
		-1 & 1 & 3 \\
		 2 & 4 & 1 \\
		 0 & 0 & 5
	\end{pmatrix}=
	\begin{pmatrix}[rrr]
		2 & -1 & 8 \\
		11 & \cellcolor{red!50}16 & -3
	\end{pmatrix}\\
	(3,0,4,-2)\cdot(0,1,4,0)=16
\end{gather*}

El producto entre matrices no es conmutativo, no siempre $A\cdot B=B\cdot A$:
\begin{align*}
	\begin{pmatrix}[rr]
		7 & -1 \\
		3 & 1
	\end{pmatrix}\cdot\begin{pmatrix}[rr]
		1 & 0 \\
		0 & 2
	\end{pmatrix}&=\begin{pmatrix}[rr]
		7 & -2 \\
		3 & 2
	\end{pmatrix}\\
	\begin{pmatrix}[rr]
		1 & 0 \\
		0 & 2
	\end{pmatrix}\cdot\begin{pmatrix}[rr]
		7 & -1 \\
		3 & 1
	\end{pmatrix}&=\begin{pmatrix}[rr]
		7 & -1 \\
		6 & 2
	\end{pmatrix}
\end{align*}

Multiplicar una matriz por su matriz identidad o viceversa es una operación
nula, es como multiplicar un numero por $1$:
\[A\cdot\mathbb{I}=\mathbb{I}\cdot A=A\]

\subsection{Ecuaciones con matrices}

En general la estrategia para resolver ecuaciones con matrices es primeramente
determinar el tamaño de la matriz incógnita. Luego se reemplaza cada valor de
la matriz por una incógnita ($a$, $b$, $c$ \ldots) y se resuelve la ecuación
normalmente.

Como atajo, es útil recordar que una ecuación con la forma $A\cdot X=B$ se
puede resolver escalonando la matriz $(A|B)$.

\subsection{Matrices inversas}

Para algunas matrices cuadradas $A\in\mathbb{R}^{n\times n}$ (no todas, mas
adelante se vera por qué) existe una matriz inversa $A^{-1}$, tal que $A\cdot
A^{-1}=A^{-1}\cdot A=\mathbb{I}_n$. Una matriz que tiene una inversa es una
matriz \textit{inversible}. Para determinar si una matriz es inversible basta
con resolver la ecuación $A\cdot X=\mathbb{I}_n$.

\subsection{Rango}

El rango de una matriz $M$, $\rg(M)$ es la cantidad de filas no nulas de su
versión escalonada.
\begin{align*}
	&\rg\begin{pmatrix}
		3 & 1 \\
		2 & 0
	\end{pmatrix}&&=&&\rg\begin{pmatrix}
		1 & 0 \\
		0 & 1
	\end{pmatrix}&&=&&2\\
	&\rg\begin{pmatrix}[rrr]
		-1 & -1 & 1 \\
		 1 &  3 & 9 \\
		 1 &  2 & 4
	\end{pmatrix}&&=&&\rg\begin{pmatrix}[rrr]
		1 & 0 & -6 \\
		0 & 1 &  5 \\
		0 & 0 &  0
	\end{pmatrix}&&=&&2
\end{align*}

Una matriz cuadrada $M\in\mathbb{R}^{n\times n}$ es inversible solamente si
$\rg(M)=n$.
\begin{align*}
	M\text{ es inversible}&\iff\\
	M\cdot x=b\text{ tiene solo una solución}&\iff\\
	M\text{ es equivalente a }\mathbb{I}_n&\iff\\
	\rg(M)=n
\end{align*}

Dada una matriz $A\in\mathbb{R}^{n\times m}$ y un vector
$b\in\mathbb{R}^{n\times 1}$, el sistema $(A|b)$ es compatible solo si
$\rg(A)=\rg(A|b)$, y es compatible determinado solo si $\rg(A)=m$.

\begin{center}
	\begin{tabular}{ccc}
		\toprule
		\multicolumn{3}{c}{$A\in\mathbb{R}^{n\times m}\:,\: b\in\mathbb{R}^{n\times 1}$} \\
		\midrule
		\multicolumn{2}{c}{$\rg(A)=\rg(A|b)$} & $\rg(A)\neq\rg(A|b)$ \\
		\midrule
		$\rg(A)=m$ & $\rg(A)<m$ & \multirow{2}{*}{Incompatible} \\
		Determinado & Indeterminado \\
		\bottomrule
	\end{tabular}
\end{center}

\subsection{Determinantes}

Dada una matriz cuadrada, se puede calcular su determinante, un numero real. El
determinante de una matriz se escribe $\det(A)$ o $|A|$.

Para calcularlo, vamos a ver el caso para una matriz $2\times2$:
\[\begin{vmatrix}a&b\\c&d\end{vmatrix}=ad-bc\]

\begin{enumerate}
	\item Se elige una fila $i$ de la matriz (conviene elegir la que tiene mas
		ceros).
	\item Se tacha $i$.
	\item Por cada columna $j$ de la matriz:
	\begin{enumerate}
		\item Se tacha $j$.
		\item Se calcula el determinante de la matriz resultante (De esta
			manera, el tamaño de la matriz a calcular se va a reducir hasta
			llegar al caso $2\times2$, que tiene una solución fácil).
		\item Se multiplica el resultado por el valor en $M_{i,j}$
		\item Si $i+j$ es impar, se invierte el resultado.
	\end{enumerate}
	\item Se suman todos estos valores.
\end{enumerate}

Este proceso también se puede hacer tachando una columna y sumando entre filas.
En general conviene hacer el método bajo el cual la fila o columna elegida
tiene la mayor cantidad de ceros, lo cual acorta el cálculo.

\[
	\begin{vmatrix}[>{\columncolor{red!30}}r>{\columncolor{green!30}}r>{\columncolor{blue!30}}r]
		\rowcolor{purple!30} 1 & 2 & 3 \\
		4 & 5 & 6 \\
		7 & 8 & 9
	\end{vmatrix}
\]

\end{document}
