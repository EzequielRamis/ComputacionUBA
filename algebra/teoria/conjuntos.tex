\documentclass[../teoria.root.tex]{subfiles}

\begin{document}

\section{Conjuntos}

Un conjunto es una colección de elementos no ordenados y no repetidos. Se
pueden describir enumerando todos sus elementos entre llaves.
\[S=\{1,2,3,4\}\]

Si un elemento $x$ pertenece a un conjunto $A$, se escribe $x\in A$, y si no
pertenece, $x\notin A$.
\[2\in S\]
\[5\notin S\]

Otra forma de describir conjuntos es mediante una expresión, seguida de dos
puntos (o una barra vertical) y una condición que las variables de la expresión
deben cumplir.\footnote{En general la pertenencia de las variables a otros
conjuntos se escribe en la parte izquierda si es una expresión simple como en
este ejemplo, para ser mas legible, pero también se puede poner en el lado
derecho.}
\[S=\{n\in\mathbb{N}:n\leq4\}\]

Si todos los elementos de un conjunto $A$ también están en un conjunto $B$, el
primer conjunto esta \textit{incluido} en el otro, y esto se escribe $A\subset
B$. Si no todos los elementos están, el conjunto no esta incluido, y se escribe
$A\not\subset B$.
\[\{1,2\}\subset S\]
\[\{4,5\}\not\subset S\]

Un conjunto $C$, que tenga todos los elementos que estén en $A$ o $B$, sería la
\textit{unión} de $A$ y $B$, y se escribe $C=A\cup B$.
\[S=\{1,2\}\cup\{3,4\}\]

Un conjunto $D$, que contenga todos los elementos que estén simultáneamente en
$A$ \textbf{y} $B$, sería la \textit{intersección} de $A$ y $B$, y se escribe
$C=A\cap B$.
\[S\cap\{3,4,5,6\}=\{3,4\}\]
\[\{5,6\}\cap\{7,8\}=\{\}\]

\subsection{Conjuntos comunes}

\begin{itemize}
	\item $\mathbb{N}=\left\{1,2,3,4,\ldots\right\}$, los números naturales.
	\item $\mathbb{Z}=\mathbb{N}\cup\{-n:n\in\mathbb{N}\}\cup\{0\}$, los números enteros.
	\item $\mathbb{Q}=\left\{\frac{a}{b}:a\in\mathbb{Z}\land b\in\mathbb{N}\right\}$, los números racionales.
	\item $\mathbb{R}$, los números reales.
	\item $\varnothing=\{\}$, el conjunto vacío.
\end{itemize}

\end{document}
