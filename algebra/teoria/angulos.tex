\documentclass[../teoria.root.tex]{subfiles}

\begin{document}

\section{Ángulos}

\subsection{Radianes}

En vez de utilizar grados, en esta materia vamos a utilizar radianes para
expresar ángulos.

Al dibujar una linea que este a un ángulo $\theta$ del eje $x$ (por convención
se empieza en el extremo positivo del eje, y se mueve en dirección anti
horaria), la longitud del arco formado en un circulo de radio $1$ hasta ese
ángulo es la medida de $\theta$ en radianes.

En grados, una rotación completa es de \ang{360}, pero en radianes es de $2\pi$
(la circunferencia de un circulo de radio $1$). En general, para convertir de
grados a radianes se puede usar la siguiente fórmula, donde $\alpha$ es el
ángulo en radianes y $\beta$ es el ángulo en grados:
\[\frac{\alpha}{\pi}=\frac{\beta}{\ang{180}}\]

\subsection{Circulo unitario}

Un circulo de radio $1$, también llamado un circulo unitario, puede servir como
una gran ayuda visual para entender los radianes y otros conceptos de
trigonometría:

\begin{center}
	\begin{tikzpicture}[scale=3]
		\pgfmathsetmacro{\a}{45}
		\draw[->] (-1.5,0) -- (1.5,0);
		\draw[->] (0,-1.5) -- (0,1.5);
		\draw[help lines] circle(1);
		\draw[purple,thick] (1,0) arc (0:\a:1) node[midway,right]{$\theta$};
		\draw[blue,thick] (0,0) -- (\a:1) node[midway,sloped,above]{$1$};
		\draw[green,thick] (0,0) -- ($(cos{\a},0)$) node[midway,below]{$\cos\theta$};
		\draw[red,thick] ($(cos{\a},0)$) -- (\a:1) node[midway,sloped,above]{$\sin\theta$};
		\fill (\a:1) circle(0.5pt) node[above right]{$(\cos{\theta},\sin{\theta})$};
		\fill (0,0) circle(0.5pt);
		\fill ( 1, 0) circle(0.5pt) node[above right]{$0$};
		\fill ( 0, 1) circle(0.5pt) node[above left]{$\frac{1}{2}\pi$};
		\fill (-1, 0) circle(0.5pt) node[below left]{$\pi$};
		\fill ( 0,-1) circle(0.5pt) node[below right]{$\frac{3}{2}\pi$};
		\fill ( 1, 0) circle(0.5pt) node[below right]{$2\pi$};
	\end{tikzpicture}
\end{center}

\subsection{Funciones Trigonométricas}

Las funciones trigonométricas sirven para conocer los ángulos internos de un
triangulo rectángulo, conociendo las longitudes de dos de sus lados:

\begin{center}
	\begin{tikzpicture}
		\pgfmathsetmacro{\a}{25}
		\pgfmathsetmacro{\x}{4}
		\coordinate (p) at ($(\x,tan{\a}*\x)$);
		\draw (0,0) -- (\x,0) node[midway,below]{$CA$};
		\draw (\x,0) -- (p) node[midway,right]{$CO$};
		\draw (0,0) -- (p) node[midway,sloped,above]{$H$};
		\draw (1,0) arc(0:\a:1) node[midway,right]{$\theta$};
	\end{tikzpicture}
\end{center}

En términos de las longitudes del cateto adyacente ($CA$), opuesto ($CO$) y la
hipotenusa ($H$) en un triangulo rectángulo, podemos definir el coseno y el
seno del angulo $\theta$ como:
\[\cos\alpha=\frac{CA}{H}\]
\[\sin\alpha=\frac{CO}{H}\]

Ya que en un circulo de radio $1$ la hipotenusa formada por la linea siempre va
a medir $1$, podemos ver que la coordenada $x$ del punto en el circulo va a ser
$\cos\alpha$, y la coordenada $y$ va a ser $\sin\alpha$.

\end{document}
