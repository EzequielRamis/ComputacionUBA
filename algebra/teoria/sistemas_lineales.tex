\documentclass[../teoria.root.tex]{subfiles}

\begin{document}

\section{Sistemas Lineales}

Una ecuación lineal es una ecuación que consiste de una suma de productos de
incógnitas:
\[a_1x_1+a_2x_2+\cdots+a_nx_n=b\]

$a_1$, $a_2$, \ldots, $a_n$ serian los coeficientes de la ecuación, y $x_1$,
$x_2$, \ldots, $x_n$ serian las incógnitas. La suma de los productos de los
coeficientes por las incógnitas van a ser iguales a un numero real $b$.

La ecuación implícita de un plano es sencillamente una ecuación lineal de grado
3. Las incógnitas $x_1$, $x_2$, \ldots, $x_n$ se pueden interpretar como las
coordenadas de un vector $(x_1,x_2,\ldots,x_n)$.

Cuando $b=0$, la ecuación es \textit{homogénea}, y el vector $(0,\ldots,0)$
siempre va a ser una solución, la \textit{solución trivial}.

Un sistema de ecuaciones lineales reúne varias ecuaciones lineales, y busca una
solución que satisfaga todas al mismo tiempo.
\[\systeme{
	2x_1+x_2-x_3+3x_4=0,
	x_2+x_3-x_4=4,
	-x_3+2x_4=-3
}\]

Si todas las ecuaciones del sistema son homogéneas, el sistema es homogéneo, y
también tiene una solución trivial.

Los sistemas de ecuaciones lineales se pueden categorizar en tres grupos
dependiendo de cuantas soluciones tiene:

\begin{itemize}
	\item \textit{Sistema Compatible Determinado:} Una sola solución.
	\item \textit{Sistema Compatible Indeterminado:} Infinitas soluciones.
	\item \textit{Sistema Incompatible:} Sin solución.
\end{itemize}

\subsection{Sistemas escalonados}

Si un sistema tiene una forma particular, puede ser muy fácil de resolver. Por
ejemplo:
\begin{gather*}
	\systeme{
		x_1+x_2=5,
		3x_2=-6
	}
	\intertext{Este sistema es fácil de resolver, ya que la ultima ecuación
	tiene solo una incógnita, que se puede fácilmente despejar:}
	\systeme{
		x_1+x_2=5,
		x_2=-2
	}
	\intertext{Luego se reemplaza en la otra ecuación, y se despeja todo:}
	\systeme{
		x_1=7,
		x_2=-2
	}
\end{gather*}

Un sistema con esta forma se llama un \textit{sistema escalonado}, ya que cada
fila tiene progresivamente menos incógnitas.

Veamos un ejemplo mas complicado de un sistema escalonado:
\[\systeme{
	2x_1+x_2-x_3+3x_4=0,
	x_2+x_3-x_4=4,
	-x_3+2x_4=-3
}\]

En este caso, la ultima fila tiene 2 incógnitas. Podemos despejar solo una, y
repetir el proceso anterior. Vamos a poder despejar todas las incógnitas
excepto una, que queda libre:
\[\systeme{
	x_1=1,
	x_2=-x_4+1,
	x_3=2x_4+3
}\]

Para cualquier valor de $x_4$, el sistema sigue resuelto. Este sistema tiene
infinitas soluciones, y es equivalente a la ecuación paramétrica de una recta
en $\mathbb{R}^4$:
\begin{align*}
	X=(x_1,x_2,x_3,x_4)&=(1,-x_4+1,2x_4+3,x_4)\\
	&=\lambda(0,-1,2,1)+(1,1,3,0)\\
\end{align*}

\subsection{Matriz ampliada asociada a un sistema}

Para solucionar un sistema linear fácilmente, se debe transformar en un sistema
escalonado. Para hacer esto mas fácil, vamos a usar una nueva notación para
describir sistemas lineales, las matrices:
\[
	\systeme{x_2+4x_3=3,x_1+2x_3=5,2x_1+3x_3=8,-x_1+x_2+3x_3=0}
	\quad\longrightarrow\quad
	\begin{pmatrix}[rrr|r]
		 0 & 1 & 4 & 3\\
		 1 & 0 & 2 & 5\\
		 2 & 0 & 3 & 8\\
		-1 & 1 & 3 & 0
	\end{pmatrix}
\]

Cada fila de la matriz es equivalente a una ecuación del sistema, con una
columna por cada incógnita, y una columna para el termino constante.

Hay 3 operaciones que se le pueden aplicar a una matriz sin modificar las
soluciones de su sistema equivalente:

\begin{center}
	\begin{tabular}{rcll}
		$F_i$&$\leftrightarrow$&$F_j$&Intercambiar dos filas\\
		$F_i+\alpha F_j$&$\to$&$F_i$&A una fila sumarle un múltiplo de otra\\
		$\beta F_i$&$\to$&$F_i$&Multiplicar una fila por un numero ($\neq0$)
	\end{tabular}
\end{center}

\subsubsection{Matrices escalonadas}

En una matriz escalonada cada fila va a empezar con mas ceros que la fila
anterior, y el primer numero distinto a cero de cada fila va a ser $1$. Por
ejemplo:
\[\begin{pmatrix}[rrr|r]
	1 & 0 & 1 &  1\\
	0 & 1 & 3 &  2\\
	0 & 0 & 1 & -1\\
\end{pmatrix}\]

Dada una matriz escalonada, es trivial usar las operaciones anteriormente
mencionadas para conseguir las soluciones del sistema:
\begin{align*}
	\begin{pmatrix}[rrr|r]
		1 & 0 & 1 &  1\\
		0 & 1 & 3 &  2\\
		0 & 0 & 1 & -1\\
	\end{pmatrix}
	&&\begin{aligned}
		F_1-F_3&\to F_1\\
		F_2-3F_3&\to F_2
	\end{aligned}
	&&\begin{pmatrix}[rrr|r]
		1 & 0 & 0 &  2\\
		0 & 1 & 0 &  5\\
		0 & 0 & 1 & -1\\
	\end{pmatrix}
\end{align*}

Nuestro objetivo va a ser combinar estas operaciones elementales para
transformar nuestra matriz en una escalonada, para poder resolverla mas fácil.

\subsection{Eliminación de Gauss}

La forma mas fácil de escalonar matrices es con el método de eliminación de
Gauss. El procedimiento es el siguiente:

\begin{enumerate}
	\item Ordenar las filas de la matriz según su cantidad de ceros iniciales.
	\item Por cada fila:
		\begin{itemize}
			\item Dividir la fila por su coeficiente principal (el primer valor
				$\neq0$ de la fila), para que este sea $1$.
			\item Restar múltiplos de la fila en las demás, tal que los valores
				debajo del coeficiente principal de la fila sean $0$.
		\end{itemize}
\end{enumerate}


\begin{gather*}
	\intertext{Como ejemplo, vamos a resolver el sistema de ecuaciones lineales
	representado por la siguiente matriz:}
	\begin{pmatrix}[rrr|r]
		 0 & 1 & 4 & 3\\
		 1 & 0 & 2 & 5\\
		 2 & 0 & 3 & 8\\
		-1 & 1 & 3 & 0
	\end{pmatrix}\\
	\begin{align*}
		\intertext{Ordenamos las filas por la cantidad de ceros iniciales:}
		F_1\leftrightarrow F_4\quad
		&\begin{pmatrix}[rrr|r]
			-1 & 1 & 3 & 0\\
			 1 & 0 & 2 & 5\\
			 2 & 0 & 3 & 8\\
			 0 & 1 & 4 & 3
		\end{pmatrix}\\
		\intertext{Dividimos la primera fila por su coeficiente principal:}
		-F_1\to F_1\quad
		&\begin{pmatrix}[rrr|r]
			1 & -1 & -3 & 0\\
			1 &  0 &  2 & 5\\
			2 &  0 &  3 & 8\\
			0 &  1 &  4 & 3
		\end{pmatrix}\\
		\intertext{Restamos la primera fila en las demás para dejar vacíos los
		valores debajo del coeficiente principal:}
		\begin{aligned}
			F_2-F_1&\to F_2\\
			F_3-2F_1&\to F_3\\
		\end{aligned}\quad
		&\begin{pmatrix}[rrr|r]
			1 & -1 & -3 & 0\\
			0 &  1 &  5 & 5\\
			0 &  2 &  9 & 8\\
			0 &  1 &  4 & 3
		\end{pmatrix}\\
		\intertext{Repetimos el proceso con la segunda fila:}
		\begin{aligned}
			F_3-2F_2&\to F_3\\
			F_4-F_2&\to F_4\\
		\end{aligned}\quad
		&\begin{pmatrix}[rrr|r]
			1 & -1 & -3 &  0\\
			0 &  1 &  5 &  5\\
			0 &  0 & -1 & -2\\
			0 &  0 & -1 & -2
		\end{pmatrix}\\
		\intertext{Y la tercera:}
		\begin{aligned}
			-F_3&\to F_3\\
			F_4+F_3&\to F_4\\
		\end{aligned}\quad
		&\begin{pmatrix}[rrr|r]
			1 & -1 & -3 & 0\\
			0 &  1 &  5 & 5\\
			0 &  0 &  1 & 2\\
			0 &  0 &  0 & 0
		\end{pmatrix}
	\end{align*}\\
	\intertext{Ahora que la matriz esta escalonada, podemos fácilmente resolver
	el sistema:}
	\begin{aligned}
		F_1+F_2&\to F_1\\
		F_1-2F_3&\to F_1\\
		F_2-5F_3&\to F_2
	\end{aligned}\quad
	\begin{pmatrix}[rrr|r]
		1 & 0 & 0 &  1\\
		0 & 1 & 0 & -5\\
		0 & 0 & 1 &  2\\
		0 & 0 & 0 &  0
	\end{pmatrix}\to
	\systeme{x_1=1,x_2=-5,x_3=2,0=0}
\end{gather*}

\subsection{Resolución simultánea de sistemas}

Si las matrices ampliadas de varios sistemas son iguales (salvo su ultima
columna), ambas se pueden resolver en simultaneo al combinarlas:
\begin{gather*}
	(A|b_1)\:(A|b_2)\to (A|b_1|b_2)\\
	\begin{pmatrix}[rrrr|r]
		1 &  1 & -1 & 1 &  1 \\
		2 & -1 &  0 & 1 &  0 \\
		2 & -4 &  2 & 0 & -2 \\
	\end{pmatrix}\:
	\begin{pmatrix}[rrrr|r]
		1 &  1 & -1 & 1 & 0 \\
		2 & -1 &  0 & 1 & 3 \\
		2 & -4 &  2 & 0 & 2 \\
	\end{pmatrix}\to
	\begin{pmatrix}[rrrr|r|r]
		1 &  1 & -1 & 1 &  1 & 0 \\
		2 & -1 &  0 & 1 &  0 & 3 \\
		2 & -4 &  2 & 0 & -2 & 2 \\
	\end{pmatrix}
\end{gather*}

\end{document}
