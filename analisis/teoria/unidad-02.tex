% !TEX root = ../teoria.root.tex

\documentclass[../teoria.root.tex]{subfiles}

\begin{document}
\section{Sucesiones}
\subsection{Introducción}
Las sucesiones son objetos matemáticos muy sencillos que se apoyan en la ordenación de un conjunto (finito o infinito) de números reales.
Sirven, por ejemplo, para estudiar, representar y predecir los fenómenos que ocurren o se miden en el tiempo, en forma intermitente.
\subsubsection{Un problema a modo de presentación}
El problema consiste en encontrar un algoritmo que calcule la raíz cuadrada de un número dado (por ejemplo \(\sqrt{2}\)), utilizando sólo las cuatro operaciones básicas.

Una solución al problema se basa en una idea geométrica:

Se construyen sucesivos rectángulos todos de área 2.
La base de cada uno de ellos es el promedio de la base y la altura del anterior.
\begin{center}
	\begin{scaletikzpicturetowidth}{\linewidth}
		\begin{tikzpicture}[scale=\tikzscale]
			\node(rect)[draw=black,ultra thick, minimum width=2*1cm,minimum height=2*2cm,label=right:2,label=below:1](A){\(\textstyle1\cdot2=2\)};
			\node(rect)[right=of A,draw=black,ultra thick, minimum width=2*1.5cm,minimum height=2*1.333cm,label=right:\(\textstyle\frac{2}{1,5}\),label=below:\(\textstyle\text{\(\frac{1+2}{2}=1,5\)}\)](B){\(\textstyle1,5\cdot\frac{2}{1,5}=2\)};
			\node(rect)[right=of B,draw=black,ultra thick, minimum width=2*1.417cm,minimum height=2*1.412cm,label=right:\(\textstyle\text{1,4118}\),label=below:\(\textstyle\text{\(\frac{1,5+\frac{2}{1,5}}{2}\cong1,4167\)}\)](c){\(\scriptstyle1,4167\cdot1,4118\cong2\)};
		\end{tikzpicture}
	\end{scaletikzpicturetowidth}
\end{center}
Designamos con \(x_1\) a la medida de la base del primer rectángulo, que elegimos que fuera igual a 1, con \(x_2\) a lo que mide la base del segundo rectángulo, y así sucesivamente.

Entonces resulta:
\[x_1=1,x_2=\frac{x_1+\frac{2}{x_1}}{2}=1,5,x_3=\frac{x_2+\frac{2}{x_2}}{2}=1,4167,\dots,x_{n+1}=\frac{x_n+\frac{2}{x_n}}{2}\]
Geométricamente se observa que los rectángulos se van aproximando a un cuadrado de área 2, por lo cual las bases \(x_n\) se van aproximando al lado del cuadrado de área 2, es decir \(x_n\rightarrow\sqrt{2}\) (\(x_n\) se aproxima a \(\sqrt{2}\)).

\subsubsection{Ejemplos de sucesiones}
Consideramos los siguientes ejemplos:
\begin{enumerate}
	\item \(1,\frac{1}{2},\frac{1}{3},\frac{1}{4},\dots\)
	\item 1, 3, 5, 7, \dots
	\item \(-\frac{1}{2},\frac{1}{4},-\frac{1}{8},\frac{1}{16},\dots\)
	\item \(\frac{1}{2},\frac{2}{3},\frac{3}{4},\frac{4}{5},\dots\)
	\item 0, 1, 0, 1, \dots
	\item 2, -4, 6, -8, \dots
\end{enumerate}
Informalmente, una \textit{sucesión} es una \textit{lista ordenada} e infinita de números reales.

Nos interesará el “comportamiento a la larga” de cada lista de números.
En otras palabras, nos interesará saber si, a medida que avanzamos en la lista de números, éstos se parecen o aproximan a un número determinado.
Habrá que dar más precisión a esta idea.

Observemos por el momento, que una lista ordenada de números se puede describir con el lenguaje de las funciones que vimos en el primer módulo, usando como conjunto “ordenador” a los números naturales.

Una \textit{sucesión} es una función \(a:\N\rightarrow\R\), se escribe \(a(n)=a_n\).
Se lee “\(a\,sub\,n\)”.
Indica el número real de la lista en la posición \(n\).

Observemos la lista de sucesiones con las que comenzamos esta sección:
En la sucesión \textbf{1.} \(a_3=\frac{1}{3}\) y \(a_{100}=\frac{1}{100}\); en la sección \textbf{4.} \(a_4=\frac{4}{5}\) y \(a_{1000}=\frac{1000}{1001}\).

Lo que interesará es el comportamiento de \(a_n\) para “valores grandes” de \(n\).
\subsubsection{Término general}
Es la expresión de \(a_n\) para cada \(n\).
Analizamos cada una de las sucesiones anteriores
\begin{enumerate}
	\item \(a_n=\frac{1}{n}\)\tab\(1,\frac{1}{2},\frac{1}{3},\frac{1}{4},\dots\)
	\item \(a_n=2n-1\)\tab\(1,3,5,7,\dots\)
	\item \(a_n=(-1)^n\frac{1}{2^n}\)\tab\(-\frac{1}{2},\frac{1}{4},-\frac{1}{8},\frac{1}{16},\dots\)
	\item \(a_n=\frac{n}{n+1}\)\tab\(\frac{1}{2},\frac{2}{3},\frac{3}{4},\frac{4}{5},\dots\)
	\item \(a_n=\begin{cases}
		      0 & \text{si \(n\) es impar} \\
		      1 & \text{si \(n\) es par}
	      \end{cases}\)\tab\(0,1,0,1,\dots\)
	\item \(a_n=(-1)^{n+1}2n\)\tab\(2,-4,6,-8,\dots\)
\end{enumerate}
\subsubsection{Representación gráfica}
Como una sucesión es una función admite una representación gráfica, veamos alguno de los ejemplos anteriores:
\begin{enumerate}
	\item \addtocounter{enumi}{1}\(1,\frac{1}{2},\frac{1}{3},\frac{1}{4},\dots\)
	      \begin{center}
		      \begin{scaletikzpicturetowidth}{\linewidth}
			      \begin{tikzpicture}[scale=\tikzscale]
				      \draw[->,thick] (0,-.1) -- (0,1.1);
				      \draw[->,thick] (-.1,0) -- (2.1,0);
				      \foreach \x/\xl/\y/\yl in {.2/1/1/1,.4/2/{1/2}/{1/2},.6/3/{1/3}/{1/3},.8/4/{1/4}/{1/4},1.4/\(n\)/{1/10}/{1/\(n\)}} {
				      \node[draw,shape=circle,fill=black,scale=0.4] at (\x,\y)(n){};
				      \node (yl) at (0,\y){};
				      \node (xl) at (\x,0){};
				      \draw[dashed] node[left=1mm of yl]{\(\scriptstyle\text{\yl}\)} (yl) -- (n) -- (\x,0) node[below=1mm of xl]{\xl};
				      }
			      \end{tikzpicture}
		      \end{scaletikzpicturetowidth}
	      \end{center}
	      \[a_n=\frac{1}{n}\rightarrow0\]
	\item \(-\frac{1}{2},\frac{1}{4},-\frac{1}{16},\dots\)
	      \begin{center}
		      \begin{scaletikzpicturetowidth}{\linewidth}
			      \begin{tikzpicture}[scale=\tikzscale]
				      \draw[->,thick] (0,-1) -- (0,.6);
				      \draw[->,thick] (-.1,0) -- (2.1,0);
				      \foreach \x/\xl/\y/\yl in {.2/1/{-1/2}/{-1/2},.4/2/{1/4}/{1/4},.6/3/{-1/8}/{-1/8},.8/4/{1/16}/{1/16},1.4/\(n\)/0/ } {
				      \node[draw,shape=circle,fill=black,scale=0.4] at (\x,\y)(n){};
				      \node (yl) at (0,\y){};
				      \node (xl) at (\x,0){};
				      \draw[dashed] node[left=1mm of yl]{\(\scriptstyle\text{\yl}\)} (yl) -- (n) -- (\x,0) node[below=2mm of xl]{\xl};
				      }
			      \end{tikzpicture}
		      \end{scaletikzpicturetowidth}
	      \end{center}
	      \[a_n=(-1)^n\frac{1}{2^n}\rightarrow0\]
	\item \(\frac{1}{2},\frac{2}{3},\frac{3}{4},\frac{4}{5},\dots\)
	      \begin{center}
		      \begin{scaletikzpicturetowidth}{\linewidth}
			      \begin{tikzpicture}[scale=\tikzscale]
				      \draw[->,thick] (0,-.1) -- (0,1.1);
				      \draw[->,thick] (-.1,0) -- (2.1,0);
				      \draw[thick] (0,1) -- node[above]{1} (2.1,1);
				      \foreach \x/\xl/\y/\yl in {.2/1/{1/2}/{1/2},.4/2/{2/3}/{2/3},.6/3/{3/4}/{3/4},.8/4/{4/5}/{4/5},1.4/\(n\)/{9/10}/\(a_n\)} {
				      \node[draw,shape=circle,fill=black,scale=0.4] at (\x,\y)(n){};
				      \node (yl) at (0,\y){};
				      \node (xl) at (\x,0){};
				      \draw[dashed] node[left=1mm of yl]{\(\scriptstyle\text{\yl}\)} (yl) -- (n) -- (\x,0) node[below=1mm of xl]{\xl};
				      }
			      \end{tikzpicture}
		      \end{scaletikzpicturetowidth}
	      \end{center}
	      \[a_n=\frac{n}{n+1}\rightarrow1\]
\end{enumerate}
\subsection{Noción de límite}
Intuitivamente, una sucesión \textit{tiende} a un valor determinado \(L\) si los valores de \(a_n\) \textit{están cerca} de \(L\) cuando \(n\) es grande.
Un poco más precisamente:
el error que se comete al aproximar \(L\) con \(a_n\) es pequeño (menor que épsilon (\(\epsilon\))) si \(n\) es bastante grande (más que \(n_0\) en el gráfico)
\begin{center}
	\begin{scaletikzpicturetowidth}{\linewidth}
		\begin{tikzpicture}[scale=\tikzscale]
			\draw[->,thick] (0,-.1) -- (0,1.1);
			\draw[->,thick] (-.1,0) -- (2,0);
			\draw[dashed] (0,.8) node[left]{\(L+\epsilon\)} -- (2,.8);
			\draw[dashed] (0,.5) node[left]{\(L\)} -- (2,.5);
			\draw[dashed] (0,.2) node[left]{\(L-\epsilon\)} -- (2,.2);
			\draw[<->,dashed,thick] (.5,.5) -- (.5,.8) node[above]{\(\epsilon\)};
			\draw[dashed] (1.2,0) node[below]{\(n_0\)} -- (1.2,1);
			\fill[green,opacity=0.2] (1.2,.2) -- (1.2,.8) -- (2,.8) -- (2,.2) -- cycle;
			\foreach \x/\y in {.2/.9,.3/.3,.6/.1,.8/.6,1/.9,1.3/.75,1.5/.4,1.7/.6,1.9/.45} {
					\node[draw,shape=circle,fill=black,scale=0.4] at (\x,\y){};
				}
			\node[draw,text width=3.5cm] at (3.5,.5) {\textbf{Idea geométrica}\\A partir de \(n_0\) la franja verde capta a todos los \(a_n\)};
		\end{tikzpicture}
	\end{scaletikzpicturetowidth}
\end{center}
La definición precisa de límite es la siguiente:

Se dice que \(a_n\) tiene \textit{límite L} si, cualquiera sea \(\epsilon>0\), existe un número natural \(n_0\) tal que si \(n\geq0\), entonces \[L-\epsilon<a_n<L+\epsilon\,\text{(o sea \(|a_n-L|<\epsilon\,\text{si}\,n\geq n_0\))}\]
Se escribe \(\lim_{n\to\infty}a_n=L\) o \(a_n\rightarrow L\), se lee ``\textit{el límite de a sub n cuando n tiende a infinito es L}''.
También se dice en tal caso que \(a_n\in(L-\epsilon,L+\epsilon)\) \textit{para casi todo n (pctn)}.
En general, una propiedad vale \textit{para casi todo n} si vale para todo \(n\) salvo un número finito de valores de \(n\).
Se pone \(pctn\).
\subsubsection{Sucesiones divergentes}
No todas las sucesiones convergen a un límite \(L\in\R\), por ejemplo:
\begin{itemize}
	\item \(a_n=2n-1\)\tab\(1,3,5,7,\dots\)
	      \begin{center}
		      \begin{scaletikzpicturetowidth}{.5\linewidth}
			      \begin{tikzpicture}[scale=\tikzscale]
				      \draw[->,thick] (0,-1) -- (0,10);
				      \draw[->,thick] (-1,0) -- (10,0);
				      \foreach \x/\xl/\y/\yl in {1/1/1/1,2/2/3/3,3/3/5/5,4/\(n_0\)/7/\(K\),5/\(n\)/9/\(2n-1\)} {
						      \node[draw,shape=circle,fill=black,scale=0.4] at (\x,\y)(n){};
						      \node (yl) at (0,\y){};
						      \node (xl) at (\x,0){};
						      \draw[dashed] node[left=1mm of yl]{\(\textstyle\text{\yl}\)} (yl) -- (n) -- (\x,0) node[below=1mm of xl]{\xl};
					      }
				      \draw[dashed,ultra thin] (1,1) -- (5,9);
				      \fill[green,opacity=0.2] (4,7) -- (4,10) -- (6,10) -- (6,7) -- cycle;
			      \end{tikzpicture}
		      \end{scaletikzpicturetowidth}
	      \end{center}
	      \[\lim_{n\to\infty}(2n-1)=+\infty\] Para todo \(K>0\) existe \(n_0\in\N\) tal que si \(n>n_0,2n-1>K\) (en el sector verde).
	\item \(b_0=\begin{cases}
		      0 & \text{si \(n\) es impar} \\
		      1 & \text{si \(n\) es par}
	      \end{cases}\)\tab\(0,1,0,1,\dots\)
	      \begin{center}
		      \begin{scaletikzpicturetowidth}{.5\linewidth}
			      \begin{tikzpicture}[scale=\tikzscale]
				      \draw[->,thick] (0,-1) -- (0,6);
				      \draw[->,thick] (-.5,0) -- (10,0);
				      \foreach \y [count=\x] in {0,1,0,1,0,1} {
						      \node[draw,shape=circle,fill=black,scale=0.4] at (\x,\y*5)(n){};
						      \node (yl) at (0,\y*5){};
						      \node (xl) at (\x,0){};
						      \draw node[left=1mm of yl]{\y} (yl);
						      \draw[dashed] (n) -- (\x,0) node[below=1mm of xl]{\x};
					      }
				      \draw[dashed] (0,5) -- (10,5);
			      \end{tikzpicture}
		      \end{scaletikzpicturetowidth}
	      \end{center}
	      \[\lim_{n\to\infty}b_n\,\text{no existe}\] En este caso se dice que \(a_n\) oscila finitamente.
	\item \(c_n=(-1)^{n-1}2n\)\tab\(2,-4,6,-8,\dots\)
	      \begin{center}
		      \begin{scaletikzpicturetowidth}{.5\linewidth}
			      \begin{tikzpicture}[scale=\tikzscale]
				      \draw[->,thick] (0,-11) -- (0,11);
				      \draw[->,thick] (-1,0) -- (20,0);
				      \foreach \y [count=\x] in {2,-4,6,-8} {
						      \node[draw,shape=circle,fill=black,scale=0.4] at (\x*2,\y)(n){};
						      \node (yl) at (0,\y){};
						      \node (xl) at (\x*2,0){};
						      \draw[dashed] node[left=1mm of yl]{\y} (yl) -- (n) -- (\x*2,0) node[below=1mm of xl]{\x};
					      }
			      \end{tikzpicture}
		      \end{scaletikzpicturetowidth}
	      \end{center}
	      Se dice que \(c_n\) tiende a infinito (sin especificar el signo) o que oscila infinitamente.
\end{itemize}
\subsection{Propiedades del límite}
La mayoría de las veces, el problema consistirá en calcular el valor de \(\lim_{n\to\infty}a_n\).
La definición no será útil para ello porque presupone conocer el valor de \(L\), de modo que nos valdremos de propiedades y diversos recursos algebraicos para poder determinar el valor del límite en los ejemplos que estudiemos.
La definición de límite es imprescindible para poder obtener esas propiedades y para introducir casi todos los conceptos de la materia que se basan en esta noción.
En la práctica no haremos un uso directo de dicha definición.

Las siguientes propiedades se deducen de la definición de límite y nos servirán para desarrollar técnicas que nos permitan calcular algunos límites.
\subsubsection{Unicidad del límite}
\begin{center}
	\begin{scaletikzpicturetowidth}{.5\linewidth}
		\begin{tikzpicture}[scale=\tikzscale]
			\draw[->,thick] (0,-1) -- (0,5);
			\draw[->,thick] (-1,0) -- (10,0);
			\foreach \x/\xl in {1/1,2/2,3/3,4/4,6/\(n\)} {
					\draw[thick] (\x,.25) -- (\x,-.25) node[below]{\xl};
				}
			\draw[dashed, red] (6,0) -- (6,5);
			\foreach \y [count=\i] in {1.5,3} {
					\draw[thick] (0,\y+.5) -- (10,\y+.5);
					\draw[dashed, red] (0,\y) node[left]{\(L_\i\)} -- (10,\y);
					\draw[thick] (0,\y-.5) -- (10,\y-.5);
				}
		\end{tikzpicture}
	\end{scaletikzpicturetowidth}
\end{center}
Una sucesión no puede converger a dos límites distintos.
Si así fuera todos los \(a_n\) a partir de \(n=n_0\) tendrían que estar simultáneamente en las dos franjas y eso no es posible.
\subsubsection{Acotación de las sucesiones convergentes}
Si \(a_n\) es convergente, entonces el conjunto \(A=\{a_n:n\in\N\}\) es acotado.
\begin{center}
	\begin{scaletikzpicturetowidth}{.5\linewidth}
		\begin{tikzpicture}[scale=\tikzscale]
			\draw[->,thick] (0,-1) -- (0,7);
			\draw[->,thick] (-1,0) -- (11,0);
			\draw[thick] (5.5,.25) -- (5.5,-.25) node[below]{\(n_0\)};
			\draw[dashed] (5.5,0) -- (5.5,7);
			\draw[red] (0,1) node[left]{\(m\)} -- (11,1);
			\draw[dashed] (0,2) node[left]{\(L-1\)} -- (11,2);
			\draw[dashed] (0,3) node[left]{\(L\)} -- (11,3);
			\draw[dashed] (0,4) node[left]{\(L+1\)} -- (11,4);
			\draw[red] (0,5) node[left]{\(M\)} -- (11,5);
			\fill[green,opacity=0.2] (5.5,2) -- (5.5,4) -- (11,4) -- (11,2) -- cycle;
			\foreach \y [count=\x] in {4.5,2.2,1.2,3.5,4.6,3.8,2.6,3.3,2.7,3.2} {
					\node[draw,shape=circle,fill=black,scale=0.4] at (\x,\y){};
				}
		\end{tikzpicture}
	\end{scaletikzpicturetowidth}
\end{center}
Se elige \(\epsilon=1\) o cualquier otro.
Una vez que \(a_n\) queda dentro de la franja roja para \(n\geq n_0\), es fácil encontrar cotas superiores e inferiores de \(A\).
\(M\) es cota superior y \(m\) es cota inferior de \(A\).
\subsubsection{Conservación de signo}
Si \(a_n\) converge a un límite \(L\) mayor que cero, entonces la sucesión \(a_n\) es mayor que cero para casi todo \(n\).
Es decir:
\[\text{Si }\lim_{n\to\infty}a_n=L>0\text{ entonces }a_n>0\,pctn\]
\begin{center}
	\begin{scaletikzpicturetowidth}{.5\linewidth}
		\begin{tikzpicture}[scale=\tikzscale]
			\draw[->,thick] (0,-1) -- (0,7);
			\draw[->,thick] (-1,0) -- (11,0);
			\draw[thick] (5.5,.25) -- (5.5,-.25) node[below]{\(n_0\)};
			\draw[dashed] (5.5,0) -- (5.5,7);
			\draw[dashed] (0,2) node[left]{\(L-L/2\)} -- (11,2);
			\draw[dashed] (0,3) node[left]{\(L\)} -- (11,3);
			\draw[dashed] (0,4) node[left]{\(L+L/2\)} -- (11,4);
			\fill[green,opacity=0.2] (5.5,2) -- (5.5,4) -- (11,4) -- (11,2) -- cycle;
			\foreach \y [count=\x] in {4.5,-.8,1.2,3.5,-.4,3.8,2.6,3.3,2.7,3.2} {
					\node[draw,shape=circle,fill=black,scale=0.4] at (\x,\y){};
				}
		\end{tikzpicture}
	\end{scaletikzpicturetowidth}
\end{center}
Se elige \(\epsilon=L/2\), de modo que \(L-L/2=L/2>0\).
Esto asegura que la franja verde esté por encime del eje de las \(x\).
Así \(a_n>0\) si \(n\geq n_0\).
\subsubsection{Álgebra de límites}
Si \(a_n\rightarrow a\in\R\) y \(b_n\rightarrow b\in\R\), entonces
\begin{itemize}
	\item \(a_n+b_n\rightarrow a+b\)
	\item \(a_n\cdot b_n\rightarrow ab\).
	      En particular \(k\cdot a_n\rightarrow ka\) si \(k\in\R\)
	\item Si \(b\neq0\) entonces \(\frac{a_n}{b_n}\rightarrow\frac{a}{b}\)
	\item \(|a_n|\rightarrow|a|\)
	\item Si \(a>0\) entonces \((a_n)^{b_n}\rightarrow a^b\)
\end{itemize}
Ejemplo.
Calcular \(\lim_{n\to\infty}=\left(\frac{1}{n}+\frac{2n}{n+1}\right)^3\).

Tenemos que \[\lim_{n\to\infty}\frac{1}{n}=0\] \[\lim_{n\to\infty}\frac{2n}{n+1}=2\cdot\lim_{n\to\infty}\frac{n}{n+1}=2\cdot1=2\]
Entonces:
\[\lim_{n\to\infty}=\left(\frac{1}{n}+\frac{2n}{n+1}\right)^3=(0+2)^3=8\]
\subsubsection{Intederminaciones}
El álgebra de límites requiere que las sucesiones involucradas sean convergentes a un número real.
Por ello, no podemos aplicar el álgebra de límites en forma directa en, por ejemplo \(\lim_{a\to\infty}\frac{3n^2+2}{2n^2+5n}\), ya que un primer análisis de la sucesión nos dice que tanto numerador como denominador tienden a más infinito y el teorema de álgebra de límites se refiere a valores numéricos del límite.

Se suele decir que estamos en presencia de una \textit{indeterminación} en este caso, del tipo \(\frac{\infty}{\infty}\) entendiendo este símbolo como el cociente de sucesiones que tienden ambas a infinito.
El nombre de indeterminación es porque no hay una propiedad general que nos indique el valor del límite en una situación como esta.

Es necesario, en cada caso, aplicar alguna técnica algebraica que permita “salvar” la indeterminación y calcular el límite.
No es el único tipo de indeterminación con el que nos vamos a encontrar.
Analicemos los siguientes ejemplos:
\begin{itemize}
	\item \(a_n=\frac{1}{n}\rightarrow0,b_n=n\rightarrow+\infty\):\tab\(a_nb_n=\frac{1}{n}n=1\rightarrow1\)
	\item \(a_n=\frac{1}{n}\rightarrow0,b_n=n^2\rightarrow+\infty\):\tab\(a_nb_n=\frac{1}{n}n^2=n\rightarrow+\infty\)
	\item \(a_n=\frac{1}{n^2}\rightarrow0,b_n=n\rightarrow+\infty\):\tab\(a_nb_n=\frac{1}{n^2}n=\frac{1}{n}\rightarrow0\)
\end{itemize}
En estos observamos que no existe una propiedad que pueda predecir sobre un límite del tipo \(0\cdot\infty\).

Los límites de los siguientes “tipos”, aunque no son todos, constituyen indeterminaciones:
\begin{itemize}
	\item \(\frac{\infty}{\infty}\)
	\item \(+\infty-\infty\)
	\item \(0\cdot\infty\)
	\item \(\frac{0}{0}\)
	\item \((+\infty)^0\)
	\item \(0^0\)
\end{itemize}
En todos los casos, hay que entender estos símbolos como el límite de la operación aritmética indicada en cada caso, entre dos sucesiones.

Como vimos, el álgebra de límites requiere que las sucesiones involucradas sean convergentes a un número real.
Cuando esto no ocurre, a veces se presentan indeterminaciones.
En cada caso hay que usar algún recurso algebraico que permita salvar la indeterminación y calcular el valor del límite.

En ocasiones no es posible aplicar el álgebra de límites porque los límites involucrados no son finitos, sin embargo no estamos ante una indeterminación.
A continuación damos algunas situaciones donde se puede saber el límite a pesar de que los límites involucrados no sean todos números reales.
\begin{itemize}
	\item \((+\infty)\cdot L=+\infty\) si \(L>0\)\tab Ej:
	      \(\lim_{n\to\infty}\sqrt{n}\left(\frac{n}{3n+1}\right)=+\infty\)
	\item \((+\infty)+\infty=+\infty\)\tab Ej:
	      \(\lim_{n\to\infty}\frac{n^2}{n+1}+\sqrt{n}=+\infty\)
	\item \((+\infty)+\) oscila finitamente \(=+\infty\)\tab Ej:
	      \(\lim_{n\to\infty}(n+\cos(n))=+\infty\)
	\item \(\frac{0}{\infty}=0\)\tab Ej:
	      \(\lim_{n\to\infty}\frac{\frac{1}{n}}{n^2+1}=0\)
	\item \((0)^{+\infty}=0\)\tab Ej:
	      \(\lim_{n\to\infty}\left(\frac{n}{n^2+1}\right)^n=0\)
\end{itemize}
Cada una de estas afirmaciones se puede demostrar a partir de la definición de límite.
\subsubsection{``Cero por acotado''}
Si bien los límites del tipo \(0\cdot\infty\) resultan ser una indeterminación y, por lo tanto, nada podemos decir a sobre el valor del límite sin salvar tal indeterminación, sí se puede decir algo cuando estamos en presencia de un producto de una sucesión que tiende a cero por otra que está acotada.
En estos casos se obtiene una sucesión que tiende a cero.
Es decir \[\text{Si }a_n\rightarrow0\text{ y }|b_n|\leq K\text{ entonces }a_nb_n\rightarrow0\]
Ejemplo: Calcular el \(\lim_{n\to\infty}\left(1-\frac{n}{n+1}\right)(-1)^{n+1}\)
La expresión \((-1)^{n+1}\) vale 1 o \(-1\) según la paridad de \(n\).
En particular está acotada: \(|(-1)^{n+1}|\leq1\).
Por otra parte, \(\lim_{n\to\infty}\left(1-\frac{n}{n+1}\right)=\lim_{n\to\infty}\left(\frac{1}{n+1}\right)=0\).
Usando la propiedad ``cero por acotado'' se obtiene
\[\lim_{n\to\infty}\left(1-\frac{n}{n+1}\right)(-1)^{n+1}=0\]
\subsubsection{Propiedad del sándwich}
Si dos sucesiones convergen a un mismo límite \(L\), entonces, cualquier sucesión comprendida entre ambas, también converge a \(L\).
\[\text{Si }b_n<a_n<c_n\text{ entonces }\lim_{n\to\infty}b_n\leq\lim_{n\to\infty}a_n\leq\lim_{n\to\infty}c_n\]
Ejemplo: Calcular el \(\lim_{n\to\infty}\left(3+\frac{1}{n}\cos(3n+1)\right)\)

El coseno es una función que toma valores entre \(-1\) y 1.
Entonces vale
\[-1\leq\cos(3n+1)\leq1\]
\[-\frac{1}{n}\leq\frac{1}{n}\cos(3n+1)\leq\frac{1}{n}\]
\[3-\frac{1}{n}\leq3+\frac{1}{n}\cos(3n+1)\leq3+\frac{1}{n}\]
Ahora bien: \(\lim_{n\to\infty}\left(3-\frac{1}{n}\right)=\lim_{n\to\infty}\left(3+\frac{1}{n}\right)=3\).
Entonces \[\lim_{n\to\infty}\left(3+\frac{1}{n}\cos(3n+1)\right)=3\]
Otra forma de resolverlo es
\[\lim_{n\to\infty}\left(3+\frac{1}{n}\cos(3n+1)\right)=\]
\[3+\lim_{n\to\infty}\left(\underbrace{\frac{1}{n}}_{\text{Tiende a cero}}\underbrace{\cos(3n+1)}_{\text{Está acotado}}\right)=3\]
\subsubsection{Sándwich en el infinito}
La propiedad del sándwich se puede generalizar de la siguiente forma.
\[\text{Si }a_n>b_n\text{ y }b_n\rightarrow+\infty\text{ entonces }a_n\rightarrow+\infty\]
Ejemplo: Calcular el \(\lim_{n\to\infty}a_n\) sabiendo que \(a_n>\frac{3n^2+1}{100n+5}\) para todo \(n\).

Calculamos el \(\lim_{n\to\infty}\frac{3n^2+1}{100n+5}\).
Para ello, dividimos por \(n\) numerador y denominador:
\[\lim_{n\to\infty}\frac{3n^2+1}{100n+5}=\lim_{n\to\infty}\frac{3n+\frac{1}{n}}{100+\frac{5}{n}}=+\infty\]
Entonces \(\lim_{n\to\infty}a_n=+\infty\).
\subsection{Sucesiones monótonas}
Las sucesiones son funciones que tienen por dominio a los números naturales.
Estudiaremos aquellas sucesiones que son funciones crecientes o decrecientes de su variable natural.
Es decir:
\[a_{n+1}\leq a_n \text{para (casi) todo \(n\). En tal caso será \textit{decreciente} o}\]
\[a_{n+1}\geq a_n \text{para (casi) todo \(n\). En tal caso será \textit{creciente}.}\]
En ambos casos decimos que se trata de una \textit{sucesión monótona}.

La importancia de las sucesiones monótonas radica en que siempre tienen límite, ya sea éste finito o infinito.
Antes de enunciar con precisión este resultado, hacemos una observación que será de utilidad en lo que sigue.

Si la sucesión es de términos positivos se tiene que:
\[\frac{a_{n+1}}{a_n}\leq1\text{ equivale a \(a_n\) decreciente.}\]
\[\frac{a_{n+1}}{a_n}\geq1\text{ equivale a \(a_n\) creciente.}\]
Ejemplo.
Determinemos si las sucesiones \(a_n=(0,8)^n\) y \(b_n=(-0,8)^n\) son monótonas.

La sucesión \(a_n=(0,8)^n\) tiene todos sus términos positivos.
Sus primeros términos son: \(a_1=0,8, a_2=0,64, a_3=0,512\dots\) Aparentemente es decreciente, pero no alacanza con visualizar tres términos para concluir que es decreciente.
Sirve para hacer una conjetura.
Para demostrar que la misma es efectivamente cierta usamos la observación precendente
\[\frac{a_{n+1}}{a_n}=\frac{(0,8)^{n+1}}{(0,8)^n}=(0,8)^{n+1-n}=0,8<1\]
Entonces, como \(\frac{a_{n+1}}{a_n}<1\), la sucesión resulta decreciente.

En el otro caso, la sucesión no es de términos positivos pues \(b_n=(-0,8)^n\) va cambiando de signo según sea \(n\) par o impar.
Los primeros términos de \(b_n\) son \(b_1=-0,8, b_2=0,64, b_3=-0,512\dots\) Se observa que \(b_n<0\) si \(n\) es impar y que \(b_n>0\) sin \(n\) es par.
Podemos concluir entonces que la sucesión \(b_n\) no es monótona.
Este ejemplo nos muestra que no todas las sucesiones son monótonas y que éstas constituyen una clase particular de sucesiones.
El siguiente teorema nos dice cómo se comporta el límite de una sucesión monótona.
\subsubsection{Teorema sobre sucesiones monótonas}
Si \(a_n\) es una sucesión creciente puede ser que el conjunto \(A=\{a_n:n\in\N\}\) esté acotado superiormente o que no lo esté.
Para cada uno de estos dos casos se tiene el siguiente teorema.
\begin{itemize}
	\item Si \(A=\{a_n:n\in\N\}\) está acotado superiormente, entonces existe \(\lim_{n\to\infty}a_n=L\in\R\).
	\item Si \(A=\{a_n:n\in\N\}\) no está acotado superiormente, entonces existe \(\lim_{n\to\infty}a_n=+\infty\).
\end{itemize}
De forma análoga, hay una versión del teorema cambiando \textit{creciente} por \textit{decreciente}, \textit{acotada superiormente} por \textit{acotada inferiormente} y \(\lim_{n\to\infty}a_n=+\infty\) por \(\lim_{n\to\infty}a_n=-\infty\).

En otras palabras, el teorema dice que la sucesión no puede oscilar:
o tiene límite finito o tiende a infinito (más o menos según sea creciente o decreciente).

El teorema es de los llamados \textit{teoremas de existencia}, esto es, asegura que el límite existe (en el caso de acotación) pero no dice cuánto vale.

\subsubsection{Algunos ejemplos importantes}
Estudiaremos algunos ejemplos importantes de sucesiones, no sólo por los ejemplos en sí, sino por las técnicas usadas para calcular sus límites.
Haremos uso del teorema recién enunciado sobre sucesiones monótonas.
\begin{itemize}
	\item \(a_n=r^n, 0<r<1\)
\end{itemize}
Si experimentamos con algún caso particular (\(r=\frac{1}{2},\,a_n=\frac{1}{2^n}\), por ejemplo) nos podemos convencer de que la sucesión tiende a 0 ya que \(2^n\) crece a más infinito.
Veamos cómo este convencimiento se puede plasmar en una demostración.

La sucesión es de términos positivos.
Estudiamos el cociente \(\frac{a_{n+1}}{a_n}\) a los efectos de compararlo, se tiene que
\[\frac{a_{n+1}}{a_n}=\frac{r^{n+1}}{r^n}=r<1\text{ para todo }n\]
Entonces, la sucesión es decreciente.
Además, dijimos que es de términos positivos, por lo que \(0<a_n\), es decir, está acotada inferiormente.

El teorema nos dice entonces que existe el límite \(\lim_{n\to\infty}a_n=L\geq0\).
Apostamos a (“conjeturamos” es más apropiado) que \(L=0\).

Pues bien, veamos qué pasa si fuera \(L>0\).
En tal caso, podemos aplicar el teorema de álgebra de límites al cociente \(\frac{a_{n+1}}{a_n}\) y obtenemos la siguiente contradicción:

Por un lado:
\[\lim_{n\to\infty}\frac{a_{n+1}}{a_n}=\frac{\lim_{n\to\infty}a_{n+1}}{\lim_{n\to\infty}a_n}=\frac{L}{L}=1\]
Por otro lado:
\[\lim_{n\to\infty}\frac{a_{n+1}}{a_n}=\lim_{n\to\infty}r=r\]
Como el límite es único, debe ser \(1=r\).
Pero \(r<1\).
¡Contradicción!
Luego no queda otra que \(L=0\).
Es decir
\[0<r<1\Rightarrow\lim_{n\to\infty}r^n=0\]
Con una cuenta similar se obtiene que si
\[r>1\Rightarrow\lim_{n\to\infty}r^n=+\infty\]
\begin{itemize}
	\item \(b_n=\frac{r^n}{n}, r>1\)
\end{itemize}
Otra vez tenemos una sucesión de términos positivos.
Estamos ante una indeterminación del tipo \(\frac{\infty}{\infty}\).
Si tuviéramos que conjeturar un resultado, habría que decidir quién va ``más rápido'' a más infinito, ¿el numerador o el denominador?
Si fuera \(r=2\), los primeros términos serían:
\(2,2,\frac{8}{3},4,\frac{32}{5},\dots\)
Aparentemente va creciendo y nada la detiene.

Calculamos el cociente \(\frac{b_{n+1}}{b_n}\) y lo comparamos con 1 como en el ejemplo anterior.
\[\frac{b_{n+1}}{b_n}=\frac{\frac{r^{n+1}}{n+1}}{\frac{r^n}{n}}=r\frac{n}{n+1}\]
Observamos que:
\begin{enumerate}
	\item \(r>1\) (Es dato)
	\item \(\frac{n}{n+1}=\frac{1}{1+\frac{1}{n}}\rightarrow1\)
	\item El producto \(r\frac{n}{n+1}>1\) para casi todo \(n\)
\end{enumerate}
Entonces la sucesión \(b_n=\frac{r^n}{n}\) es creciente.
El teorema de las sucesiones monótonas nos dice que si está acotada superiormente tiene límite finito, caso contrario, tiende a más infinito.

Supongamos que esté acotado superiormente.
En tal caso \(\lim_{n\to\infty}b_n=L>0\) y se puede usar el álgebra de límites y obtener:

Por un lado, \[\lim_{n\to\infty}\frac{b_{n+1}}{b_n}=\frac{L}{L}=1\]
y, por otro lado, \[\lim_{n\to\infty}\frac{b_{n+1}}{b_n}=\lim_{n\to\infty}r\frac{n}{n+1}=r\]
Se llega a la contradicción: \(r=1\) cuando teníamos de movida que \(r>1\).
Esta contradicción proviene de suponer que la sucesión creciente \(b_n=\frac{r^n}{n}\) está acotada superiormente.
Por lo tanto \(b_n=\frac{r^n}{n}\) \textit{no está acotada superiormente}.
El teorema nos dice entonces que \[r>1\Rightarrow\lim_{n\to\infty}\frac{r^n}{n}=+\infty\]
Con una cuenta similar se obtiene que \[0<r<1\Rightarrow\lim_{n\to\infty}nr^n=0\]
Con una demostración análoga a la precedente, vale que
\(r>1\Rightarrow\lim_{n\to\infty}\frac{r^n}{n^k}=+\infty\) y que \(0<r<1\Rightarrow\lim_{n\to\infty}n^kr^n=0\) cualquiera sea \(k\in\N\).
\begin{itemize}
	\item \(c_n=\sqrt[n]{n}\)
\end{itemize}
Nos apoyaremos en que \(\lim_{n\to\infty}\frac{r^n}{n}=+\infty\) (\(r>1\)).
Observemos en primer lugar que \[c_n=\sqrt[n]{n}\geq1\text{ para todo }n\]
Vamos a probar que \[\lim_{n\to\infty}\sqrt[n]{n}=1\]
Cualquiera sea \(\epsilon>0\), basta probar que \[1\leq\sqrt[n]{n}<1+\epsilon\text{ para casi todo }n\]
Llamamos \(r=1+\epsilon>1\) y la desigualdad a probar es equivalente a probar que \[n<(1+\epsilon)^n=r^n\text{ para casi todo }n\]
Es decir \(\frac{r^n}{n}>1\) para casi todo \(n\).
Pero sabemos que \(\frac{r^n}{n}\rightarrow+\infty\) pues \(r>1\).
Entonces, es seguro que \(\frac{r^n}{n}>1\) para casi todo \(n\).
Luego \[\lim_{n\to\infty}\sqrt[n]{n}=1\]
\subsubsection{El número \textit{e}}
\begin{itemize}
	\item \(e_n=\left(1+\frac{1}{n}\right)^n\)
\end{itemize}
Esta sucesión ejemplifica un nuevo tipo de indeterminación \(1^\infty\), siempre entendiendo este símbolo como una sucesión que tiende a 1 elevada a una sucesión que tiende a infinito.

Sin pretender dar una demostración, mostraremos que \(e_n\) es creciente \((e_{n+1}>e_n)\) y acotada \(e_n<K\).
Aceptados estos dos hechos, el teorema de sucesiones monótonas nos asegurará que la sucesión converja a un límite finito.

Para visualizar que \(e_n\) es creciente y acotada recurriremos al gráfico de la función \(f(x)=\log(1+x)\) para valores positivos de la variable \(x\).
Más precisamente, fijamos la atención en \[x_1=\frac{1}{n+1}\text{ y }x_2=\frac{1}{n}\]
\begin{center}
	\begin{scaletikzpicturetowidth}{\linewidth}
		\begin{tikzpicture}[scale=\tikzscale]
			\draw[->,thick] (0,-.1) -- (0,1.1);
			\draw[->,thick] (-.1,0) -- (2.1,0);
			\foreach \x/\xl/\y/\yl/\c in {.7/\(\frac{1}{n+1}\)/.45/\(f(\frac{1}{n+1})\)/green,1.4/\(\frac{1}{n}\)/.75/\(f(\frac{1}{n})\)/blue} {
					\draw[\c,very thick] (0,0) -- (\x,\y);
					\node[draw,shape=circle,fill=black,scale=0.4] at (\x,\y)(n){};
					\node (yl) at (0,\y){};
					\node (xl) at (\x,0){};
					\draw[dashed] node[left=1mm of yl]{\(\scriptstyle\text{\yl}\)} (yl) -- (n) -- (\x,0) node[below=1mm of xl]{\xl};
				}
			\draw[very thick,domain=0:2, smooth, variable=\x] plot ({\x}, {2*Log(10,1+\x)});
			\draw[red,very thick] (0,0) -- (1,.9);
		\end{tikzpicture}
	\end{scaletikzpicturetowidth}
\end{center}
Se observa que la recta verde tiene pendiente mayor que la recta azul.
Estas dos rectas pasan por el origen y lo unen con los puntos del gráfico \((x_1,f(x_1))\) y \((x_2,f(x_2))\) respectivamente.
Esta observación se traduce en la desigualdad
\[\frac{\log\left(1+\frac{1}{n+1}\right)}{\frac{1}{n+1}}>\frac{\log\left(1+\frac{1}{n}\right)}{\frac{1}{n}}\]
\[(n+1)\log\left(1+\frac{1}{n+1}\right)>n\log\left(1+\frac{1}{n}\right)\]
\[\log\left[\left(1+\frac{1}{n+1}\right)^{n+1}\right]>\log\left[\left(1+\frac{1}{n}\right)^n\right]\]
Como el logaritmo es una función creciente la desigualdad vale para las expresiones que están entre corchetes.
Es decir \[\left(1+\frac{1}{n+1}\right)^{n+1}>\left(1+\frac{1}{n}\right)^n\]
Esto muestra que la sucesión \(e_n\) es creciente.
Para “ver” que además está acotada, también apelamos al gráfico de \(f(x)=\log(1+x)\) y observamos que las pendientes de las rectas de colores (sean azules o verdes) son todas menores que la pendiente de la recta roja.

Si tal pendiente es \(m\) esta observación se traduce en la desigualdad
\[\frac{\log\left(1+\frac{1}{n}\right)}{\frac{1}{n}}<m\]
\[\log\left[\left(1+\frac{1}{n}\right)^n\right]<m\]
\[\left(1+\frac{1}{n}\right)^n<10^m=K\]
El teorema de las sucesiones monótonas, nos asegura que existe \[\lim_{n\to\infty}\left(1+\frac{1}{n}\right)^n=e\]
El número \(e\), valor límite de la sucesión \(e_n\) es un número muy importante de la matemática que aparece en diversas situaciones.
Es un número irracional que está entre 2 y 3.
Más precisamente, su expresión decimal aproximada es \(e\cong2,718281\dots\)

Conocer el límite de \(\left(1+\frac{1}{n}\right)^n\) nos permite calcular el límite de otras sucesiones con la misma “pinta”.

Ejemplo.
Calcular el \(\lim_{n\to\infty}\left(1+\frac{2}{n+1}\right)^{\frac{n+1}{2}}\)

Estamos ante una indeterminación del tipo \(1^\infty\).
Observemos, además, que si llamamos \(a_n=\frac{n+1}{2}\), podemos escribir el límite a calcular como
\[\lim_{n\to\infty}\left(1+\frac{1}{a_n}\right)^{a_n}\text{ donde }\lim_{n\to\infty}a_n=+\infty\]
Este límite, con un tratamiento similar al realizado para el caso \(a_n=n\), tiende al número real \(e\).
De modo que, en general vale
\[\lim_{n\to\infty}\left(1+\frac{1}{a_n}\right)^{a_n}=e\text{ si }\lim_{n\to\infty}a_n=+\infty\]
Es equivalente decir, cambiando \(b_n=\frac{1}{a_n}\) que
\[\lim_{n\to\infty}\left(1+b_n\right)^{\frac{1}{b_n}}=e\text{ si }\lim_{n\to\infty}b_n=+\infty\]
Cuando estemos ante una indeterminación del tipo \(1^\infty\), la estrategia será “llevar” por medio de transformaciones algebraicas, el límite a calcular a una de estas situaciones.

Ejemplo.
Calcular el \(\lim_{n\to\infty}\left(1+\frac{2}{n+2}\right)^{n-1}\)

Es una indeterminación del tipo \(1^\infty\).
Utilicemos la estrategia recién propuesta.
\[\left(1+\frac{2}{n+1}\right)^{n-1}=\]
Preparamos la base como \(\left(1+\frac{1}{\text{algo}}\right)\).
El exponente queda igual.
\[\left(1+\frac{1}{\frac{n+1}{2}}\right)^{n-1}=\]
Hacemos ``aparecer'' en el exponente ``algo'' para poder decir que el corchete tiende a \(e\)
\[\left[\left(1+\frac{1}{\frac{n+1}{2}}\right)^{\frac{n+1}{2}}\right]^{\frac{2}{n+1}(n-1)}\]
Para compensar que hicimos aparecer en el exponente ``algo'' ponemos \(\frac{1}{\text{algo}}\) y mantenemos lo que ya estaba.

Si bien la expresión que quedó tiene un aspecto temible, si la miramos con optimismo, podemos ver que lo que está entre corchetes es del tipo
\[\lim_{n\to\infty}\left(1+\frac{1}{a_n}\right)^{a_n}\text{ con }\lim_{n\to\infty}a_n=+\infty\]
Sabemos en estos casos que
\[\left(1+\frac{1}{\frac{n+1}{2}}\right)^{\frac{n+1}{2}}\rightarrow e\]
Por otra parte, si concentramos la atención en lo que quedó en el exponente por fuera del corchete, vemos que tenemos una sucesión que sabemos atacar.
\[\frac{2}{n+1}(n-1)=\frac{2n-2}{n+1}=\frac{2-\frac{2}{n}}{1+\frac{1}{n}}\rightarrow2\]
De modo que podemos usar álgebra de límite:
la base (lo que está entre corchetes) tiende al número real \(e\) y el exponente tiende a 2, entonces
\[\lim_{n\to\infty}\left(1+\frac{2}{n+2}\right)^{n-1}=e^2\]
\subsection{El Criterio de Cauchy o de la raíz enésima}
Hemos visto que \(r^n\) tiende a 0 si \(r\) está entre 0 y 1 y tiende a más infinito si \(r\) es mayor que 1.
Se puede extender fácilmente este resultado para valores negativos de \(r\) diciendo que
\[r^n\rightarrow0\text{ si }-1<r<1\]
\[|r^n|\rightarrow+\infty\text{ si }|r|>1\]
Este ejemplo sirve para dar un criterio que será muy útil para el cálculo de límites.
Se quiere calcular el \(\lim_{n\to\infty}a_n\).

Criterio de Cauchy: si \(\lim_{n\to\infty}\sqrt[n]{|a_n|}=L\) vale que:
\begin{itemize}
	\item Si \(0\leq L<1\) entonces \(\lim_{n\to\infty}a_n=0\).
	\item Si \(L>1\) o es más infinito entonces \(\lim_{n\to\infty}|a_n|=+\infty\).
	\item Si \(L=1\) entonces el criterio no sirve para decidir el \(\lim_{n\to\infty}a_n\).
\end{itemize}
\subsection{El Criterio de D ́Alembert o del cociente}
Se basa en la idea que usamos en los primeros ejemplos donde estudiamos el cociente \(\frac{a_{n+1}}{a_n}\) y lo comparábamos con 1.

Se quiere calcular el \(\lim_{n\to\infty}a_n\).

Criterio de D ́Alembert: si \(\lim_{n\to\infty}\left|\frac{a_{n+1}}{a_n}\right|=L\) vale que:
\begin{itemize}
	\item Si \(0\leq L<1\) entonces \(\lim_{n\to\infty}a_n=0\).
	\item Si \(L>1\) o es más infinito entonces \(\lim_{n\to\infty}|a_n|=+\infty\).
	\item Si \(L=1\) entonces el criterio no sirve para decidir el \(\lim_{n\to\infty}a_n\).
\end{itemize}
\subsection{Subsucesiones}
Hemos visto que algunas sucesiones carecen de límite finito o infinito.
Es el caso de las sucesiones que oscilan finitamente o infinitamente.
Comprobar que una sucesión no tiene límite en forma rigurosa puede resultar difícil con sólo la definición de límite ya que hay que descartar \textit{todo posible} candidato a ser el límite de la sucesión.

Para resolver este problema será útil introducir la idea de \textit{subsucesión}.

Consideremos una sucesión de números reales:
\[a_1,a_2,a_3,a_4,a_5,\dots,a_n,\dots\]
Con dicha sucesión se puede realizar de muchas maneras la siguiente construcción:
se suprimen de la sucesión una cantidad finita o infinita de términos de manera que queden infinitos términos.
Los que quedan forman una nueva sucesión que volvemos a numerar.
Por ejemplo:

Si sacamos el primer término nos queda la nueva sucesión:
\[a_2,a_3,a_4,a_5,\dots,a_n,\dots\]
que volvemos a numerar \(b_1,b_2,b_3,b_4,b_5,\dots,b_n,\dots\) de modo que \(b_n=a_{n+1}\).
Esta nueva sucesión resulta ser una \textit{subsucesión} de la primera.

Si, en cambio, sacamos los infinitos términos impares, nos quedan los infinitos términos pares \(a_2,a_4,a_6,a_8,a_{10},\dots,a_{2n},\dots\) de modo que si volvemos a numerarla \(b_1,b_2,b_3,b_4,b_5,\dots,b_n,\dots\) resulta ser \(b_n=a_{2n}\).
Como antes, se dice que \(a_{2n}\) es una \textit{subsucesión} de \(a_n\).

Una \textit{subsucesión} de \(a_n\) es una sucesión \(b_k=a_{ n_k}\) donde \(n_1<n_2<n_3<n_4<n_5<n_6<\dots<n_k<\dots\) es la nueva enumeración.

Dada la sucesión \(a_n\) cuyos primeros términos son \(2,4,6,4,2,4,6,4,2,\dots\) escribir el término general de \(a_{2n}\) y de \(a_{2n+1}\).
Determinen si alguna de las dos subsucesiones es convergente.

\(a_{2n}\) es la subsucesión de los términos pares.
Los resaltamos en negrita para poder visualizarlos:
\begin{center}
	2, \textbf{4}, 6, \textbf{4}, 2, \textbf{4}, 6, \textbf{4}, 2,
\end{center}
Claramente se observa que \(a_{2n}=4\) para todo \(n\).
De modo que, al ser una sucesión constante, resulta convergente.

\(a_{2n+1}\) es la subsucesión de los términos impares.
De vuelta en negrita.
\begin{center}
	\textbf{2}, 4, \textbf{6}, 4, \textbf{2}, 4, \textbf{6},
\end{center}
Vemos que esta subsucesión se obtiene sacando todos los 4 de la sucesión.
Si hacemos eso queda \(2,6,2,6,2,\dots\) de modo que
\[b_n=a_{2n+1}=\begin{cases}
		2 & \text{si \(n\) impar} \\
		6 & \text{si \(n\) par}
	\end{cases}\]
que no resulta convergente.
Pero, ¿cómo probar que no es convergente?

El siguiente resultado, que se deduce directamente de la definición de límite, vendrá en nuestra ayuda.
Sea \(a_n\) una sucesión de números reales.
Entonces \(a_n\rightarrow L\) sí y sólo si toda subsucesión \(b_k=a_{n_k}\) de \(a_n\) converge a \(L\) (\(L\) puede ser finito o infinito).

El cuantificador “toda subsucesión” la hace poco práctica para usarla para calcular límites.
Pero alcanza con que dos subsucesiones tiendan a límites diferentes para que la sucesión original no sea convergente.

Probemos que la sucesión \(b_n=\begin{cases}
	2 & \text{si \(n\) impar} \\
	6 & \text{si \(n\) par}
\end{cases}\) no tiene límite.\\
Consideremos la subsucesión de los términos pares y la subsucesión de los impares:
\[2,6,2,6,\dots\]
\[b_{2n}=6\text{ y }b_{2n+1}=2\]
Es inmediato que \(b_{2n}\rightarrow6\) y que \(b_{2n+1}\rightarrow2\).
Como estos límites son distintos, se concluye que la sucesión \(b_n\) no tiene límite.
\subsection{Sucesiones dadas en forma recurrente}
Hasta ahora hemos tratado cada sucesión por medio de su término general.

Sin embargo, en muchas situaciones vinculadas con las aplicaciones y procesos iterativos, las sucesiones sepresentan en forma recurrente.
Esto es, se define el primer término \(a_1\), luego del mismo surge \(a_2\) y en general, se define \(a_{n+1}\) a partir del término anterior \(a_n\) o más generalmente, a partir de todos los términos anteriores.

Ejemplo.
Estudiemos la convergencia de la sucesión definida como \(a_1=5,\frac{a_{n+1}}{a_n}=\frac{n+1}{3n}\) para todo \(n>1\).

La sucesión viene servida para aplicar el criterio del cociente:
\[\lim_{n\to\infty}\frac{a_{n+1}}{a_n}=\lim_{n\to\infty}\frac{n+1}{3n}=\frac{1}{3}<1\]
Entonces
\[\lim_{n\to\infty}a_n=0\]
Ejemplo.
Sea \(a_n\) la sucesión definida en forma recurrente por \(a_1=1,a_{n+1}=\frac{n^n+3^n}{n!}a_n\).
Calcular, si existe, el \(\lim_{n\to\infty}2^{1+\frac{1}{a_n}}\)

La sucesión es de términos positivos.
En primer lugar calculamos el \(\lim_{n\to\infty}a_n\).
Para ello usaremos el Criterio del cociente o Criterio de D ́Alembert.

Aprovechamos la forma recurrente en que viene definida la sucesión:
\[\frac{a_{n+1}}{a_n}=\frac{n^n+3^n}{n!}=\frac{n^n}{n!}+\frac{3^n}{n!}\]
Estudiamos cada término por separado, usando otra vez, el criterio del cociente:
\(x_n=\frac{n^n}{n!}\).
El cociente de D ́Alembert es
\[\frac{x_{n+1}}{x_n}=\frac{\frac{(n+1)^{n+1}}{(n+1)!}}{\frac{n^n}{n!}}=\frac{(n+1)^{n+1}}{(n+1)!}\cdot\frac{n!}{n^n}\]
Trabajamos un poco esta última expresión
\[\frac{x_{n+1}}{x_n}=\frac{(n+1)^{n+1}}{(n+1)!}\cdot\frac{n!}{n^n}=\frac{(n+1)(n+1)^n}{(n+1)\cdot n!}\cdot\frac{n!}{n^n}=\]
\[=\frac{(n+1)^n}{n^n}=\left(\frac{n+1}{n}\right)^n=\left(1+\frac{1}{n}\right)^n\]
Entonces \(\lim_{n\to\infty}\frac{x_{n+1}}{x_n}=\lim_{n\to\infty}\left(1+\frac{1}{n}\right)^n=e\).
Como \(e>1\), el criterio del cociente nos dice que la sucesión \(x_n=\frac{n^n}{n!}\) tiende a más infinito.
Es decir \(\lim_{n\to\infty}\frac{n^n}{n!}=+\infty\)

Con este resultado alcanza para asegurar que el cociente
\[\frac{a_{n+1}}{a_n}=\frac{n^n}{n!}+\frac{3^n}{n!}\]
tiende a más infinito y así, por el Criterio del cociente podemos afirmar que \(\lim_{n\to\infty}a_n=+\infty\).

De todas maneras estudiemos el segundo término:

\(y_n=\frac{3^n}{n!}\).
El cociente de D ́Alembert es en este caso:
\[\frac{y_{n+1}}{y_n}=\frac{\frac{3^{n+1}}{(n+1)!}}{\frac{3^n}{n!}}=\frac{3^{n+1}}{(n+1)!}\cdot\frac{n!}{3^n}=\frac{3}{n+1}\]
Entonces \(\lim_{n\to\infty}frac{y_{n+1}}{y_n}=\lim_{n\to\infty}\frac{3}{n+1}=0\).
Como \(0<1\) el criterio del cociente nos dice que \(\lim_{n\to\infty}y_n=0\).

En consecuencia \(\lim_{n\to\infty}\frac{a_{n+1}}{a_n}=\lim_{n\to\infty}\frac{n^n}{n!}+\frac{3^n}{n!}=+\infty\).

El criterio del cociente afirma que \(\lim_{n\to\infty}a_n=+\infty\).

Estamos en condiciones de calcular el límite que nos pide el problema, teniendo en cuenta que \(\lim_{n\to\infty}\frac{1}{a_n}=0\).
\[\lim_{n\to\infty}2^{1+\frac{1}{a_n}}=\lim_{n\to\infty}2^{1+\lim_{n\to\infty}\frac{1}{a_n}}=2^{1+0}=2\]
\subsubsection{La raíz cuadrada de 2}
Comenzamos esta unidad planteando el problema de diseñar un algoritmo para calcular la raíz cuadrada de un número utilizando las cuatro operaciones elementales de la aritmética.
Tomamos el caso particular \(a=\sqrt{2}\).

Con la ayuda de una original \textit{idea geométrica}, llegamos a conjeturar que las bases de los rectángulos aproximaban a \(\sqrt{2}\).

Las medidas de las bases de los sucesivos rectángulos vienen dados por la sucesión dada en forma recurrente por la fórmula
\[x_1=1,x_{n+1}=\frac{1}{2}\left(x_n+\frac{2}{x_n}\right)\]
Esta sucesión resulta ser una sucesión de términos positivos que podemos ver que está acotada inferiormente y es decreciente, usando la desigualdad entre el promedio geométrico y el promedio aritmético que recordamos en el recuadro.
(Si \(a>0,b>0\) vale \(\frac{a+b}{2}\geq\sqrt{ab}\))

Para ver que está acotada inferiormente ponemos \(a=x_n\) y \(b=\frac{2}{x_n}\) en la desigualdad y queda
\[x_{n+1}=\frac{1}{2}\left(x_n+\frac{2}{x_n}\right)\geq\sqrt{x_n\cdot\frac{2}{x_n}}=\sqrt{2}\quad n\geq1\]
Para ver que es decreciente analizamos el cociente de D ́Alembert como en ocasiones anteriores y volvemos a usar la desigualdad entre promedios y la acotación que acabamos de demostrar.
El cociente es
\[\frac{x_{n+1}}{x_n}=\frac{1}{2}\left(1+\frac{2}{x^2_n}\right)\]
Al poner en la desigualdad \(a=1\) y \(b=\frac{2}{x^2_n}\) se obtiene
\[\frac{x_{n+1}}{x_n}=\frac{1}{2}\left(1+\frac{2}{x^2_n}\right)\geq\sqrt{1\cdot\frac{2}{x^2_n}}=\frac{\sqrt{2}}{x_n}\leq1\quad n\geq2\]
Entonces, la sucesión de la medida de las bases de los rectángulos \(x_n\) es decreciente y acotadainferiormente.
Por el Teorema de las sucesiones monótonas podemos afirmar que existe \(\lim_{n\to\infty}x_n=L\).
Solo queda calcular \(L\).
Para ello, usamos el álgebra de límites en la definición recurrente de \(x_n\) y obtenemos
\[L=\lim_{n\to\infty}x_{n+1}=\lim_{n\to\infty}\frac{1}{2}\left(x_n+\frac{2}{x_n}\right)=\frac{1}{2}\left(L+\frac{2}{L}\right)\]
Es decir,
\[2L=L+\frac{2}{L}\]
o lo que es equivalente
\[L^2=2\Longleftrightarrow L=\sqrt{2}\text{ ó }L=-\sqrt{2}\]
Como la sucesión es de términos positivos, el límite no puede ser negativo (recordar la propiedad de conservación de signo).
Entonces, podemos asegurar que la solución verdadera es la positiva.
Es decir
\[\lim_{n\to\infty}x_n=\sqrt{2}\]
\end{document}
